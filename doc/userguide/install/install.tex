\chapter{Installation and Getting Help}

\section{Introduction}

Installation of PyLith on a desktop or laptop machine is, in most
cases, very easy. Binary packages have been created for the most
common platforms, including Linux, OSX, and Windows. Installation on
machines that are not compatible with any of these operating systems
requires building the software from the source code, which can be
difficult for inexperienced users.

\section{Getting Help}

Help is available via the CIG Short-Term Crustal Dynamics Mailing List
and CIG's issue tracking system.

\subsection{Requesting help}

The CIG Short-Term Crustal Dynamics Mailing List,
\email{cig-short@geodynamics.org}, is a mailing list dedicated to CIG
issues associated with short-term crustal dynamics, including the use
of PyLith. You can subscribe to the mailing list and view messages at
the \link{CIG website}{http://www.geodynamics.org}.

\subsection{Reporting bugs and errors}

CIG uses the \application{Roundup} for bug tracking. If you find a bug
in PyLith, please submit a bug report to the \link{CIG
\application{Roundup} system}{http://www.geodynamics.org/roundup}. Of
course, it is helpful to first check to see if someone else already
submitted a report related to the issue; one of the CIG developers may
have already posted a solution to the problem. You can reply to a
current issue by clicking on the issue title. To submit a new issue,
click on \guibutton{Create New} under \guimenu{Issues}.

\section{Installation}

Binary executables are available for three of the most widely used
platforms: 32-bit Linux, Mac OSX, and Windows. If your platform is not
compatible with any of these, you can build the software from the
source code.

\subsection{Linux}

Running the Linux binary version of PyLith requires a 32-bit
compatible machine with GLIBC 2.2 or later.

\begin{enumerate}
\item Download the \link{tarball}{http://crust.geodynamics.org/~leif/shipping/}
\item Unpack the tarball in a suitable location.
  \begin{screen}
    \prompt{bash~}\userinput{tar -zxvf pylith-0.8-linux-x86.tar.gz}
  \end{screen}
\item Add \directory{pylith-0.8-linux-x86/bin} to your \envvar{PATH}.
  You will likely want to add something like
  \begin{screen}
    PATH=\$\{PATH\}:\replaceable{replace\_with\_absolute\_path}/pylith-0.8-linux-x86/bin
  \end{screen}
  to your \filename{.bashrc} file (if you are using bash as your shell)
  or the equivalent to your \filename{.cshrc} file (if you are using
  tcsh as your shell).
\item Add \directory{pylith-0.8-linux-x86/lib}to your
  \envvar{LD\_LIBRARY\_PATH}. You will likely want to add something like
  \begin{screen}
    export LD\_LIBRARY\_PATH=\$\{LD\_LIBRARY\_PATH\}:\replaceable{replace\_with\_}
    \replaceable{absolute\_path}/pylith-0.8-linux-x86/lib
  \end{screen}
  to your \filename{.bashrc} file (if you are using bash as your
  shell) or the equivalent to your \filename{.cshrc</filename} file
  (if you are using tcsh as your shell).
\item To uninstall PyLith, simply remove the \directory{pylith-0.8-linux-x86}
  directory and all of its sub-directories.
\end{enumerate}

\subsection{Mac OSX}

The OSX binary version was built to run on the Macintosh PowerPC
architecture with OSX version 10.2 or later. The binary will run on
the Macintosh Intel architecture, but only in emulation mode (i.e.,
rather slowly).

\begin{enumerate}
\item Download the \link{disk image}{http://crust.geodynamics.org/~leif/shipping/}.
\item Double click on the disk image to mount the disk.
\item Double click on the disk and copy the \directory{PyLith} folder
  to a suiteable location.
\item To run PyLith, double click on the PyLith icon provided in the \directory{PyLith} folder.
\item To uninstall PyLith, simply drag the \directory{PyLith} folder
  to the trash.
\end{enumerate}

\subsection{Windows}

This Windows binary version of PyLith should be compatible with
Windows NT, Windows 2000, and Windows XP.

\begin{enumerate}
\item Download theq
  \link{installer}{http://crust.geodynamics.org/~leif/shipping/}.
\item Double click on the \filename{setup.exe} file you just
  downloaded and follow the instructions for installing.
\item To run PyLith, double click on the PyLith icon on the desktop or
  select
  \guimenu{Start}\guiselect\guimenuitem{All~Programs}\guiselect\guimenuitem{PyLith}\guiselect\guimenuitem{PyLith}.
\item To uninstall PyLith, simply select
  \guimenu{Start}\guiselect\guimenuitem{All
    Programs}\guiselect\guimenuitem{PyLith}\guiselect\guimenuitem{Uninstall~PyLith}.
\end{enumerate}

\section{Building using source tarball}

Building PyLith from the source code is not a trivial task because
PyLith depends on several other packages. In general, each package
must be compiled from source using compilers for each language that
are compatible with one another. The stable version of the PyLith
source code is available from the \link{Geodynamics subversion
  repository}{http://www.geodynamics.org:8080/cig/software/Repository/}
\directory{cig/short/3D/PyLith/branches/pylith-0.8}.

\subsection{System Requirements}

\begin{itemize}
\item Unix flavored operating system.
\item Fortran, C, and C++ compilers.
\item Python (2.3 or later)
\end{itemize}

The software must be built starting from the bottom of the dependency
list and working upwards. The steps below describe the recommended way
to build PyLith and the external packages on which it depends. Some of
the dependencies can be satisfied using precompiled binaries (e.g.,
RedHat and Fink packages). When considering whether to use a
precompiled binary package to satisfy any of the dependencies,
remember that all of the compilers and settings used in building the
code must be compatible.

\begin{enumerate}
\item Build PETSc and MPI.
  \begin{enumerate}
  \item ADD STUFF HERE (LINK TO PETSC SITE? GET MATT'S THOUGHTS ON HOW
    MUCH DETAIL TO PUT HERE)
  \end{enumerate}
\item Build Pythia.
  \begin{enumerate}
  \item ADD STUFF HERE
  \end{enumerate}
\item Build PyLith.
  \begin{enumerate}
  \item ADD STUFF HERE
  \end{enumerate}
\end{enumerate}

\section{Building using source repositories}

\begin{warning}
  Building PyLith using the source repositories is recommended
  only for expert users who are willing to work with a moving
  target and rebuild on a frequent basis. The installation
  instructions cover the basic steps and assume the user has
  experience building and installing software.
\end{warning}

The PyLith-0.8 source code is available from the \link{Geodynamics
  subversion
  repository}{http://www.geodynamics.org:8080/cig/software/Repository}
\directory{cig/short/3D/PyLith/branches/pylith-0.8}.

\subsection{System Requirements}

\begin{itemize}
\item Unix flavored operating system
\item Fortran, C, and C++ compilers
\item Python (2.3 or later)
\item Subversion
\item Mercurial (0.9 or later)
\end{itemize}

\begin{tip}
  Many flavors of Unix have Subversion packages. In most
  cases you do not need anything but the basic subversion
  package. You will likely have to build Mercurial from
  source, but this is a very easy task that takes only a
  couple of minutes.
\end{tip}

\subsection{Building the packages}

The software must be built starting from the bottom of the dependency
list and working upwards. The steps below describe the recommended way
to build PyLith and the external packages on which it depends.

\begin{enumerate}
\item Build \link{PETSc (developers
    version)}{http://www-unix.mcs.anl.gov/petsc/petsc-as/developers/index.html}
  and MPI.
  
  If you have an architecture optimized version of BLAS/LAPACK you
  should use those instead of asking PETSc to download and build one
  for you. In some cases, PETSc will find known optimized
  implementations of BLAS/LAPACK (e.g., vecLib on Mac OSX).
  
  If you have an MPI implementation installed, you can use it or let
  PETSc download and install one for you.

  \begin{tip}
    If you run into problems configuring or building PETSc, send {\em
      both} \filename{configure.log} and
    \filename{make\_\replaceable{PETSC\_ARCH}} to
    \email{petsc-maint@mcs.anl.gov}.
  \end{tip}

  \begin{enumerate}
  \item Pull the source code from ANL and place the source tree in a
    suitable location. The steps below will create a
    \directory{petsc-dev} sub-directory in the current directory.

    \begin{screen}
      \shellprompt\userinput{hg clone http://mercurial.mcs.anl.gov/petsc/petsc-dev}
      \shellprompt\userinput{cd petsc-dev/python}
      \shellprompt\userinput{hg clone http://mercurial.mcs.anl.gov/petsc/BuildSystem \
        BuildSystem}
    \end{screen}
    
  \item Set the \envvar{PETSC\_DIR} and \envvar{PETSC\_ARCH} environment
    variables. \envvar{PETSC\_DIR} corresponds to the top-level
    directory in the PETSc source tree and \envvar{PETSC\_ARCH} is a tag
    for this configuration of PETSc (e.g., linux\_gcc-4.0\_debug,
    darwin\_gcc-3.4\_opt, etc). You will want to set these environment
    variables in your \filename{.bashrc}, \filename{.cshrc}, or
    similar file.

    \begin{screen}
      \shellprompt\userinput{export PETSC\_DIR=\replaceable{replace\_with\_absolute\_path}/petsc-dev}
      \shellprompt\userinput{export PETSC\_ARCH=\replaceable{replace\_with\_PETSc\_arch\_tag}}
    \end{screen}
  \item Configure PETSc.
    
    Run \command{config/configure.py} with the appropriate arguments
    to configure PETSc for your computer. Run
    \command{config/configure.py --help} to see the long list of
    possible options. Building PETSc for use with PyLith requires the
    following arguments:

    \begin{screen}
      \option{--with-clanguage=c++}
      \option{--with-c-support}
      \option{--with-shared=1}
      \option{--with-boost=1}
      \option{--download-boost=1}
      \option{--with-sieve=1}
    \end{screen}
    
    Be sure to include flags indicating where MPI and
    BLAS/LAPACK are if you want to use a preinstalled
    implementation. If not, be sure to tell PETSc to
    download those packages. If you do not want build
    PETSc using the gnu compilers, be sure to let PETSc
    know which compilers you want to use instead.

    After several minutes, the configure script should
    finish and display the PETSc settings. Check to make
    sure these match your expectations before continuing.

  \item Build PETSc.
    \begin{screen}
      \shellprompt\userinput{make}
    \end{screen}
  \item Test PETSc installation.
    
    One of the tests will attempt to display some X windows. If X
    windows is not available from the shell in which you run
    \command{make test}, then that test will fail.

    \begin{screen}
      \shellprompt\userinput{make test}
    \end{screen}
    
  \item In the future, you will likely want to update the PETSc source
    tree to include bug fixes, new features etc.
    
    To update PETSc (use to \command{hg update -v} to see what files
    are being updated):

    \begin{screen}
      \shellprompt\userinput{cd petsc-dev}
      \shellprompt\userinput{hg pull}
      \shellprompt\userinput{hg update}
      \shellprompt\userinput{cd python/BuildSystem}
      \shellprompt\userinput{hg pull}
      \shellprompt\userinput{hg update}
    \end{screen}
  \end{enumerate}

\item Build Pythia.

  \begin{tip}
    If you run into problems configuring or building Pythia, send
    {\em both} the \filename{config.log} and
    \filename{make.log} files to
    \email{cig-short@geodynamics.org}. To create the
    \filename{make.log} file, run make via \command{make $>$ make.log}.
  \end{tip}

  \begin{enumerate}
  \item Pull the source code from CIG and place the source tree in a
    suitable location. The steps below will create a
    \directory{pythia-0.8} sub-directory in the current directory.

    \begin{screen}
      \shellprompt\userinput{svn co svn://geodynamics.org/cig/vendor/pythia [cont.]\\
        /v0.8/pythia-0.8}
    \end{screen}
    
  \item Configure Pythia.
    
    Run \command{configure} with the appropriate arguments.
    Run \command{configure --help} to see a list of possible
    options. You may want to specify compilers and/or an install
    location.

    \begin{screen}
      \shellprompt\userinput{cd pythia-0.8}
      \shellprompt\userinput{autoreconf -i}
      \shellprompt\userinput{./configure}
    \end{screen}

  \item Build Pythia.

    Run \command{make} and \command{make install}.

    \begin{screen}
      \shellprompt\userinput{make}
      \shellprompt\userinput{make install}
    \end{screen}
    
  \end{enumerate}

\item Build PyLith.

  \begin{tip}
    If you run into problems configuring or building PyLith, send
    {\em both} the \filename{config.log} and
    \filename{make.log} files to
    \email{cig-short@geodynamics.org}. To create the
    \filename{make.log} file, run make via \command{make $>$ make.log}.
  \end{tip}

  \begin{enumerate}
  \item Pull the source code from CIG and place the source tree in a
    suitable location. The steps below will create a
    \directory{pylith-0.8} sub-directory in the current directory.

    \begin{screen}
      \shellprompt\userinput{svn co svn://geodynamics.org/cig/short/3D/PyLith [cont.]\\
        /branches/pylith-0.8}
    \end{screen}

  \item Configure PyLith.
    
    Run \command{configure} with the appropriate arguments. Run
    \command{configure --help} to see a list of possible options. You
    may want to specify compilers and/or an install location. If you
    installed pythia-0.8 in a non-system location, you will need to
    set \envvar{CPPFLAGS} and \envvar{LDFLAGS} appropriately.

    \begin{screen}
      \shellprompt\userinput{cd pylith-0.8}
      \shellprompt\userinput{autoreconf -i}
      \shellprompt\userinput{./configure}
    \end{screen}
    
  \item Build PyLith.

    Run \command{make} and \command{make install}.

    \begin{screen}
      \shellprompt\userinput{make}
      \shellprompt\userinput{make install}
    \end{screen}
  \end{enumerate}
\end{enumerate}
