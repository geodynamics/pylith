\section{PETSc Finite-Element Implementation}

Formulating the weak form of the governing equation in terms of
point-wise functions allows the PyLith implementation of the equations
to be done at a rather high level. Most of the finite-element details
are encapsulated in PETSc routines that compute the integrals and
solve the system of equations. In fact, adding materials and boundary
conditions rarely requires calling any PETSc finite-element
functions. Instead, a new material supplies and registers the
point-wise functions. The new material may need to add to the library
of solution subfields and auxiliary subfields, but adding these fields
is also done at a high-level.

The remainder of this section discusses three aspects of the
finite-element implementation handled by PyLith to give you a peek of
what is doing on under the hood.

\subsection{Meshing}

We have chosen to employ the \textit{cohesive element} formulation for
faults, in order to apply separate constitutive relations to the fault
surface. Lets examine why we made this choice. We would like the two
sides of the faults to operate independently in some sense, but
maintain a coupling (prescribed or dynamic). Thus we would like the
displacement of the two sides to be independent variables, constrained
by the coupling equation. We could have maintained the same mesh
topology, but added extra variables on the vertices, edges, and faces
on the fault surface. However, this would change assembly on all cells
along the fault, and greatly complicate optimization and maintenance
of the assembly code. So we pushed the complication to the mesh
topology.

\subsection{\object{DMPlex}}

The finite-element mesh is stored as a \object{DMPlex} object. This is
a particular implementation of the PETSc Data Management (\object{DM},
renamed \object{PetscDM} within PyLith) object. Within a
\object{DMPlex} object vertices, edges, faces, and cells are all
called points. The points are numbered sequentially, beginning with
cells, followed by vertices, edges, and then faces. Treating all
topological pieces of the mesh the same way, as points in an abstract
graph, allows us to write algorithms which apply to very different
meshes without change. For example, we can write a fintie element
assembly loop that applies to meshes of any dimension, with any cell
shape, with (almost) any finite element.

\subsubsection{Point Depth and Height}

In general, vertices are at a {\bf depth} of 0 and cells are at the maximum depth. Similarly, cells are at a {\bf
height} of 0 and vertices are at the maximum depth. Notice that depth and height correspond the the usual dimension and
codimension. For a 3-D mesh with vertices, edges, faces, and cells:

\begin{table}
\begin{center}
\begin{tabular}{lcc}
\hline
Point type & Depth & Height \\
\hline
Vertices  &  0  &  3 \\
Edges  &  1  &  2 \\
Faces  &  2  &  1 \\
Cells  &  3  &  0 \\
\hline
\end{tabular}
\end{center}
\caption{Depth and height of various topological pieces.}
\end{table}

For a boundary mesh, we currently store the full set (vertices, edges, faces, and cells). Obviously for 2-D meshes, the
boundary mesh doesn't contain "volume" cells, but just vertices, edges, and faces. This means the "boundary" cells are
at a height of 1 and maximum depth - 1 .

\todo{brad}{Add diagram of finite-element mesh in DMPlex form.}

\subsection{\object{PetscSection} and \object{PetscVec}}

We store a field over the mesh in a vector (PETSc \object{Vec} object,
renamed \object{PetscVec} in PyLith). The \object{PetscSection} object
describes the layout of the vector over the DMPlex object. The vector
may hold multiple subfields, each with its own discretization. The
chart of the \object{PetscSection} defines the range of points
(minimum and maximum) over which the section is defined. For each
point in the chart, the section holds the number of degrees of freedom
it has and the offset in the \object{PetscVec} for its first degree of
freedom. The section also holds the number of degrees of freedom and
offset for each individual subfield within the section.

Because the \object{PetscSection} knows only about points, and not
about topolgoy or geometry, it allows us to express the mathematical
concepts without getting lost in the details. For example, a $P_1$ 3D
vector field assigns 3 dofs to each vertex. This same section could be
used to layout a field over a circle, surface of a cylinder, M\"obius
strip, sphere, cube, ball, or Pylith mesh. It separates the layout of
data over a mesh from the actual storage of values, which is done by a
plain old PETSc \object{Vec}. The section tells you how to look up the
values, which are ``attached'' to a piece of the mesh, inside the
vector. For example, you can ask for the offset into the vector for
the values associated with an edge in the mesh, and how many dofs
there are on the edge.

The \object{PetscSection} also includes the information about the
constrained degrees of freedom. We refer to the \object{PetscVec} with
the constrained degrees of freedom included as the {\em local} vector
and the \object{PetscVec} with the constrained degrees of freedom
removed as the {\em global} vector. Constraints often arise from
Dirichlet boundary conditions, which change directly the basis
functions of the approximating space, but can also arise from
algebraic relations between unknowns. A local vector is often used for
assembly of the residual vector and Jacobian matrix, since we need the
boundary values in order to compute those integrals. Whereas global
vectors are used for the algebraic solver since we do not want
unknowns fixed by constraints to participate in the solve.

\todo{brad}{Add ASCII view of PetscSection corresponding to
  finite-element mesh.}

\subsection{Integration}

Integration involves integrals over the domain or the boundary of the
domain. These two operations are done by different PETSc functions.

\begin{description}
\item[\object{DMPlexComputeResidual\_Internal}] Compute the
  contribution to the LHS or RHS residual for a single material. This
  function and the functions it calls handle looping over the cells in
  the material, integrating the weak form for each of the fields, and
  adding them to the residual.

  A more appropriate name would be
  \object{DMPlexComputeResidualSingle} although that may change once
  we have a separate \object{PetscDM} for each material.
%
\item[\object{DMPlexComputeBdResidualSingle}] Compute the contribution
  to the LHS or RHS residual for a single boundary condition. This
  function and the functions it calls handle looping over the
  boundary, integrating the weak form for each of the fields, and
  adding them to the residual.
\end{description}

\subsection{Projection}

Input and output often involve projecting fields to/from the
finite-element space. PETSc provides a family of functions for
this. We generally use two of these, one for analytical functions and
one for discretized fields. Projection may be a misleading term here,
since we are not refering to the common $L_2$ projection, but rather
interpolation of the function by functions in our finite element
space.

How do we find the interpolant? Lets start with the simple example of
Fourier analysis, which most people have experience with. If I want
the Fourier interpolant $\tilde f$ for a given function $f$, then I
need to determine its Fourier coefficients,
\begin{align}
  \tilde f = \sum_k f_k e^{i k x}.
\end{align}
This is not difficult because the basis functions in the Fourier
representation are orthogonal
\begin{align}
  \int^{2\pi}_0 e^{-i m x} e^{i k x} = 2\pi \delta_{km}.
\end{align}
Thus, we just multiply by the conjugate of the basis function and integrate,
\begin{align}
  \int^{2\pi}_0 e^{-i m x} \tilde f &= \int^{2\pi}_0 e^{-i m x} \sum_k f_k e^{i k x}, \\
                                  &= \sum_k f_k \int^{2\pi}_0 e^{-i m x} e^{i k x}, \\
                                  &= \sum_k f_k 2\pi \delta_{km}, \\
                                  &= 2\pi f_m, \\
\end{align}
and we have our coefficient $f_m$. The finite element basis
$\{\phi_i\}$ is not orthogonal, so we have an extra step. We could
take the inner product of $f$ with all the basis functions, and then
sort out the dependecies by solving a linear system (the mass matrix),
which is what happens in $L_2$ projection. However, suppose we have
another basis $\{\psi_i\}$ of linear functionals which is
\textit{biorthogonal} to $\{\phi_i\}$, meaning
\begin{align}
  \psi_i(\phi_j) = \delta_{ij}.
\end{align}
Then we can easily pick out the coefficient of $\tilde f$ by acting
with the corresponding basis functional,
\begin{align}
  \tilde f &= \sum_k f_k \phi_k(x),\\
  \psi_i(\tilde f) &= \psi_i(\sum_k f_k \phi_k(x)),\\
  \psi_i(\tilde f) &= \sum_k f_k \psi_i(\phi_k(x)),\\
  \psi_i(\tilde f) &= \sum_k f_k \delta_{ik},\\
  \psi_i(\tilde f) &= f_i,\\
\end{align}
where we used the linearity of the $\psi_i$ in the third step. We will
call $\{\phi_i\}$ the \textit{primal} basis, and $\{\psi_i\}$ the
\textit{dual} basis. We note that if $f$ does not lie in our
approximation space spanned by $\{\phi_i\}$, then interpolation is not
equivalent to $L_2$ projection. This will not usually be important for
our purposes.

\begin{description}
\item[\object{DMProjectFunctionLocal}] Project an analytical function
  into the given finite-element space.
\item[\object{DMProjectFieldLocal}] Project a discretized field in one
  finite-element space into another finite-element space.
\end{description}

\begin{cplusplus}[PetscPointFunc Interface]
/** Interface for PETSc point-wise functions.
 *
 * @param[in] dim Spatial dimension of problem.
 * @param[in] Nf Number of subfields in the field.
 * @param[in] NfAux Number of subfields in the auxiliary field.
 * @param[in] uOff Offset into u and u_t[] for each subfield.
 * @param[in] uOff_x Offset into u_x for each subfield.
 * @param[in] u Values of each subfield at the current point.
 * @param[in] u_t Time derivative of each subfield at the current point.
 * @param[in] u_x Gradient of each subfield at the current point.
 * @param[in] aOff Offset into u and u_t[] for each auxiliary subfield.
 * @param[in] aOff_x Offset into u_x for each auxiliary subfield.
 * @param[in] a Values of each auxiiliary subfield at the current point.
 * @param[in] a_t Time derivative of each auxiliary subfield at the current point.
 * @param[in] a_x Gradient of each auxiliary subfield at the current point.
 * @param[in t Current time.
 * @param[in] x Coordinates of current point.
 * @param[in] numConstants Number of constant parameters.
 * @param[in] constants Constant parameters.
 * @param[out] f Output values at current point.
 */
 void
 func(PetscInt dim,
      PetscInt Nf,
      PetscInt NfAux,
      const PetscInt uOff[],
      const PetscInt uOff_x[],
      const PetscScalar u[],
      const PetscScalar u_t[],
      const PetscScalar u_x[],
      const PetscInt aOff[],
      const PetscInt aOff_x[],
      const PetscScalar a[],
      const PetscScalar a_t[],
      const PetscScalar a_x[],
      PetscReal t,
      const PetscReal x[],
      PetscScalar f0[]);
\end{cplusplus}

\begin{cplusplus}[PetscPointJac Interface]
/** Interface for PETSc point-wise Jacobians.
 *
 * This is identical to the PetscPointFunc with the addition of the
 * u_tShift argument.
 *
 * @param[in] u_tShift The multiplier for dF/dU_t.
 */
 void
 func(PetscInt dim,
      PetscInt Nf,
      PetscInt NfAux,
      const PetscInt uOff[],
      const PetscInt uOff_x[],
      const PetscScalar u[],
      const PetscScalar u_t[],
      const PetscScalar u_x[],
      const PetscInt aOff[],
      const PetscInt aOff_x[],
      const PetscScalar a[],
      const PetscScalar a_t[],
      const PetscScalar a_x[],
      PetscReal t,
      PetscReal u_tShift,
      const PetscReal x[],
      PetscScalar Jf0[]);
\end{cplusplus}

\begin{cplusplus}[PetscDSSetResidual Function]
/** Set point-wise functions for LHS or RHS residual.
 *
 * @param[in] prob PetscDS associated with solution field.
 * @param[in] f Index of solution subfield for residual term.
 * @param[in] f0 Point-wise function for f0/g0 term in weak form.
 * @param[in] f1 Point-wise function for f1/g1 term in weak form.
 *
 * @returns PETSc error code (0 indicates no errors).
 */
 PetscErrorCode
 PetscDSSetResidual(PetscDS prob,
                    PetscInt f,
                    PetscPointFunc f0,
                    PetscPointFunc f1);
\end{cplusplus}

\begin{cplusplus}[PetscDSSetJacobian Function]
/** Set point-wise functions for LHS or RHS Jacobian.
 *
 * @param[in] prob PetscDS associated with solution field.
 * @param[in] f Index of trial subfield for Jacobian term.
 * @param[in] g Index of field subfield for Jacobian term.
 * @param[in] Jf0 Point-wise function for Jf0/Jg0 term in weak form.
 * @param[in] Jf1 Point-wise function for Jf1/Jg1 term in weak form.
 * @param[in] Jf2 Point-wise function for Jf2/Jg2 term in weak form.
 * @param[in] Jf3 Point-wise function for Jf3/Jg3 term in weak form.
 *
 * @returns PETSc error code (0 indicates no errors).
 */
 PetscErrorCode
 PetscDSSetResidual(PetscDS prob,
                    PetscInt f,
                    PetscInt g,
                    PetscPointJac Jf0,
                    PetscPointJac Jf1,
                    PetscPointJac Jf2,
                    PetscPointJac Jf3);
\end{cplusplus}



% End of file
