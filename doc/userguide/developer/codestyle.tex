\chapter{Coding Style}
\label{cha:code:style}

This chapter will be an appendix to the PyLith Manual.

There are a number of standard coding styles for programming
languages, notable PEP8 for Python. For PyLith we try to be consistent
in naming conventions across Python and C++ while following a subset
of the used in PETSc and PEP8 with documentation styles consistent
with Doxygen.

The principal guidelines include:
\begin{itemize}
\item Class names use upper camel case, e.g., \object{TimeDependent}.
\item Public method names use camel case, e.g.,
  \object{computeRHSResidual()}.
\item Protected and private method names use camel case preceded by an
  underscore, e.g., \object{\_setFEKernelsRHSResidual}.
\item In C++ data members are private and use camel case preceded by
  an underscore, e.g., \object{\_gravityField}.
\item In Python data members are public and use camel case, e.g., \object{gravityField}.
\item Local variables use camel case, e.g., \object{numIntegrators}.
\end{itemize}

\section{C/C++}

\subsection{Object Definition Files}

Object definition (header) files use the \filename{.hh} suffix. C
header files use the \filename{.h} suffix.

\todo{brad}{Insert example here}

\subsection{Object Implementation Files}

Object implementation files use the \filename{.cc} suffix. Inline
implementation files use the \filename{icc} suffix and are included
from the definition files. C implementation files use the \object{.c}
suffix.

To facilitate debugging and error messages, we use the following
macros:
\begin{description}
\item[PYLITH\_METHOD\_BEGIN] Use this macro at the beginning of all
  but methods using any PETSc routines and most other methods except
  trivial or inline methods.
\item[PYLITH\_METHOD\_END] Use thie macro at the end of all methods
  using PYLITH\_METHOD\_BEGIN that return void.
\item[PYLITH\_RETURN\_END] Use thie macro at the end of all methods
  using PYLITH\_METHOD\_BEGIN that return non-void values.
\item[PYLITH\_JOURNAL\_DEBUG] Use this macro immediate after
  PYLITH\_METHOD\_BEGIN in all methods for objects that {\em do not} have a corresponding
  Pyre component.
\item[PYLITH\_CHECK\_ERROR] after {\em every} call to a PETSc function
  to check the return value.
\item[PYLITH\_COMPONENT\_DEBUG] Use this macro immediate after
  PYLITH\_METHOD\_BEGIN in all methods for objects that {\em do} have
  a corresponding Pyre component. The class {\em must} inherit from
  \object{pylith::utils::PyreComponent} and non-abstract classes
  should call \object{PyreComponent::name()} in the constructor. We
  recommend using a static data member for the name with the lowercase
  name matching the Pyre component, e.g., timedependent.
\end{description}

\todo{brad}{Insert example here}



\section{Python}

\todo{brad}{Insert example here}
