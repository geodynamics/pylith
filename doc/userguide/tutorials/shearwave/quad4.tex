
\section{\label{sec:tutorial:shearwave:quad4}3D Bar Discretized with Quadrilaterals}

PyLith features discussed in this tutorial:
\begin{itemize}
\item Dynamic solution
\item CUBIT mesh format
\item Absorbing dampers boundary conditions
\item Kinematic fault interface conditions
\item Dynamic fault interface conditions
\item Plane strain linearly elastic material
\item VTK output
\item Linear quadrilateral cells
\item SimpleDB spatial database
\item ZeroDispDB spatial database
\item UniformDB spatial database
\end{itemize}
All of the files necessary to run the examples are contained in the
directory \texttt{examples/bar\_shearwave/quad4.}


\subsection{Mesh Generation}

The mesh is a simple rectangular prism 8 km by 400 m by 400 m (Figure
\ref{fig:shearwave:quad4:mesh}). We provide documented CUBIT journal
files in \texttt{examples/bar\_shearwave/quad4.} We first create the
geometry, mesh the domain using quadrilateral cells, and then create
blocks and nodesets associated with the materials and boundary conditions.

\noindent \begin{center}
\begin{figure}
\begin{centering}
\includegraphics[scale=0.5]{tutorials/shearwave/figs/quad4mesh}
\par\end{centering}

\caption{Mesh composed of hexahedral cells generated by CUBIT used for the
example problem.\label{fig:shearwave:quad4:mesh}}
\end{figure}

\par\end{center}


\subsection{Kinematic Fault (Prescribed Slip)}

The simulation parameters match those in the tri3, tet4, and hex8
examples. Using four-point quadrature permits use of a time step of
1/20 s, which is slightly larger than the time step of 1/30 s used
in the tri3 and tet4 simulations. In contrast to the tri3, tet4, and
hex8 shear wave examples which only contained a single simulation
in a directory, in this example we consider several different simulations.
Consequently, we separate the parameters into multiple \texttt{.cfg}
files. The common parameters are placed in \texttt{pylithapp.cfg}
with the parameters specific to the kinematic fault (prescribed rupture)
example in \texttt{prescribedrup.cfg}. To run the problem, simply
run PyLith via:
\begin{lyxcode}
pylith~prescribedrup.cfg
\end{lyxcode}
The VTK files will be written to the \texttt{output} directory with
the prefix \texttt{prescribedrup}. The output includes the displacement
field over the entire domain at every other time step (0.10 s), the
slip and traction vectors on the fault surface in along-strike and
normal directions at every other time step (0.10 s), and the strain
and stress tensors for each cell at every 20th time step (1.0 s).
If the problem ran correctly, you should be able to generate a figure
such as Figure \ref{fig:shearwave:quad4:kinematic}, which was generated
using ParaView.

\noindent \begin{center}
\begin{figure}
\begin{centering}
\includegraphics[scale=0.5]{tutorials/shearwave/figs/quad4kinematic30}
\par\end{centering}

\caption{Displacement field in the bar at 3.0 s. Deformation has been exaggerated
by a factor of 800.\label{fig:shearwave:quad4:kinematic}}
\end{figure}

\par\end{center}


\subsection{Dynamic Fault (Spontaneous Rupture)}

In this set of examples we replace the kinematic fault interface with
the dynamic fault interface, resulting in fault slip controlled by
a fault-constitutive model. See Section \ref{sec:fault:constitutive:models}
for detailed information about the fault constitutive models available
in PyLith. Because this is a dynamic simulation we want the generated
shear wave to continue to be absorbed at the ends of the bar, so we
drive the fault by imposing initial tractions directly on the fault
surface rather than through deformation within the bar. We impose
initial tractions (75 MPa of right-lateral shear and 120 MPa of compression)
plus a temporal variation (smoothly increasing from 0 to 25 MPa of
right-lateral shear) similar to what would be used in a 2-D or 3-D
version. While the magnitude of these stresses is reasonable for tectonic
problems, they give rise to very large slip rates in this 1-D bar.
The temporal variation, as specified via the \texttt{traction\_change.timedb}
file, has the functional form:
\begin{equation}
f(t)=\begin{cases}
\exp\left(\frac{\left(t-t_{n}\right)^{2}}{t\left(t-2t_{n}\right)}\right), & 0<t\le t_{n}\\
1, & t>t_{n}
\end{cases}
\end{equation}
where $t_{n}$ = 1.0 s. We request that the fault output include the
initial traction value and the slip, slip rate, and traction fields:
\begin{lyxcode}
{[}pylithapp.timedependent.interfaces.fault.output{]}

vertex\_info\_fields={[}traction\_initial\_value{]}

vertex\_data\_fields~=~{[}slip,slip\_rate,traction{]}
\end{lyxcode}
The steady-state solution for this problem is constant velocity and
slip rate with uniform strain within the bar. A Python script, \texttt{analytical\_soln.py},
is included for computing values related to the steady-state solution.


\subsubsection{Dynamic Fault with Static Friction}

The parameters specific to this example involve the static friction
fault constitutive model. We set the fault constitutive model via
\begin{lyxcode}
{[}pylithapp.timedependent.interfaces.fault{]}

friction~=~pylith.friction.StaticFriction
\end{lyxcode}
and use a UniformDB to set the static friction parameters. We use
a coefficient of friction of 0.6 and no cohesion (0 MPa). The parameters
specific to this example are in \texttt{spontaneousrup\_staticfriction.cfg},
so we run the problem via:
\begin{lyxcode}
pylith~spontaneousrup.cfg~spontaneousrup\_staticfriction.cfg
\end{lyxcode}
The VTK files will be written to the \texttt{output} directory with
the prefix \texttt{staticfriction}. The output includes the displacement
and velocity fields over the entire domain at every other time step
(0.10 s), the slip, slip rate, and traction vectors on the fault surface
in along-strike and normal directions at every other time step (0.10
s), and the strain and stress tensors for each cell at every 20th
time step (1.0 s). If the problem ran correctly, you should be able
to generate a figure such as Figure \ref{fig:shearwave:quad4:staticfriction},
which was generated using ParaView. The steady-state solution is a
constant slip rate of 22.4 m/s, a shear traction of 72.0 MPa on the
fault surface, a uniform shear strain of 5.6e-3 in the bar with uniform,
and constant velocities in the y-direction of +11.2 m/s and -11.2
m/s on the -x and +x sides of the fault, respectively.

\noindent \begin{center}
\begin{figure}
\begin{centering}
\includegraphics[scale=0.5]{tutorials/shearwave/figs/quad4staticfriction30}
\par\end{centering}

\caption{Velocity field in the bar at 3.0 s for the static friction fault constitutive
model. Deformation has been exaggerated by a factor of 20.\label{fig:shearwave:quad4:staticfriction}}
\end{figure}

\par\end{center}


\subsubsection{Dynamic Fault with Slip-Weakening Friction}

The parameters specific to this example are related to the use of
the slip-weakening friction fault constitutive model (see Section
\ref{sec:fault:constitutive:models}). We set the fault constitutive
model via
\begin{lyxcode}
{[}pylithapp.timedependent.interfaces.fault{]}

friction~=~pylith.friction.SlipWeakening
\end{lyxcode}
and use a UniformDB to set the slip-weakening friction parameters.
We use a static coefficient of friction of 0.6, a dynamic coefficient
of friction of 0.5, a slip-weakening parameter of 0.2 m, and no cohesion
(0 MPa). The fault constitutive model is associated with the fault,
so we can append the fault constitutive model parameters to the vertex
information fields:
\begin{lyxcode}
{[}pylithapp.timedependent.interfaces.fault.output{]}

vertex\_info\_fields~=~{[}strike\_dir,normal\_dir,initial\_traction,static\_coefficient,~\\
dynamic\_coefficient,slip\_weakening\_parameter,cohesion{]}
\end{lyxcode}
The parameters specific to this example are in \texttt{spontaneousrup\_slipweakening.cfg},
so we run the problem via:
\begin{lyxcode}
pylith~spontaneousrup.cfg~spontaneousrup\_slipweakening.cfg
\end{lyxcode}
The VTK files will be written to the \texttt{output} directory with
the prefix \texttt{slipweakening}. If the problem ran correctly, you
should be able to generate a figure such as Figure \ref{fig:shearwave:quad4:slipweakening},
which was generated using ParaView. The steady-state solution is a
constant slip rate of 32.0 m/s and shear traction of 60.0 MPa on the
fault surface, a uniform shear strain of 8.0e-3 in the bar with uniform,
constant velocities in the y-direction of +16.0 m/s and -46.0 m/s
on the -x and +x sides of the fault, respectively.

\noindent \begin{center}
\begin{figure}
\begin{centering}
\includegraphics[scale=0.5]{tutorials/shearwave/figs/quad4slipweakening30}
\par\end{centering}

\caption{Velocity field in the bar at 3.0 s for the slip-weakening friction
fault constitutive model. Deformation has been exaggerated by a factor
of 20.\label{fig:shearwave:quad4:slipweakening}}
\end{figure}

\par\end{center}


\subsubsection{Dynamic Fault with Rate-State Friction}

The parameters specific to this example are related to the use of
the rate- and state-friction fault constitutive model (see Section
\ref{sec:fault:constitutive:models}). The evolution of the state
variable uses the ageing law. We set the fault constitutive model
and add the state variable to the output via
\begin{lyxcode}
{[}pylithapp.timedependent.interfaces.fault{]}

friction~=~pylith.friction.RateStateAgeing~\\


{[}pylithapp.timedependent.interfaces.fault.output{]}~\\
vertex\_data\_fields~=~{[}slip,~slip\_rate,~traction,~state\_variable{]}~
\end{lyxcode}
and use a UniformDB to set the rate-state friction parameters. We
use a reference coefficient of friction of 0.6, reference slip rate
of 1.0e-6 m/s, characteristic slip distance of 0.02 m, coefficients
a and b of 0.008 and 0.012, and no cohesion (0 MPa). We set the initial
value of the state variable so that the fault is in equilibrium for
the initial tractions. The parameters specific to this example are
in \texttt{spontaneousrup\_ratestateageing.cfg}, so we run the problem
via:
\begin{lyxcode}
pylith~spontaneousrup.cfg~spontaneousrup\_ratestateageing.cfg
\end{lyxcode}
The VTK files will be written to the \texttt{output} directory with
the prefix \texttt{ratestateageing}. If the problem ran correctly,
you should be able to generate a figure such as Figure \ref{fig:shearwave:quad4:ratestateageing},
which was generated using ParaView. The steady-state solution is a
constant slip rate of 30.0 m/s and shear traction of 63.7 MPa on the
fault surface, a uniform shear strain of 7.25e-3 in the bar with uniform,
constant velocities in the y-direction of +15.0 m/s and -15.0 m/s
on the -x and +x sides of the fault, respectively.

\noindent \begin{center}
\begin{figure}
\begin{centering}
\includegraphics[scale=0.5]{tutorials/shearwave/figs/quad4ratestateageing30}
\par\end{centering}

\caption{Velocity field in the bar at 3.0 s for the rate- and state-friction
fault constitutive model. Deformation has been exaggerated by a factor
of 20.\label{fig:shearwave:quad4:ratestateageing}}
\end{figure}

\par\end{center}
