\subsection{xx.skew}

The \filename{xx.skew} file specifies local coordinate systems for
nodes. The local coordinate system is specified using two Euler angles
that rotate the local coordinate system to the global coordinate
system.

The applied coordinate rotations apply to all boundary conditions
associated with the nodes listed in the file. These are useful, for
example, if it is desired to apply boundary conditions in a direction
normal or tangential to a side of the mesh when the side does not
align with the global coordinate directions.  Similarly, skew
conditions could be used when specifying slip on a fault that lies at
an angle to the global coordinates.

\begin{figure}
  \begin{center}
\begin{verbatim}
# File containing local nodal coordinates.
#
# Comment lines begin with '#'
#
# First, specify units for rotations.
#
rotation_units = degree
#
#
# Columns:
#   (1) Node number
#   (2) Euler angle for rotation in the x-y plane
#   (3) Euler angle for rotation in the x-z plane
#
68    12.3   4.2
132  -12.3  -4.2
\end{verbatim}
    \ldots
    \caption{Format of \filename{xx.time} files.}
  \end{center}
\end{figure}
