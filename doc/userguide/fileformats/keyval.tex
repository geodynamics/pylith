\subsection{xx.keyval}

The \filename{xx.keyval} file specifies some simple parameter
settings.

\subsubsection{Winkler forces}

Scaling factors can be applied to Winkler forces, permitting a quick
and easy way to change the density or gravitational acceleration when
Winkler forces are used to simulate gravity.

\paragraph{Quadrature order}

\begin{description}
\item[Full] Quadrature order that should give the exact element
  matrices when the elements are geometrically undistorted.
\item[Reduced] Quadrature order that is one order less than full
  quadrature. Note that for linear tetrahedra full and reduced
  quadrature are equivalent (single integration point).

  \begin{warning}
    Use with caution as reduced quadrature can lead to numerical
    instabilities.
  \end{warning}
  
\item[Selective] Uses Hughes' b-bar formulation to perform reduced
  quadrature on the dilatational parts of the strain-displacement
  matrix.  This can be useful in nearly-incompressible problems.
\end{description}

\paragraph{Prestresses}

Gravitational prestresses can be computed automatically. In such
cases, the elastic properties in the prestress calculation can be set
to uniform values independent of the parameters for any of the
material models. When gravity is being used and prestresses are not
computed automatically, each prestress component can be scaled
independently. Reading prestresses from files is presently disabled.

\begin{figure}
\begin{center}
\begin{verbatim}
# Simple parameter values for various PyLith settings. Defaults are
# listed.
#
# Scaling factors applied to Winkler forces.
#
winklerScaleX = 1.0
winklerScaleY = 1.0
winklerScaleZ = 1.0
#
# Scaling factors applied to differential Winkler forces.
#
winklerSlipScaleX = 1.0
winklerSlipScaleY = 1.0
winklerSlipScaleZ = 1.0
#
# Stress integration and numerical computation of the tangent 
# material matrix.  Default values should be reasonable for most cases.
#
stressTolerance = 1.0e-12*Pa
minimumStrainPerturbation = 1.0e-7
initialStrainPerturbation = 1.0e-1
#
# Specify whether to use the solution from the previous time step as
# the starting guess for the elastic solution in the current time step.
# This feature has not been tested.
#
usePreviousDisplacementFlag = 0
#
# Quadrature order for the problem.
#
quadratureOrder = "Full"
#
# Gravitational acceleration in each direction.
#
gravityX = 0.0*m/(s*s)
gravityY = 0.0*m/(s*s)
gravityZ = 0.0*m/(s*s)
#
# Factors controlling computation of prestresses.
#
prestressAutoCompute = False
prestressAutoChangeElasticProperties = False
prestressAutoComputePoisson = 0.49
prestressAutoComputeYoungs = 1.0e30*Pa
#
prestressScaleXx = 1.0
prestressScaleYy = 1.0
prestressScaleZz = 1.0
prestressScaleXy = 1.0
prestressScaleXz = 1.0
prestressScaleYz = 1.0
#
# Unit numbers used in Fortran code.  These defaults should work for
# most Unix systems, but may be altered if necessary.
#
f77StandardInput = 5
f77StandardOutput = 6
f77FileInput = 10
f77AsciiOutput = 11
f77PlotOutput = 12
f77UcdOutput = 13
\end{verbatim}
  \caption{Format of \filename{xx.keyval} files.}
  \end{center}
\end{figure}
