\subsection{xx.statevar}

The \filename{xx.statevar} file specifies which state variables are to
be included in the output of the elastic and time dependent solutions.

\begin{figure}
  \begin{center}
\begin{verbatim}
# File specifying which state variables to output.
#
# Comment lines begin with '#'
#
# State variables occur in groups of 6, corresponding to the number of
# stress/strain components. The present groups are:
#   1-6: Cauchy stress
#   7-12: Total strain
#   13-18: Viscous strain
#   18-24: Plastic strain
#
# Lines:
#   (1) Total accumulated values for the current time step
#   (2) Incremental values (previous to current)
#   (3) Rate values (previous to current)
#
# Columns (per line):
#   (1) Number of state variables to output (0 &le; value &le; 24)
#   (2)+ State variable number to output (1 &le; value &le; 24)
#
    12   1   2   3   4   5   6   7   8   9  10  11  12
    12   1   2   3   4   5   6   7   8   9  10  11  12
    0
\end{verbatim}
    \caption{Format of \filename{xx.statevar} files.}
  \end{center}
\end{figure}
