\subsection{xx.time}

The \filename{xx.time} file specifies the time stepping parameters for
the simulation.

\begin{warning}
  The convergence criteria depend on the type of solution and material
  models. The time step for a linear elastic problem is much different
  than that for a nonlinear or time-dependent problem.
\end{warning}

\begin{figure}
  \begin{center}
\begin{verbatim}
# File containing time stepping parameters.
#
# Comment lines begin with '#'
#
# First, specify units used in values with dimensions of time.
#
time_units = year
#
#
# Time stepping parameters are given in groups. The elastic solution
# corresponds to group 0 and must always be defined. Although some of
# the parameters do not have any meaning for the elastic solution,
# they must be present anyway.
#
# Columns:
#   (1) Time step group number (=0 for elastic solution).
#   (2) The number of time steps in the group (=1 for elastic solution).
#   (3) Time step size (given in units of time_units).
#   (4) Amount of implicitness. Real dimensionless parameter that
#       ranges from 0.0 (fully explicit) to 1.0 (fully implicit). The
#       value is generally set to 0.5.
#   (5) Maximum number of equilibrium iterations before stiffness
#       matrix is reformed.
#   (6) Number of time steps between initial reformation of stiffness
#       matrix.
#       &lt;0 Indicates that reformation should occur only for the first
#          step in each time step group.
#       =0 Indicates that reformation should never occur.
#   (7) Large deformation solution flag (only Linear strain is
#       presently available).
#       0 = Linear strain 
#       1 = Large strain but use only linear contribution to the
#           stiffness matrix (sometimes results in better convergence)
#       2 = Large strain and use nonlinear contribution to the
#           stiffness matrix
#   (8) Convergence tolerance for displacements (dimensionless value)
#   (9) Convergence tolerance for forces (dimensionless value)
#   (10) Convergence tolerance for energy (dimensionless value)
#   (11) Maximum number of equilibrium iterations
#
  0   1  0.0  5.0e-01 1001   4  0  1.0e+00  1.0e+0  1.0e+00 1
  1 100  0.1  5.0e-01 1001  -1  0  1.0e+00  1.0e+0  1.0e+00 1
\end{verbatim}
    \ldots
    \caption{Format of \filename{xx.time} files.}
  \end{center}
\end{figure}
