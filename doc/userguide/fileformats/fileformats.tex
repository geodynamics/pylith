\chapter{File Formats}
\label{cha:formats}

\section{PyLith Mesh ASCII Files}
\label{sec:format:MeshIOAscii}

PyLith mesh ASCII files allow quick specification of the mesh information
for very small, simple meshes that are most easily written by hand.
We do not recommend using this format for anything other than these
very small, simple meshes.

\begin{figure}[htbp]
  \includegraphics{fileformats/figs/meshquad4}
  \caption{Diagram of mesh specified in the \object{MeshIOAscii} file.}
  \label{fig:meshioascii:diagram}
\end{figure}


\begin{MeshIOAscii}
// This mesh file defines a finite-element mesh composed of two
// square cells of edge length 2.
//
// Comments can appear almost anywhere in these files and are
// delimited with two slashes (//) just like in C++. All text and 
// whitespace after the delimiter on a given line is ignored.

mesh = { // begin specification of the mesh
  dimension = 2 // spatial dimension of the mesh

  // Begin vertex and cell labels with 0. This is the default so 
  // this next line is optional
  use-index-zero = true

  vertices = { // vertices or nodes of the finite-element cells
    dimension = 2 // spatial dimension of the vertex coordinates
    count = 6 // number of vertices in the mesh
    coordinates = { // list of vertex index and coordinates
      // the coordinates must coincide with the coordinate 
      // system specified in the Mesh object
      // exactly one vertex must appear on each line
      // (excluding whitespace)
      0    -2.0  -1.0
      1    -2.0  +1.0
      2     0.0  -1.0
      3     0.0  +1.0
      4    +2.0  -1.0
      5    +2.0  +1.0
    } // end of coordinates list
  } // end of vertices

  cells = { // finite-element cells
    count = 2 // number of cells in the mesh
    num-corners = 4 // number of vertices defining the cell
    simplices = { // list of vertices in each cell
      // see Section 4.2 for diagrams giving the order for each 
      // type of cell supported in PyLith
      // index of cell precedes the list of vertices for the cell 
      // exactly one cell must appear on each line
      // (excluding whitespace)
      0    0  2  3  1
      1    4  5  3  2
    } // end of simplices list

    material-ids = { // associated each cell with a material model
      // the material id is specified using the index of the cell 
      // and then the corresponding material id
      0   0 // cell 0 has a material id of 0
      1   2 // cell 1 has a material id of 2
    } // end of material-ids list
  } // end of cells

  // This next section lists groups of vertices that can be used
  // in applying boundary conditions to portions of the domain
  group = { // start of a group
    // the name can have whitespace, so no comments are allowed
    // after the name
    name = face +y

    // Either groups of vertices or groups of cells can be     
    // specified, but currently PyLith only makes use of groups 
    // of vertices
    type = vertices // 'vertices' or 'cells'
    count = 2 // number of vertices in the group
    indices = { // list of vertex indices in the group
      // multiple vertices may appear on a line
      0  4 // this group contains vertices 0 and 4
    } // end of list of vertices
  } // end of group

  // additional groups can be listed here
\end{MeshIOAscii}


\section{\object{SimpleDB} Spatial Database Files}
\label{sec:format:SimpleIOAscii}

\object{SimpleDB} spatial database files contain a header describing
the set of points and then the data with each line listing the
coordinates of a point followed by the values of the fields for that
point.

\begin{SimpleIOAscii}
// This spatial database specifies the distribution of slip on the
// fault surface. In this case we prescribe a piecewise linear, 
// depth dependent distribution of slip. The slip is 2.0 m 
// right-lateral with 0.25 m of reverse slip at the surface with
// a linear taper from 2.0 m to 0.0 m from -2 km to -4 km.
//
// Comments can appear almost anywhere in these files and are
// delimited with two slashes (//) just like in C++. All text and 
// whitespace after the delimiter on a given line is ignored.
//
// The next line is the magic header for spatial database files 
// in ASCII format.
#SPATIAL.ascii 1
SimpleDB { // start specifying the database parameters
  num-values = 3 // number of values in the database

  // Specify the names and the order of the values as they appear 
  // in the data. The names of the values must correspond to the 
  // names PyLith requests in querying the database.
  value-names =  left-lateral-slip  reverse-slip  fault-opening

  // Specify the units of the values in Python syntax (e.g., kg/m**3).
  value-units =  m  m  m
  num-locs = 3 // Number of locations where values are given
  data-dim = 1 // Locations of data points form a line
  space-dim = 3 // Spatial dimension in which data resides

  // Specify the coordinate system associated with the 
  // coordinates of the locations where data is given
  cs-data = cartesian { // Use a Cartesian coordinate system
    to-meters = 1.0e+3 // Coordinates are in km

    // Specify the spatial dimension of the coordinate system
    // This value must match the one associated with the database
    space-dim = 3

  } // cs-data // end of coordinate system specification

} // end of SimpleDB parameters
// The locations and values are listed after the parameters.
// Columns are coordinates of the points (1 column for each 
// spatial dimension) followed by the data values in the order 
// specified by the value-names field.
0.0  0.0  0.0    -2.00  0.25  0.00
0.0  0.0 -2.0    -2.00  0.00  0.00
0.0  0.0 -4.0     0.00  0.00  0.00
\end{SimpleIOAscii}

\subsection{Spatial Database Coordinate Systems}

The spatial database files support four different types of coordinate
systems. Conversions among the three geographic coordinate systems
are supported in 3D. Conversions among the geographic and geographic
projected coordinate systems are supported in 2D. In using the coordinate
systems, we assume that you have installed the \filename{proj} program
in addition to the Proj.4 libraries, so that you can obtain a list
of support projections, datums, and ellipsoids. Alternatively, refer
to the documentation for the Proj.4 Cartographic Projections library
\url{trac.osgeo.org/proj}.

\subsubsection{Cartesian}

This is a conventional Cartesian coordinate system. Conversions to
other Cartesian coordinate systems are possible.
\begin{SimpleIOAscii}
cs-data = cartesian {
  to-meters = 1.0e+3 // Locations in km
  space-dim = 2 // 1, 2, or 3 dimensions
}
\end{SimpleIOAscii}

\subsubsection{Geographic}

This coordinate system is for geographic coordinates, such as longitude
and latitude. Specification of the location in three-dimensions is
supported. The vertical datum can be either the reference ellipsoid
or mean sea level. The vertical coordinate is positive upwards.
\begin{SimpleIOAscii}
cs-data = geographic {
  // Conversion factor to get to meters (only applies to vertical 
  // coordinate unless you are using a geocentric coordinate system).
  to-meters = 1.0e+3
  space-dim = 2 // 2 or 3 dimensions

  // Run ``proj -le'' to see a list of available reference ellipsoids.
  // Comments are not allowed at the end of the next line.
  ellipsoid = WGS84

  // Run ``proj -ld'' to see a list of available datums.
  // Comments are not allowed at the end of the next line.
  datum-horiz = WGS84

  // ``ellipsoid'' or ``mean sea level''
  // Comments are not allowed at the end of the next line.
  datum-vert = ellipsoid

  // Use a geocentric coordinate system?
  is-geocentric = false // true or false
}
\end{SimpleIOAscii}

\subsubsection{Geographic Projection}

This coordinate system applies to geographic projections. As in the
geographic coordinate system, the vertical coordinate (if used) can
be specified with respect to either the reference ellipsoid or mean
sea level. The coordinate system can use a local origin and be rotated
about the vertical direction.
\begin{SimpleIOAscii}
cs-data = geo-projected {
  to-meters = 1.0e+3 // Conversion factor to get to meters.
  space-dim = 2 // 2 or 3 dimensions

  // Run ``proj -le'' to see a list of available reference ellipsoids.
  // Comments are not allowed at the end of the next line.
  ellipsoid = WGS84

  // Run ``proj -ld'' to see a list of available datums.
  // Comments are not allowed at the end of the next line.
  datum-horiz = WGS84

  // ``ellipsoid'' or ``mean sea level''
  // Comments are not allowed at the end of the next line.
  datum-vert = ellipsoid

  // Longitude of local origin in WGS84.
  origin-lon = -120.0

  // Latitude of local origin in WGS84.
  origin-lat = 37.0

  // Rotation angle in degrees (CCW) or local x-axis from east.
  rotation-angle = 23.0

  // Run ``proj -lp'' to see a list of available geographic 
  // projections.
    projector = projection {
    // Name of the projection. run ``proj -lp'' to see a list of 
    // supported projections. Use the Universal Transverse Mercator
    // projection.
    projection = utm
    units = m // Units in the projection.

    // Provide a list of projection options; these are the command 
    // line arguments you would use with the proj program. Refer to
    // the Proj.4 library documentation for complete details.
    // Comments are not allowed at the end of the next line.
    proj-options = +zone=10
}
\end{SimpleIOAscii}

\subsubsection{Geographic Local Cartesian}

This coordinate system is a geographically referenced, local 3D Cartesian
coordinate system. This allows use of a conventional Cartesian coordinate
system with accurate georeferencing. For example, one can construct
a finite-element model in this coordinate system and use spatial databases
in geographic coordinates. This coordinate system provides an alternative
to using a geographic projection as the Cartesian grip. The advantage
of this coordinate system is that it retains the curvature of the
Earth, while a geographic projection does not.
\begin{SimpleIOAscii}
cs-data = geo-local-cartesian {
  // Conversion factor to get to meters (only applies to vertical
  // coordinate unless you are using a geocentric coordinate system).
  to-meters = 1.0 // use meters
  space-dim = 2 // 2 or 3 dimensions

  // Run ``proj -le'' to see a list of available reference ellipsoids.
  // Comments are not allowed at the end of the next line.
  ellipsoid = WGS84

  // Run ``proj -ld'' to see a list of available datums.
  // Comments are not allowed at the end of the next line.
  datum-horiz = WGS84

  // ``ellipsoid'' or ``mean sea level''
  // Comments are not allowed at the end of the next line.
  datum-vert = ellipsoid

  // Origin of the local Cartesian coordinate system. To avoid
  // round-off errors it is best to pick a location near the center of
  // the region of interest. An elevation on the surface of the Earth
  // in the middle of the region also works well (and makes the
  // vertical coordinate easy to interpret).
  origin-lon = -116.7094 // Longitude of the origin in decimal degrees
                         // (west is negative).

  origin-lat = 36.3874 // Latitude of the origin in decimal degrees 
                       // (north is positive).

  // Elevation with respect to the vertical datum. Units are the
  // same as the Cartesian coordinate system (in this case meters).
  origin-elev = 3.5
}
\end{SimpleIOAscii}

\section{\object{SimpleGridDB} Spatial Database Files}
\label{sec:format:SimpleGridDB}

\object{SimpleGridDB} spatial database files contain a header
describing the grid of points and then the data with each line listing
the coordinates of a point followed by the values of the fields for
that point. The coordinates for each dimension of the grid do not need
to be uniformly spaced. The coordinate systems are specified the same
way as they are in \object{SimpleDB} spatial database files as
described in Section \vref{sec:format:SimpleIOAscii}.
\begin{SimpleGridDB}
// This spatial database specifies the elastic properties on a
// 2-D grid in 3-D space.
//
// Comments can appear almost anywhere in these files and are
// delimited with two slashes (//) just like in C++. All text and 
// whitespace after the delimiter on a given line is ignored.

// The next line is the magic header for spatial database files 
// in ASCII format.
#SPATIAL_GRID.ascii 1
SimpleGridDB { // start specifying the database parameters
  num-values = 3 // number of values in the database

  // Specify the names and the order of the values as they appear 
  // in the data. The names of the values must correspond to the
  // names PyLith requests in querying the database.
  value-names =  Vp  Vs  Density

  // Specify the units of the values in Python syntax.
  value-units =  km/s  km/s  kg/m{*}{*}3
  num-x = 3 // Number of locations along x coordinate direction
  num-y = 1 // Number of locations along y coordinate direction
  num-z = 2 // Number of locations along z coordinate direction
  space-dim = 3 // Spatial dimension in which data resides

  // Specify the coordinate system associated with the 
  // coordinates of the locations where data is given
  cs-data = cartesian \{ // Use a Cartesian coordinate system
    to-meters = 1.0e+3 // Coordinates are in km

    // Specify the spatial dimension of the coordinate system
    // This value must match the one associated with the database
    space-dim = 3
  } // cs-data // end of coordinate system specification
} // end of SimpleGridDB specification
// x coordinates
-3.0  1.0  2.0
// y coordinates
8.0
// z coordinates
2.0  4.0

// The locations and values are listed after the parameters.
// Columns are coordinates of the points (1 column for each 
// spatial dimension) followed by the data values in the order 
// specified by the value-names field. The points can be in any order.
-3.0  8.0  2.0    6.0  4.0  2500.0
 1.0  8.0  2.0    6.2  4.1  2600.0
 2.0  8.0  2.0    5.8  3.9  2400.0
-3.0  8.0  4.0    6.1  4.1  2500.0
 1.0  8.0  4.0    5.9  3.8  2450.0
 2.0  8.0  4.0    5.7  3.7  2400.0
\end{SimpleGridDB}

\section{\object{TimeHistory} Database Files}
\label{sec:format:TimeHistoryIO}

\object{TimeHistory} database files contain a header describing the
number of points in the time history and the units for the time stamps
followed by a list with pairs of time stamps and amplitude values. The
amplitude at an arbitrary point in time is computed via interpolation
of the values in the database. This means that the time history
database must span the range of time values of interest. The points in
the time history must also be ordered in time.
\begin{TimeHistoryIO}
// This time history database specifies temporal variation in
// amplitude. In this case we prescribe a triangular slip time
// history. 
//
// Comments can appear almost anywhere in these files and are
// delimited with two slashes (//) just like in C++. All text and 
// whitespace after the delimiter on a given line is ignored.
//
// The next line is the magic header for spatial database files 
// in ASCII format.
#TIME HISTORY ascii
TimeHistory { // start specifying the database parameters
  num-points = 5 // number of points in time history

  // Specify the units used in the time stamps.
  time-units = year
} // end of TimeHistory header
// The time history values are listed after the parameters.
// Columns time and amplitude where the amplitude values are unitless.
 0.0     0.00
 2.0     1.00
 6.0     4.00
10.0     2.00
11.0     0.00
\end{TimeHistoryIO}


\section{User-Specified Time-Step File}
\label{sec:format:TimeStepUser}

This file lists the time-step sizes for nonuniform, user-specified
time steps associated with the \object{TimeStepUser} object. The
file's format is an ASCII file that includes the units for the
time-step sizes and then a list of the time steps.
\begin{TimeStepUser}
// This time step file specifies five time steps with the units in years.
// Comments can appear almost anywhere in these files and are
// delimited with two slashes (//) just like in C++. All text and 
// whitespace after the delimiter on a given line is ignored.
//
// Units for the time steps
units = year
1.0 // Comment
2.0
3.0
2.5
3.0
\end{TimeStepUser}

\section{\object{PointsList} File}
\label{sec:format:PointsList}

This file lists the coordinates of the locations where output is requested
for the \texttt{OutputSolnPoints} component. The coordinate system
is specified in the \texttt{OutputSolnPoints} component. 
\begin{PointsList}
# Comments are limited to complete lines. The default delimiter for comments
# is '#', which can be changed via parameters. Additionally, the delimiter 
# separating values can also be customized (default is whitespace).
#
# The first column is the station name. The coordinates of the points are given
# in the subsequent columns.
P0  1.0  -2.0   0.0
P1  2.0  -4.0  -0.1
P2  0.0  +2.0   0.0
P3  2.5  -0.2  -0.2 
P4  0.0   2.0  +0.2
\end{PointsList}

% End of file
