\chapter{Preface}

\section{About This Document}

This document is organized into two parts. Part I begins with an
introduction to PyLith and discusses how to run the software. Part II
provides appendices and references.

\section{Who Will Use This Documentation}

This documentation is aimed at scientists who prefer to use
prepackaged and specialized analysis tools. Users are likely to be
experienced computational earth scientists and have familiarity with
basic scripting, software installation, and programming; but are not
likely to be professional programmers. Of those, there are likely to
be two classes of users: those who run models and those who modify the
source code.

\section{Citation}

The Computational Infrastructure for Geodynamics (CIG) is making this
source code available to you at no cost in hopes that the software
will enhance your research in geophysics. A number of individuals have
contributed a significant portion of their careers toward the
development of this software. It is essential that you recognize these
individuals in the normal scientific practice by citing the
appropriate peer reviewed papers and making appropriate
acknowledgements in talks and publications. NEED CITATION INFORMATION

\section{Support}

Current PyLith development is supported by the Southern California
Earthquake Center, the National Science Foundation, the CIG, and
internal U.S. Geological Survey funding. Pyre development is funded by
the \link{Department of
  Energy's}{http://www.doe.gov/engine/content.do} Advanced Simulation
and Computing program and the \link{National Science
  Foundation's}{http://www.nsf.gov} Information Technology Research
(ITR) program.

\section{Request for Comments}

Your suggestions and corrections can only improve this documentation.
Please report any errors, inaccuracies, or typos to sue (at)
geodynamics.org.
