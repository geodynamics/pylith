% ----------------------------------------------------------------------
\section{Elasticity with Infinitesimal Strain and Prescribed Slip on Faults}

We begin with the elasticity equation including the inertial term,
\begin{gather}
  \label{eqn:elasticity:strong:form}
  \rho \frac{\partial^2\vec{u}}{\partial t^2} - \vec{f}(\vec{x},t) - \tensor{\nabla} \cdot 
\tensor{\sigma}
(\vec{u}) = \vec{0} \text{ in }\Omega, \\
%
  \label{eqn:bc:Neumann}
  \tensor{\sigma} \cdot \vec{n} = \vec{\tau}(\vec{x},t) \text{ on }\Gamma_\tau, \\
%
  \label{eqn:bc:Dirichlet}
  \vec{u} = \vec{u}_0(\vec{x},t) \text{ on }\Gamma_u,
\end{gather}
where $\vec{u}$ is the displacement vector, $\rho$ is the mass
density, $\vec{f}$ is the body force vector, $\tensor{\sigma}$ is the
Cauchy stress tensor, $\vec{x}$ is the spatial coordinate, and $t$ is
time. We specify tractions $\vec{\tau}$ on boundary $\Gamma_\tau$, and
displacements $\vec{u}_0$ on boundary $\Gamma_u$. Because both $\vec{\tau}$
and $\vec{u}$ are vector quantities, there can be some spatial overlap
of boundaries $\Gamma_\tau$ and $\Gamma_u$; however, a degree of freedom at
any location cannot be associated with both prescribed displacements
(Dirichlet) and traction (Neumann) boundary conditions simultaneously.

For each fault, which is an internal interface, we add a boundary
condition to the elasticity equation prescribing the jump in the
displacement field across the fault,
\begin{gather}
  \label{eqn:bc:prescribed_slip}
  \vec{u}^+(\vec{x},t) - \vec{u}^-(\vec{x},t) - \vec{d}(\vec{x},t) = \vec{0} \text{ on }\Gamma_f,
\end{gather}
where $\vec{u}^+$ is the displacement vector on the ``positive'' side
of the fault, $\vec{u}^-$ is the displacement vector on the
``negative'' side of the fault, $\vec{d}$ is the slip vector on the
fault, and $\vec{n}$ is the fault normal which points from the
negative side of the fault to the positive side of the fault. Note
that as an alternative to prescribing the jump in displacement across
the fault, we can prescribe the jump in velocity across the fault in
terms of slip rate. 

Using a domain decomposition approach for constraining the fault slip,
yields additional Neumann-like boundary conditions on the fault
surface,
\begin{gather}
  \tensor{\sigma} \cdot \vec{n} = -\vec{\lambda}(\vec{x},t) \text{ on }\Gamma_{f^+}, \\
  \tensor{\sigma} \cdot \vec{n} = +\vec{\lambda}(\vec{x},t) \text{ on }\Gamma_{f^-},
\end{gather}
where $\vec{\lambda}$ is the vector of Lagrange multipliers
corresponding to the tractions applied to the fault surface to
generate the prescribed slip. 


\begin{table}[htbp]
  \caption{Mathematical notation for elasticity equation with
    infinitesimal strain and prescribed slip on faults.}
  \label{tab:notation:elasticity:prescribed:slip}
  \begin{tabular}{lcp{3in}}
    \toprule
    {\bf Category} & {\bf Symbol} & {\bf Description} \\
    \midrule
    Unknowns & $\vec{u}$ & Displacement field \\
    & $\vec{v}$ & Velocity field \\
    & $\vec{\lambda}$ & Lagrange multiplier field \\
    Derived quantities & $\tensor{\sigma}$ & Cauchy stress tensor \\
                   & $\tensor{\epsilon}$ & Cauchy strain tensor \\
    Common constitutive parameters & $\rho$ & Density \\
  & $\mu$ & Shear modulus \\
  & $K$ & Bulk modulus \\
Source terms & $\vec{f}$ & Body force per unit volume, for example $\rho \vec{g}$ \\
    & $\vec{d}$ & Slip vector field on the fault corresponding to a
      jump in the displacement field across the fault \\
    \bottomrule
  \end{tabular}
\end{table}


We form a first order equation using displacement $\vec{u}$ and
velocity $\vec{v}$ as unknowns. We prescribe the fault rupture in
terms of a jump in velocity for symmetry with the Lagrange
multipliers. The complete set of equations is now:
\begin{align}
  % Displacement-velocity
  \frac{\partial \vec{u}}{\partial t} &= \vec{v} \text{ in } \Omega, \\
  % Elasticity
  \rho(\vec{x}) \frac{\partial\vec{v}}{\partial t} &= \vec{f}(\vec{x},t) + \tensor{\nabla} \cdot \tensor{\sigma}(\vec{u}) \text{ in } \Omega, \\
  % Neumann
  \tensor{\sigma} \cdot \vec{n} &= \vec{\tau}(\vec{x},t) \text{ on } \Gamma_\tau, \\
  % Dirichlet
  \vec{u} &= \vec{u}_0(\vec{x},t) \text{ on } \Gamma_u, \\
  % Fault
  \vec{v}^+(\vec{x},t) - \vec{v}^-(\vec{x},t) &= \vec{\dot{d}}(\vec{x},t) \text{ on } \Gamma_f, \\
  \tensor{\sigma} \cdot \vec{n} &= -\vec{\lambda}(\vec{x},t) \text{ on }\Gamma_{f^+}, \text{ and} \\
  \tensor{\sigma} \cdot \vec{n} &= +\vec{\lambda}(\vec{x},t) \text{ on }\Gamma_{f^-}.
\end{align}

We create the weak form by taking the dot product with the trial
function $\trialvec[u]$, $\trialvec[u]$, or $\trialvec[\lambda]$ and
integrating over the domain:
\begin{gather}
  % Displacement-velocity
  \int_\Omega \trialvec[u] \cdot \frac{\partial \vec{u}}{\partial t} \, d\Omega = 
  \int_\Omega \trialvec[u] \cdot \vec{v} \, d\Omega, \\
  % Elasticity
    \int_\Omega \trialvec[v] \cdot \rho(\vec{x}) \frac{\partial \vec{v}}{\partial t} \, d\Omega 
 = \int_\Omega \trialvec[u] \cdot \left( \vec{f}(t) + \tensor{\nabla} \cdot \tensor{\sigma} (\vec{u}) \right) \, d\Omega, \\
  % Prescribed slip
  \int_{\Gamma_{f}} \trialvec[\lambda] \cdot \left(
    \vec{v}^+(\vec{x},t) - \vec{v}^-(\vec{x},t) - \vec{\dot{d}}(\vec{x},t) \right) \, d\Gamma = 0.
\end{gather}
Using the divergence theorem and incorporating the Neumann boundary and fault interface
conditions, we can rewrite the second equation as
\begin{multline}
% 
  \int_\Omega \trialvec[v] \cdot \rho(\vec{x}) \frac{\partial \vec{v}}{\partial t} \, d\Omega
  = \int_\Omega \trialvec[v] \cdot \vec{f}(\vec{x},t) + \nabla \trialvec[v] : -\tensor{\sigma}(\vec{u}) \, d\Omega
  + \int_{\Gamma_\tau} \trialvec[v] \cdot \vec{\tau}(\vec{x},t) \, d\Gamma\\
  + \int_{\Gamma_{f}} \trialvec[v^+] \cdot \left(-\vec{\lambda}(\vec{x},t)\right)
  + \trialvec[v^-] \cdot \left(+\vec{\lambda}(\vec{x},t)\right)\, d\Gamma
\end{multline}
Rearranging the terms so that the Lagrange multiplier related terms appear on the LHS (implicit part), we have
\begin{gather}
  % Displacement-velocity
  \int_\Omega \trialvec[u] \cdot \frac{\partial \vec{u}}{\partial t} \, d\Omega = 
  \int_\Omega \trialvec[u] \cdot \vec{v} \, d\Omega, \\
  % Elasticity
  \begin{multlined}
  \int_\Omega \trialvec[v] \cdot \rho(\vec{x}) \frac{\partial \vec{v}}{\partial t} \, d\Omega
  + \int_{\Gamma_{f}} \trialvec[v^+] \cdot \left(+\vec{\lambda}(\vec{x},t)\right)
  + \trialvec[v^-] \cdot \left(-\vec{\lambda}(\vec{x},t)\right)\, d\Gamma \\
  = \int_\Omega \trialvec[v] \cdot \vec{f}(\vec{x},t) + \nabla \trialvec[v] : -\tensor{\sigma}(\vec{u}) \, d\Omega
  + \int_{\Gamma_\tau} \trialvec[v] \cdot \vec{\tau}(\vec{x},t) \, d\Gamma,
  \end{multlined}\\
  % Prescribed slip
  \int_{\Gamma_{f}} \trialvec[\lambda] \cdot \left(
    \vec{v}^+(\vec{x},t) - \vec{v}^-(\vec{x},t) - \vec{\dot{d}}(\vec{x},t) \right) \, d\Gamma = 0.
\end{gather}

\subsection{Residual Pointwise Kernels}

Identifying $F(t,s,\dot{s})$ and $G(t,s)$, we have
\begin{align}
  % Fu
  F^u(t,s,\dot{s}) &=  \int_\Omega \trialvec[u] \cdot \eqnannotate{\frac{\partial \vec{u}}{\partial t}}{\vec{f}^u_0} \, d\Omega, \\
  % Fv
  F^v(t,s,\dot{s}) &=  \int_\Omega \trialvec[v] \cdot \eqnannotate{\rho(\vec{x}) \frac{\partial \vec{v}}{\partial t}}{\vec{f}^v_0} \, d\Omega
            + \int_{\Gamma_{f}} \trialvec[v^+] \cdot \eqnannotate{\left(+\vec{\lambda}(\vec{x},t)\right)}{\vec{f}^v_0}
                     + \trialvec[v^-] \cdot \eqnannotate{\left(-\vec{\lambda}(\vec{x},t)\right)}{\vec{f}^v_0}\, d\Gamma \\
  % Fl
  F^\lambda(t,s,\dot{s}) &= \int_{\Gamma_{f}} \trialvec[\lambda] \cdot \eqnannotate{\left(
    \vec{v}^+(\vec{x},t) - \vec{v}^-(\vec{x},t) - \vec{\dot{d}}(\vec{x},t) \right)}{\vec{f}^\lambda_0} \, d\Gamma, \\
  % Gu
  G^u(t,s) &= \int_\Omega \trialvec[u] \cdot \eqnannotate{\vec{v}}{\vec{g}^u_0} \, d\Omega, \\
  % Gv
  G^v(t,s) &=  \int_\Omega \trialvec[v] \cdot \eqnannotate{\vec{f}(\vec{x},t)}{\vec{g}^v_0} + \nabla \trialvec[v] : \eqnannotate{-\tensor{\sigma}(\vec{u})}{\tensor{g^v_1}} \, d\Omega
  + \int_{\Gamma_\tau} \trialvec[v] \cdot \eqnannotate{\vec{\tau}(\vec{x},t)}{\vec{g}^v_0} \, d\Gamma, \\
  % Gl
  G^\lambda(t,s) &= 0
\end{align}
Note that we have multiple $\vec{f}_0$ and $\vec{g}_0^u$ functions, each
associated with a trial function and an integral over a different
domain or boundary. Each material, internal interface, and boundary
condition (except Dirichlet) contribute pointwise functions. The
integrals over the domain $\Omega$ are associated with materials, the
integral over the boundary $\Gamma_\tau$ is associated with Neumann
boundary conditions, and the integrals over the interface $\Gamma_{f}$
are associated with faults (cohesive cells).

\subsection{Jacobian Pointwise Kernels}

The LHS Jacobians are:
\begin{align}
  % J_F uu
  J_F^{uu} &= \frac{\partial F^u}{\partial u} + s_\mathit{tshift} \frac{\partial F^u}{\partial \dot{u}} =
             \int_\Omega \trialscalar[u]_i \eqnannotate{s_\mathit{tshift} \delta_{ij}}{J^{uu}_{f0}} \basisscalar[u]_j  \, d\Omega, \\
  % J_F vv
  J_F^{vv} &= \frac{\partial F^v}{\partial v} + s_\mathit{tshift} \frac{\partial F^v}{\partial \dot{v}} =
             \int_\Omega \trialscalar[v]_i \eqnannotate{\rho(\vec{x}) s_\mathit{tshift} \delta_{ij}}{J ^{vv}_{f0}} \basisscalar[v]_j \, d\Omega \\
  % J_F vl
  J_F^{v\lambda} &= \frac{\partial F^v}{\partial \lambda} + s_\mathit{tshift} \frac{\partial F^v}{\partial \dot{\lambda}} =
                    \int_{\Gamma_{f}} \trialscalar[v^+]_i \eqnannotate{\left(+\delta_{ij}\right)}{J^{v\lambda}_{f0}} \basisscalar[\lambda]_j
                   + \trialscalar[v^-]_i \eqnannotate{\left(-\delta_{ij}\right)}{J^{v\lambda}_{f0}} \basisscalar[\lambda]_j\, d\Gamma \\
  % J_F lv
  J_F^{\lambda v} &= \frac{\partial F^\lambda}{\partial v} + s_\mathit{tshift} \frac{\partial F^\lambda}{\partial \dot{v}} =
                    \int_{\Gamma_{f}} \trialscalar[\lambda]_i 
                    \eqnannotate{\left(+\delta_{ij}\right)}{J^{\lambda v}_{f0}} \basisscalar[v^+]_j
                    + \trialscalar[\lambda]_i \eqnannotate{\left(-\delta_{ij}\right)}{J^{\lambda v}_{f0}} \basisscalar[v^-]_j \, d\Gamma.
\end{align}
The RHS Jacobians are:
\begin{align}
  % J_G uv
  J_G^{uv} &= \frac{\partial G^u}{\partial v} =
            \int_\Omega \trialscalar[u]_i \eqnannotate{\delta_{ij}}{J^{uv}_{g0}} \basisscalar[v]_j \, d\Omega , \\
% J_G vu
  J_G^{vu} &= \frac{\partial G^v}{\partial u} = \int_\Omega \nabla \trialvec[v] : 
\frac{\partial}{\partial u}(-\tensor{\sigma}) \, d\Omega 
  = \int_\Omega \nabla \trialvec[v] : -\tensor{C} : \frac{1}{2}(\nabla + \nabla^T)\basisvec[u] 
\, d\Omega 
  = \int_\Omega \trialscalar[v]_{i,k} \, \eqnannotate{\left( -C_{ikjl} \right)}{J_{g3}^{vu}} \, \basisscalar[u]_{j,l}\, d\Omega
\end{align}


\subsection{Quasistatic}

If we neglect the inertial term
($\rho \frac{\partial v}{\partial t}$), then time dependence only
arises from history-dependent constitutive equations and boundary
conditions. We can drop velocity from the solution field and use the
displacement $\vec{u}$ and Lagrange multiplier $\vec{\lambda}$ fields
as the unknowns. In doing so, we prescribe fault rupture using a jump
in the displacement field in terms of fault slip. Our set of equations reduces to

\begin{align}
  % Solution
  \vec{s}^T &= \left( \begin{array}{cc} \vec{u} & \vec{\lambda} \end{array} \right)^T \\
  % Elasticity
  \vec{0} &= \vec{f}(\vec{x},t) + \tensor{\nabla} \cdot \tensor{\sigma}(\vec{u}) \text{ in } \Omega, \\
  % Neumann
  \tensor{\sigma} \cdot \vec{n} &= \vec{\tau}(\vec{x},t) \text{ on } \Gamma_\tau, \\
  % Dirichlet
  \vec{u} &= \vec{u}_0(\vec{x},t) \text{ on } \Gamma_u, \\
  % Fault
  \vec{u}^+(\vec{x},t) - \vec{u}^-(\vec{x},t) &= \vec{d}(\vec{x},t) \text{ on } \Gamma_f, \\
  \tensor{\sigma} \cdot \vec{n} &= -\vec{\lambda}(\vec{x},t) \text{ on }\Gamma_{f^+}, \text{ and} \\
  \tensor{\sigma} \cdot \vec{n} &= +\vec{\lambda}(\vec{x},t) \text{ on }\Gamma_{f^-}.
\end{align}

We solve the system of equations using implicit time stepping, making
use of residuals functions and Jacobians for both the LHS and RHS.

\subsubsection{Residual Pointwise Kernels}

The residual functions simplify to
\begin{align}
  % Fu
  F^u(t,s,\dot{s}) &= \int_{\Gamma_{f}} \trialvec[u^+] \cdot \eqnannotate{\left(+\vec{\lambda}(\vec{x},t)\right)}{\vec{f}^u_0}
                     + \trialvec[u^-] \cdot \eqnannotate{\left(-\vec{\lambda}(\vec{x},t)\right)}{\vec{f}^u_0}\, d\Gamma \\
  % Fl
  F^\lambda(t,s,\dot{s}) &= \int_{\Gamma_{f}} \trialvec[\lambda] \cdot \eqnannotate{\left(
    \vec{u}^+(\vec{x},t) - \vec{u}^-(\vec{x},t) - \vec{d}(\vec{x},t) \right)}{\vec{f}^\lambda_0} \, d\Gamma, \\
  % Gu
  G^u(t,s) &=  \int_\Omega \trialvec[u] \cdot \eqnannotate{\vec{f}(\vec{x},t)}{\vec{g}^u_0} + \nabla \trialvec[u] : \eqnannotate{-\tensor{\sigma}(\vec{u})}{\tensor{g^u_1}} \, d\Omega
  + \int_{\Gamma_\tau} \trialvec[u] \cdot \eqnannotate{\vec{\tau}(\vec{x},t)}{\vec{g}^u_0} \, d\Gamma, \\
  % Gl
  G^\lambda(t,s) &= 0
\end{align}

\subsubsection{Jacobian Pointwise Kernels}

The LHS Jacobians are:
\begin{align}
  % J_F uu
  J_F^{uu} &= \tensor{0} \\
  J_F^{u\lambda} &= \frac{\partial F^u}{\partial \lambda} + s_\mathit{tshift} \frac{\partial F^u}{\partial \dot{\lambda}} =
                    \int_{\Gamma_{f}} \trialscalar[u^+]_i \eqnannotate{\left(+\delta_{ij}\right)}{J^{u\lambda}_{f0}} \basisscalar[\lambda]_j
                     + \trialscalar[u^-]_i \eqnannotate{\left(-\delta_{ij}\right)}{J^{u\lambda}_{f0}} \basisscalar[\lambda]_j\, d\Gamma, \\
  J_F^{\lambda u} &= \frac{\partial F^\lambda}{\partial u} + s_\mathit{tshift} \frac{\partial F^\lambda}{\partial \dot{u}} =
                    \int_{\Gamma_{f}} \trialscalar[\lambda]_i 
                    \eqnannotate{\left(+\delta_{ij}\right)}{J^{\lambda u}_{f0}} \basisscalar[u^+]_j
                    + \trialscalar[\lambda]_i \eqnannotate{\left(-\delta_{ij}\right)}{J^{\lambda u}_{f0}} \basisscalar[u^-]_j \, d\Gamma, \\
  J_F^{\lambda \lambda} &= \tensor{0}
\end{align}
This resulting LHS Jacobian has the structure
\begin{equation}
  J_F = \left( \begin{array} {cc} 0 & J_F^{u\lambda} \\ J_F^{\lambda u} & 0 \end{array} \right).
\end{equation}

The RHS Jacobians are:
\begin{align}
% J_G uu
  J_G^{uu} &= \frac{\partial G^u}{\partial u} = \int_\Omega \nabla \trialvec[u] : 
\frac{\partial}{\partial u}(-\tensor{\sigma}) \, d\Omega 
  = \int_\Omega \nabla \trialvec[u] : -\tensor{C} : \frac{1}{2}(\nabla + \nabla^T)\basisvec[u] 
\, d\Omega 
  = \int_\Omega \trialscalar[u]_{i,k} \, \eqnannotate{\left( -C_{ikjl} \right)}{J_{g3}^{uu}} \, \basisscalar[u]_{j,l}\, d\Omega
\end{align}
The resulting RHS Jacobian has the structure
\begin{equation}
  J_G = \left( \begin{array} {cc} J_G^{uu} & 0 \\ 0 & 0 \end{array} \right).
\end{equation}
The complete Jacobian has the form
\begin{equation}
  J = \left(
    \begin{array}{cc}
      J_G^{uu} & -J_f^{u \lambda} \\
      -J_f^{\lambda u} & 0
    \end{array} \right) = 
  \left(
    \begin{array}{cc}
      J_G^{uu} & -C^T \\
      -C & 0
    \end{array} \right),
\end{equation}
where $C$ contains entries of $\pm 1$ for degrees of
freedom on the two sides of the fault. The Schur complement of $J$
with respect to $J_G^{uu}$ is $-C\left(J_G^{uu}\right)^{-1}C^T$.

\subsection{Dynamic Without Fault}

If we do not have a fault, then the system of equations reduces to
\begin{align}
  % Displacement-velocity
  \frac{\partial \vec{u}}{\partial t} &= \vec{v} \text{ in } \Omega, \\
  % Elasticity
  \rho(\vec{x}) \frac{\partial\vec{v}}{\partial t} &= \vec{f}(\vec{x},t) + \tensor{\nabla} \cdot \tensor{\sigma}(\vec{u}) \text{ in } \Omega, \\
  % Neumann
  \tensor{\sigma} \cdot \vec{n} &= \vec{\tau}(\vec{x},t) \text{ on } \Gamma_\tau, \\
  % Dirichlet
  \vec{u} &= \vec{u}_0(\vec{x},t) \text{ on } \Gamma_u.
\end{align}

We can solve these equations using explicit time stepping. We solve an
augmented system in which we multiply the RHS residual function by the
inversion of the lumped LHS Jacobian,
\begin{gather}
  F^*(t,s,\dot{s}) = G^*(t,s) \text{, where} \\
  F^*(t,s,\dot{s}) = \dot{s} \text{ and} \\
  G^*(t,s) = J_F^{-1} G(t,s).
\end{gather}


\subsubsection{Residual Pointwise Kernels}

The PETSc TS assumes the LHS is $\dot{s}$, so we only need the RHS residual functions:
\begin{align}
  % Gu
  G^u(t,s) &= \int_\Omega \trialvec[u] \cdot \eqnannotate{\vec{v}}{\vec{g}^u_0} \, d\Omega, \\
  % Gv
  G^v(t,s) &=  \int_\Omega \trialvec[v] \cdot \eqnannotate{\vec{f}(\vec{x},t)}{\vec{g}^v_0} + \nabla \trialvec[v] : \eqnannotate{-\tensor{\sigma}(\vec{u})}{\tensor{g^v_1}} \, d\Omega
  + \int_{\Gamma_\tau} \trialvec[v] \cdot \eqnannotate{\vec{\tau}(\vec{x},t)}{\vec{g}^v_0} \, d\Gamma,
\end{align}

\subsubsection{Jacobian Pointwise Kernels}

These are the kernels for computing the lumped LHS Jacobian. We
premultiply the RHS residual function by the inverse of the lumped LHS
Jacobian while $s_\mathit{tshift}$ remains on the LHS with
$\dot{s}$. As a result, we use the usual LHS Jacobian pointwise
kernel, but set $s_\mathit{tshift}=1$.  The LHS Jacobians are:
\begin{align}
  % J_F uu
  J_F^{uu} &= \frac{\partial F^u}{\partial u} + s_\mathit{tshift} \frac{\partial F^u}{\partial \dot{u}} =
             \int_\Omega \trialscalar[u]_i \eqnannotate{s_\mathit{tshift} \delta_{ij}}{J^{uu}_{f0}} \basisscalar[u]_j  \, d\Omega, \\
  % J_F vv
  J_F^{vv} &= \frac{\partial F^v}{\partial v} + s_\mathit{tshift} \frac{\partial F^v}{\partial \dot{v}} =
             \int_\Omega \trialscalar[v]_i \eqnannotate{\rho(\vec{x}) s_\mathit{tshift} \delta_{ij}}{J ^{vv}_{f0}} \basisscalar[v]_j \, d\Omega
\end{align}


\subsection{Dynamic}

The equation prescribing fault slip is independent of the Lagrange
multiplier, so we do not have a system of equations that we can put in
the form $\dot{s} = G^*(t,s)$. Instead, we have a
differential-algebraic set of equations (DAEs), which we solve using an
implicit-explicit (IMEX) time integration scheme. The strong form for
our equations is:
\begin{gather}
% Displacement-velocituy
  \frac{\partial \vec{u}}{\partial t} = \vec{v}, \\
  % Elasticity
  \rho(\vec{x}) \frac{\partial \vec{v}}{\partial t} =
  \vec{f}(\vec{x},t) + \nabla \cdot \tensor{\sigma}(\vec{u}), \\
  % Presribed slip
  \vec{u}^+ - \vec{u}^- - \vec{d}(\vec{x},t) = \vec{0,} \\
  % Dirichlet BC
  \vec{u} = \vec{u}_0 \text{ on } \Gamma_u, \\
  % Neumann BC
  \tensor{\sigma} \cdot \vec{n} = \vec{\tau} \text{ on } \Gamma_\tau.
\end{gather}
The differentiation index is 2 because we must take the second time
derivative of the prescribed slip equation to match the order of the
time derivative in the elasticity equation. This leads to:
\begin{gather}
% Displacement-velocituy
  \frac{\partial \vec{u}}{\partial t} = \vec{v}, \\
  % Elasticity
  \label{eqn:dynamic:strong:form:elasticity}
  \rho(\vec{x}) \frac{\partial \vec{v}}{\partial t} =
  \vec{f}(\vec{x},t) + \nabla \cdot \tensor{\sigma}(\vec{u}), \\
  % Presribed slip
  \label{eqn:dynamic:strong:form:prescribed:slip}
  \frac{\partial \vec{v}^+}{\partial t} - \frac{\partial \vec{v}^-}{\partial t} -
  \frac{\partial^2 \vec{d}(\vec{x},t)}{\partial t^2} = \vec{0,} \\
  % Dirichlet BC
  \vec{u} = \vec{u}_0 \text{ on } \Gamma_u, \\
  % Neumann BC
  \tensor{\sigma} \cdot \vec{n} = \vec{\tau}(\vec{x},t) \text{ on } \Gamma_\tau.
\end{gather}
We generate the weak form in the usual way,
\begin{gather}
  % Displacement-velocity
  \int_{\Omega} \trialvec[u] \cdot \frac{\partial \vec{u}}{\partial t} \, d\Omega = 
  \int_{\Omega} \trialvec[u] \cdot \vec{v} \, d\Omega, \\
  % Elasticity
  \begin{multlined}
  \int_{\Omega} \trialvec[v] \cdot \rho(\vec{x}) \frac{\partial \vec{v}}{\partial t} \, d\Omega 
  = \int_\Omega \trialvec[v] \cdot \vec{f}(\vec{x},t) + \nabla \trialvec[v] : -\tensor{\sigma}(\vec{u}) \, d\Omega
  + \int_{\Gamma_\tau} \trialvec[v] \cdot \vec{\tau}(\vec{x},t) \, d\Gamma \\
  + \int_{\Gamma_{f}} \trialvec[v^+] \cdot \left(-\vec{\lambda}(\vec{x},t)\right)
  + \trialvec[v^-] \cdot \left(+\vec{\lambda}(\vec{x},t)\right)\, d\Gamma,
  \end{multlined}\\
  % Prescribed slip
  \int_{\Gamma_f} \trialvec[\lambda] \cdot \left(
    \frac{\partial \vec{v}^+}{\partial t} - \frac{\partial \vec{v}^-}{\partial t} -
    \frac{\partial^2 \vec{d}(\vec{x},t)}{\partial t^2} \right) \, d\Gamma = 0.
\end{gather}

For compatibility with PETSc TS IMEX implementations, we need
$\vec{s}$ on the LHS for the explicit part (displacement-velocity and
elasticity equations) and we need $\vec{\lambda}$ in the equation for
the implicit part (prescribed slip equation). We first focus on the
explicit part and create a lumped LHS Jacobian matrix, $M$, so that we
have
\begin{gather}
  % Displacement-velocity
  \frac{\partial \vec{u}}{\partial t} = M_u^{-1} \int_{\Omega} \trialvec[u] \cdot \vec{v} \, d\Omega, \\
  % Elasticity
  \begin{multlined}
  \frac{\partial \vec{v}}{\partial t}
  = M_v^{-1} \int_\Omega \trialvec[v] \cdot \vec{f}(\vec{x},t) + \nabla \trialvec[v] : -\tensor{\sigma}(\vec{u}) \, d\Omega
  + M_v^{-1} \int_{\Gamma_\tau} \trialvec[v] \cdot \vec{\tau}(\vec{x},t) \, d\Gamma \\
  + M_{v^+}^{-1} \int_{\Gamma_{f}} \trialvec[v^+] \cdot \left(-\vec{\lambda}(\vec{x},t)\right) \, d\Gamma
  + M_{v^-}^{-1} \int_{\Gamma_{f}}\trialvec[v^-] \cdot \left(+\vec{\lambda}(\vec{x},t)\right) \, d\Gamma,
\end{multlined}\\
% Mu
M_u = \mathit{Lump}\left( \int_\Omega \trialscalar[u]_i \delta_{ij} \basisscalar[u]_j \, d\Omega \right), \\
% Mv
M_v = \mathit{Lump}\left( \int_\Omega \trialscalar[v]_i \rho(\vec{x}) \delta_{ij} \basisscalar[v]_j \, d\Omega \right).
\end{gather}
Now, focusing on the implicit part we want to introduce
$\vec{\lambda}$ into the prescribed slip equation. We solve the
elasticity equation for $\frac{\partial \vec{v}}{\partial t}$, create
the weak form, and substitute into the prescribed slip
equation. Solving the elasticity equation for
$\frac{\partial \vec{v}}{\partial t}$, we have
\begin{equation}
  \frac{\partial \vec{v}}{\partial t} = \frac{1}{\rho(x)} \vec{f}(\vec{x},t) + \frac{1}{\rho(x)} \left(\nabla \cdot \tensor{\sigma}(\vec{u}) \right),
\end{equation}
and the corresponding weak form is
\begin{equation}
  \int_{\Omega} \trialvec[v] \cdot \frac{\partial \vec{v}}{\partial t} \, d\Omega
  = \int_\Omega \trialvec[v] \cdot \frac{1}{\rho(x)} \vec{f}(\vec{x},t) + \trialvec[v] \cdot \frac{1}{\rho(x)} \left(\nabla \cdot \tensor{\sigma}(\vec{u}) \right) \, d\Omega,
\end{equation}
We apply the divergence theorem,
\begin{equation}
  \int_{\Omega} \nabla \cdot \vec{F} \, d\Omega = \int_\Gamma \vec{F} \cdot \vec{n} \, d\Gamma,
\end{equation}
with $\vec{F} = \trialvec[v] \cdot \frac{1}{\rho(x)} \left(\nabla \cdot \tensor{\sigma}(\vec{u})\right)$ to get
\begin{equation}
  \int_\Omega \trialvec[v] \cdot \frac{1}{\rho(x)} \left(\nabla \cdot \tensor{\sigma}(\vec{u}) \right) \, d\Omega
  = \int_\Gamma \trialvec[v] \cdot \left( \frac{1}{\rho(x)} \tensor{\sigma}(\vec{u}) \cdot \vec{n} \right) \, d\Gamma
  + \int_\Omega \nabla\trialvec[v] : \left(-\frac{1}{\rho(\vec{x})} \tensor{\sigma}(\vec{u}) \right)
  + \trialvec[v] \cdot \left(-\frac{\nabla \rho(\vec{x})}{\rho^2} \cdot \tensor{\sigma}(\vec{u}) \right) \, d\Omega.
\end{equation}
Restricting the trial function to $v^+$ and $v^-$ while recognizing that there is no overlap between the external Neumann boundary conditions $\Gamma_\tau$ and the fault interfaces $\Gamma_f$, yields
\begin{gather}
  \int_\Omega \trialvec[v^+] \cdot \frac{1}{\rho(x)} \left(\nabla \cdot \tensor{\sigma}(\vec{u}) \right) \, d\Omega
  = \int_{\Gamma_f} \trialvec[v^+] \cdot \left(-\frac{1}{\rho(x)} \vec{\lambda} \right) \, d\Gamma
  + \int_\Omega \nabla\trialvec[v^+] : \left(-\frac{1}{\rho(\vec{x})} \tensor{\sigma}(\vec{u}) \right)
  + \trialvec[v^+] \cdot \, \left(-\frac{\nabla \rho(\vec{x})}{\rho^2} \cdot \tensor{\sigma}(\vec{u}) \right) \, d\Omega, \\
  \int_\Omega \trialvec[v^-] \cdot \frac{1}{\rho(x)} \left(\nabla \cdot \tensor{\sigma}(\vec{u}) \right) \, d\Omega
  = \int_{\Gamma_f} \trialvec[v^-] \cdot \left(+\frac{1}{\rho(x)} \vec{\lambda} \right) \, d\Gamma
  + \int_\Omega \nabla\trialvec[v^-] : \left(-\frac{1}{\rho(\vec{x})} \tensor{\sigma}(\vec{u}) \right)
  + \trialvec[v^-] \cdot \, \left(-\frac{\nabla \rho(\vec{x})}{\rho^2} \cdot \tensor{\sigma}(\vec{u}) \right) \, d\Omega. \end{gather}
Picking $\trialvec[v]=\trialvec[\lambda]$ and substituing into the prescribed slip equation gives
\begin{multline}
  \int_{\Gamma_f} \trialvec[\lambda] \cdot \left(
    \frac{\partial \vec{v}^+}{\partial t} - \frac{\partial \vec{v}^-}{\partial t} -
    \frac{\partial^2 \vec{d}(\vec{x},t)}{\partial t^2} \right) \, d\Gamma = \\
  \int_\Omega \trialvec[v^+] \cdot \left( \frac{1}{\rho(\vec{x})} \vec{f}(\vec{x}, t)
    -\frac{\nabla \rho(\vec{x})}{\rho^2(\vec{x})} \cdot \tensor{\sigma}(\vec{u}) \right) 
  + \nabla \trialvec[v^+] : \left(-\frac{1}{\rho(\vec{x})} \tensor{\sigma}(\vec{u}) \right) \, d\Omega
  + \int_{\Gamma_f} \trialvec[v^+] \cdot \left(-\frac{1}{\rho(\vec{x})} \vec{\lambda} \right) \, d\Gamma \\
  - \int_\Omega \trialvec[v^-] \cdot \left( \frac{1}{\rho(\vec{x})} \vec{f}(\vec{x}, t)
  -\frac{\nabla \rho(\vec{x})}{\rho^2(\vec{x})} \cdot \tensor{\sigma}(\vec{u}) \right)
  + \nabla \trialvec[v^-] : \left(-\frac{1}{\rho(\vec{x})} \tensor{\sigma}(\vec{u}) \right) \, d\Omega
  + \int_{\Gamma_f} \trialvec[v^+] \cdot \left(-\frac{1}{\rho(\vec{x})} \vec{\lambda} \right) \, d\Gamma \\
  - \int_{\Gamma_f} \trialvec[\lambda] \cdot \frac{\partial^2 \vec{d}(\vec{x}, t)}{\partial t^2} \, d\Gamma.
\end{multline}
Isolating the terms with $\vec{\lambda}$ on the LHS yields
\begin{multline}
  \int_\Omega \trialvec[v^+] \cdot \frac{1}{\rho(\vec{x})} \vec{\lambda}
  + \trialvec[v^-] \cdot \frac{1}{\rho(\vec{x})} \vec{\lambda} \, d\Omega \\
  - \int_\Omega \trialvec[v^+] \cdot \left( \frac{1}{\rho(\vec{x})} \vec{f}(\vec{x}, t)
  -\frac{\nabla \rho(\vec{x})}{\rho^2(\vec{x})} \cdot \tensor{\sigma}(\vec{u}) \right)
  + \nabla \trialvec[v^+] : \left(-\frac{1}{\rho(\vec{x})} \tensor{\sigma}(\vec{u}) \right) \, d\Omega \\
  + \int_\Omega \trialvec[v^-] \cdot \left( \frac{1}{\rho(\vec{x})} \vec{f}(\vec{x}, t)
  -\frac{\nabla \rho(\vec{x})}{\rho^2(\vec{x})} \cdot \tensor{\sigma}(\vec{u}) \right)
  + \nabla \trialvec[v^-] : \left(-\frac{1}{\rho(\vec{x})} \tensor{\sigma}(\vec{u}) \right) \, d\Omega \\
  + \int_{\Gamma_f} \trialvec[\lambda] \cdot \frac{\partial^2 \vec{d}(\vec{x}, t)}{\partial t^2} \, d\Gamma
  = 0.
\end{multline}
The integrals over the domain $\Omega$ with $\trialvec[v^+]$ and
$\trialvec[v^-]$ are effectively integrals over the faces of cells
adjacent to the fault. We choose to write them as integrals over the
domain $\Omega$ rather than over the fault surface because the density
$\rho(\vec{x})$ is defined only within the domain cells.

\subsection{Residual Pointwise Kernels}

%Combining the explicit and implict parts and identifying $F(t,s,\dot{s})$ and $G(t,s)$, we have
%\begin{align}
  % Gu
  % Gv
%\end{align}



% End of file