\section{Multiphysics Finite-Element Formulation}
\label{sec:multiphysics:formulation}

Within the PETSc solver framework, we want to solve a system of
partial differential equations in which the weak form can be
expressed as $F(t,s,\dot{s}) = G(t,s)$, $s(t_0) = s_0$, where $F$ and
$G$ are vector functions, $t$ is time, and $s$ is the solution vector.

Using the finite-element method we manipulate the weak form of the
system of equations involving a vector field $\vec{u}$ into integrals
over the domain $\Omega$ matching the form,
\begin{equation}
  \label{eqn:problem:form}
  \int_\Omega \trialvec[u] \cdot \vec{f}_0(t,s,\dot{s}) + \nabla \trialvec[u] : \tensor{f}
_1(t,s,\dot{s}) \, 
d\Omega =
  \int_\Omega \trialvec[u] \cdot \vec{g}_0(t,s) + \nabla \trialvec[u] : \tensor{g}_1(t,s) \, 
d\Omega,
\end{equation}
where $\trialvec[u]$ is the trial function for field $\vec{u}$,
$\vec{f}_0$ and $\vec{g}_0$ are vectors, and $\tensor{f}_1$ and
$\tensor{g}_1$ are tensors. With multiple partial differential
equations we will have multiple equations of this form, and the
solution vector $s$, which we usually write as $\vec{s}$, will be
composed of several different fields, such as displacement $\vec{u}$,
velocity $\vec{v}$, pressure $p$, and temperature $T$. Boundary
conditions will also contribute similar terms with integrals over the
corresponding boundaries.

For consistency with the PETSc time stepping formulation, we call
$G(t,s)$ the RHS function and call $F(t,s,\dot{s})$ the LHS (or I)
function. Likewise, the Jacobian of $G(t,s)$ is the RHS Jacobian and
the Jacobian of $F(t,s,\dot{s})$ is the LHS Jacobian. In most cases,
we can take $F(t,s,\dot{s}) = \dot{s}$, or as close to this as
possible. This results in miminal changes to the formulation in order
to accommodate both implicit and explicit time stepping algorithms.

Using a finite-element discretization we break up the domain and
boundary integrals into sums over the cells and boundary faces/edges,
respectively. Using numerical quadrature those sums in turn involve
sums over the values at the quadrature points with appropriate
weights. Thus, computation of the RHS function boils down to
pointwise evaluation of $\vec{g}_0(t,s)$ and $\tensor{g}_1(t,s)$, and
computation of the LHS function boils down to pointwise evaluation of
$\vec{f}_0(t,s,\dot{s})$ and $\tensor{f}_1(t,s,\dot{s})$.

\subsection{Jacobian}

The LHS Jacobian $J_F = \frac{\partial F}{\partial s} +
s_\mathit{tshift} \frac{\partial F}{\partial \dot{s}}$ and the RHS
Jacobian $J_G = \frac{\partial G}{\partial s}$, where
$s_\mathit{tshift}$ arises from the temporal discretization. We put
the Jacobians for each equation into the form:
\begin{align}
  \label{eqn:jacobian:form}
  J_F &= \int_\Omega \trialvec \cdot \tensor{J}_{f0}(t,s,\dot{s}) \cdot \basisvec
  + \trialvec \cdot \tensor{J}_{f1}(t,s,\dot{s}) : \nabla \basisvec
  + \nabla \trialvec : \tensor{J}_{f2}(t,s,\dot{s}) \cdot \basisvec
  + \nabla \trialvec : \tensor{J}_{f3}(t,s,\dot{s}) : \nabla \basisvec \, d\Omega \\
%
  J_G &= \quad \int_\Omega \trialvec \cdot \tensor{J}_{g0}(t,s) \cdot \basisvec
  + \trialvec \cdot \tensor{J}_{g1}(t,s) : \nabla \basisvec
  + \nabla \trialvec : \tensor{J}_{g2}(t,s) \cdot \basisvec
  + \nabla \trialvec : \tensor{J}_{g3}(t,s) : \nabla \basisvec \, d\Omega,
\end{align}
where $\basisvec$ is a basis function.  Expressed in index notation
the Jacobian coupling solution field components $s_i$ and $s_j$ is
\begin{equation}
\label{eqn:jacobian:index:form}
J^{s_is_j} = \int_\Omega \trialscalar_i J_0^{s_is_j} \basisscalar_j + \trialscalar_i 
J_1^{s_js_jl} 
\basisscalar_{j,l} + \trialscalar_{i,k} J_2^{s_is_jk} \basisscalar_j + \trialscalar_{i,k} 
J_3^{s_is_jkl} 
\basisscalar_{j,l} \, d\Omega, 
\end{equation}
It is clear that the tensors $J_0$, $J_1$, $J_2$, and $J_3$ have
various sizes: $J_0(n_i,n_j)$, $J_1(n_i,n_j,d)$, $J_2(n_i,n_j,d)$,
$J_3(n_i,n_j,d,d)$, where $n_i$ is the number of components in
solution field $s_i$, $n_j$ is the number of components in solution
field $s_j$, and $d$ is the spatial dimension.  Alternatively,
expressed in discrete form, the Jacobian for the coupling between
solution fields $s_i$ and $s_j$ is
\begin{equation}
  \label{eqn:jacobian:discrete:form}
  J^{s_is_j} = J_{0}^{s_is_j} + J_{1}^{s_is_j} B + B^T J_{2}^{s_is_j} + B^T J_{3}^{s_is_j} B,
\end{equation}
where $B$ is a matrix of the derivatives of the basis functions and $B^T$
is a matrix of the derivatives of the trial functions. 

\important{See
  \url{https://www.mcs.anl.gov/petsc/petsc-master/docs/manualpages/FE/PetscFEIntegrateJacobian.html}
  for the ordering of indicies in the Jacobian pointwise functions.}

\subsection{PETSc TS Notes}

\begin{itemize}
\item If no LHS (or I) function is given, then the PETSc TS assumes $F(t,s,\dot{s}) = \dot{s}
$.
\item If no RHS function is given, then the PETSc TS assumes $G(t,s) = 0$.
\item Explicit time stepping with the PETSc TS requires
  $F(t,s,\dot{s}) = \dot{s}$.
  \begin{itemize}
  \item Because $F(t,s,\dot{s}) = \dot{s}$, we do not specify the
    functions $\vec{f}_0(t,s,\dot{s})$ and $\tensor{f}_1(t,s,\dot{s})$
    because the PETSc TS will assume this is the case if no LHS (or I)
    function is given.
  \item We also do not specify $J_F$ or $J_G$.
  \item This leaves us with only needing to specify $\vec{g}_0(t,s)$
    and $\tensor{g}_1(t,s)$. 
  \item The PETSc TS will verify that the LHS (or I) function is null.
  \end{itemize}
\end{itemize}

For explicit time stepping with the PETSc TS, we need
$F(t,s,\dot{s}) = \dot{s}$. Using a finite-element formulation for
elastodynamics, $F(t,s,\dot{s})$ generally involves integrals of the
inertia over the domain. It is tempting to simply move these terms to
the RHS, but that introduces inertial terms into the boundary
conditions, which makes them less intuitive. Instead, we transform our
equation into the form $\dot{s} = G^*(t,s)$ where
$G^*(t,s) = M^{-1} G(t,s)$. We take $M$ to be a lumped (diagonal)
matrix, so that $M^{-1}$ is trivial to compute. In computing the RHS
function, $G^*(t,s)$, we compute $G(t,s)$, then compute $M$ and
$M^{-1}$, and then $M^{-1}G(t,s)$. For the elasticity equation with
inertial terms, $M$ contains the mass matrix.
