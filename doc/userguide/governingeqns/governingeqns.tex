\chapter{Governing Equations}
\label{cha:governing:equations}

We present here a brief derivation of the equations for both quasi-static
and dynamic computations. Since the general equations are the same
(except for the absence of inertial terms in the quasi-static case),
we first derive these equations. We then present solution methods
for each specific case. In all of our derivations, we use the notation
described in Table \vref{tab:notation} for both index
and vector notation.

\begin{table}[htbp]
  \caption{Mathematical notation}
  \label{tab:notation}
  \begin{tabular}{ccp{3in}}
    \multicolumn{2}{c}{{\bf Symbol}} & {\bf Description} \\
    {\bf Index notation} & {\bf Vector Notation} & \\
    \hline 
    $a_{i}$ & \raisebox{12pt}{}$\overrightarrow{a}$ & Vector field a \\
    $a_{ij}$ & $\underline{a}$ & Second order tensor field a \\
    $u_{i}$ & $\overrightarrow{u}$ & Displacement vector field \\
    $d_{i}$ & $\vec{{d}}$ & Fault slip vector field \\
    $f_{i}$ & $\overrightarrow{f}$ & Body force vector field \\
    $T_{i}$ & $\overrightarrow{\tau}$ & Traction vector field \\
    $\sigma_{ij}$ & $\underline{\sigma}$ & Stress tensor field \\
    $n_{i}$ & $\overrightarrow{n}$ & Normal vector field \\
    $\rho$ & $\rho$ & Mass density scalar field \\
    \hline 
  \end{tabular}
\end{table}

\section{Derivation of Elasticity Equation}

\subsection{Vector Notation}

Consider volume $V$ bounded by surface $S$. Applying a Lagrangian
description of the conservation of momentum gives
\begin{equation}
\label{eqn:momentum:vec}
\frac{\partial}{\partial t}\int_{V}\rho\frac{\partial\vec{u}}{\partial t}\, dV=\int_{V}\vec{f}\, dV+\int_{S}\vec{\tau}\, dS.
\end{equation}
The traction vector field is related to the stress tensor through
\begin{equation}
\vec{\tau}=\underline{\sigma}\cdot\vec{n},
\end{equation}
where $\vec{n}$ is the vector normal to $S$. Substituting
into equation \vref{eqn:momentum:vec} yields
\begin{equation}
\frac{\partial}{\partial t}\int_{V}\rho\frac{\partial\vec{u}}{\partial t}\, dV=\int_{V}\vec{f}\, dV+\int_{S}\underline{\sigma}\cdot\vec{n}\, dS.
\end{equation}
Applying the divergence theorem,
\begin{equation}
\int_{V}\nabla\cdot\vec{a}\: dV=\int_{S}\vec{a}\cdot\vec{n}\: dS,
\end{equation}
to the surface integral results in
\begin{equation}
\frac{\partial}{\partial t}\int_{V}\rho\frac{\partial\vec{u}}{\partial t}\, dV=\int_{V}\vec{f}\, dV+\int_{V}\nabla\cdot\underline{\sigma}\, dV,
\end{equation}
which we can rewrite as
\begin{equation}
\int_{V}\left(\rho\frac{\partial^{2}\vec{u}}{\partial t^{2}}-\vec{f}-\nabla\cdot\vec{\sigma}\right)\, dV=\vec{0}.
\end{equation}
Because the volume $V$ is arbitrary, the integrand must be the zero
vector at every location in the volume, so that we end up with
\begin{gather}
\rho\frac{\partial^{2}\vec{u}}{\partial t^{2}}-\vec{f}-\nabla\cdot\vec{\sigma}=\vec{0}\text{ in }V,\\
\underline{\sigma}\cdot\vec{n}=\vec{\tau}\text{ on }S_{\tau}\text{,}\\
\vec{u}=\vec{u^{o}}\text{ on }S_{u},\text{ and}\\
\underbar{R}\cdot(\vec{u^{+}}-\vec{u^{-}})=\vec{d}\text{ on }S_{f}.
\end{gather}
We specify tractions, $\vec{\tau}$, on surface $S_{f}$, displacements,
$\vec{u^{o}}$, on surface $S_{u}$, and slip, $\vec{d}$,
on fault surface $S_{f}$ (we will consider the case of fault constitutive
models in Section \vref{sec:fault}). The rotation matrix $\underline{R}$
transforms vectors from the global coordinate system to the fault
coordinate system. Note that since both $\vec{\tau}$ and
$\vec{u}$ are vector quantities, there can be some spatial
overlap of the surfaces $S_{\tau}$ and $S_{u}$; however, the same degree
of freedom cannot simultaneously have both types of boundary conditions.

\section{Time-Dependent Problem (\facilityshape{formulation})}

This type of problem applies to transient static, quasi-static, and
dynamic simulations. The time-dependent problem adds the
\facility{formulation} facility to the general-problem. The
formulation specifies the time-stepping formulation to integrate the
elasticity equation. PyLith provides several alternative formulations,
each specific to a different type of problem.
\begin{description}
\item[\object{Implicit}] Implicit time stepping for static and
   quasi-static problems with infinitesimal strains. The implicit
   formulation neglects inertial terms (see Section
   \vref{eq:elasticity:integral:quasistatic}).
\item[\object{ImplicitLgDeform}] Implicit time stepping for static
  and quasi-static problems including the effects of rigid body motion
  and small strains.  This formulation requires the use of the
  nonlinear solver, which is selected automatically.
\item[\object{Explicit}] Explicit time stepping for dynamic problems
  with infinitesimal strains and lumped system Jacobian. The cell
  matrices are lumped before assembly, permitting use of a vector for
  the diagonal system Jacobian matrix. The built-in lumped solver is
  selected automatically.
\item[\object{ExplicitLgDeform}] Explicit time stepping for dynamic
  problems including the effects of rigid body motion and small
  strains. The cell matrices are lumped before assembly, permitting
  use of a vector for the diagonal system Jacobian matrix. The
  built-in lumped solver is selected automatically.
\item[\object{ExplicitTri3}] Optimized elasticity formulation for
  linear triangular cells with one point quadrature for dynamic
  problems with infinitesimal strains and lumped system Jacobian. The
  built-in lumped solver is selected automatically.
\item[\object{ExplicitTet4}] Optimized elasticity formulation for
  linear tetrahedral cells with one point quadrature for dynamic
  problems with infinitesimal strains and lumped system Jacobian. The
  built-in lumped solver is selected automatically.
\end{description}
In many quasi-static simulations it is convenient to compute a static
problem with elastic deformation prior to computing a transient response.
Up through PyLith version 1.6 this was hardwired into the Implicit
Forumulation as advancing from time step $t=-\Delta t$ to $t=0$,
and it could not be turned off. PyLith now includes a property, \property{elastic\_prestep}
in the TimeDependent component to turn on/off this behavior (the default
is to retain the previous behavior of computing the elastic deformation).

\warning{Turning off the elastic
prestep calculation means the model only deforms when an {\it increment}
in loading or deformation is applied, because the time-stepping formulation
is implemented using the increment in displacement.}

The \object{TimeDependent} properties and facilities include
\begin{inventory}
  \propertyitem{elastic\_preset}{If true, perform a static calculation with elastic
    behavior before time stepping (default is True).}
  \facilityitem{formulation}{Formulation for solving the partial differential
    equation.}
\end{inventory}
An example of setting the properties and components in a \filename{.cfg} file
is
\begin{cfg}
<h>[pylithapp.timedependent]</h>
<f>formulation</f> = pylith.problems.Implicit ; default
<f>progres_monitor</f> = pylith.problems.ProgressMonitorTime ; default
<p>elastic_preset</p> = True ; default
\end{cfg}
The formulation value can be set to the other formulations in a similar
fashion. 


\subsection{Time-Stepping Formulation}

The explicit and implicit time stepping formulations use a common
set of facilities and properties. The properties and facilities include
\begin{inventory}
\propertyitem{matrix\_type}{Type of PETSc matrix for the system Jacobian (sparse
matrix, default is symmetric, block matrix with a block size of 1).}
\propertyitem{view\_jacobian}{Flag to indicate if system Jacobian (sparse matrix)
should be written to a file (default is false).}
\propertyitem{split\_fields}{Split solution field into a displacement portion
(fields 0..ndim-1) and a Lagrange multiplier portion (field ndim)
to permit application of sophisticated PETSc preconditioners (default
is false).}
\facilityitem{time\_step}{Time step size specification (default is \object{TimeStepUniform} (uniform time step).}
\facilityitem{solver}{Type of solver to use (default is \object{SolverLinear}).}
\facilityitem{output}{Array of output managers for output of the solution (default
is [output]).}
\facilityitem{jacobian\_viewer}{Viewer to dump the system Jacobian (sparse matrix)
to a file for analysis (default is PETSc binary).}
\end{inventory}

An example of setting these parameters in a \filename{.cfg} file is
\begin{cfg}
<h>[pylithapp.timedependent.formulation]</h>
<p>matrix_type</p> = sbaij ; Non-symmetric sparse matrix is 'aij'
<p>view_jacobian</p> = false

# Nonlinear solver is pylith.problems.SolverNonlinear
<f>solver</f> = pylith.problems.SolverLinear
<f>output</f> = [domain, ground_surface]
<f>time_step</f> = pylith.problems.TimeStepUniform
\end{cfg}

\subsection{Numerical Damping in Explicit Time Stepping}

In explicit time-stepping formulations for elasticity, boundary conditions
and fault slip can excite short waveform elastic waves that are not
accurately resolved by the discretization. We use numerical damping
via an artificial viscosity\cite{Knopoff:Ni:2001,Day:Ely:2002} to
reduce these high frequency oscillations. In computing the strains
for the elasticity term in equation \vref{eq:elasticity:integral:dynamic:t},
we use an adjusted displacement rather than the actual displacement,
where 
\begin{equation}
\vec{u}^{adj}(t)=\vec{u}(t)+\eta^{*}\Delta t\vec{\dot{u}}(t),
\end{equation}
$\vec{u}^{adj}(t)$ is the adjusted displacement at time t, $\vec{u}(t)$is
the original displacement at time (t), $\eta^{*}$is the normalized
artificial viscosity, $\Delta t$ is the time step, and $\vec{\dot{u}}(t)$
is the velocity at time $t$. The default value for the normalized
artificial viscosity is 0.1. We have found values in the range 0.1-0.4
sufficiently suppress numerical noise while not excessively reducing
the peak velocity. An example of setting the normalized artificial
viscosity in a \filename{.cfg} file is
\begin{cfg}
<h>[pylithapp.timedependent.formulation]</h>
<p>norm_viscosity</p> = 0.2
\end{cfg}

\subsection{Solvers}
\label{sec:solvers}

PyLith supports three types of solvers. The linear solver,
SolverLinear, corresponds to the PETSc KSP solver and is used in
linear problems with linear elastic and viscoelastic bulk constitutive
models and kinematic fault ruptures. The nonlinear solver,
SolverNonlinear, corresponds to the PETSc SNES solver and is used in
nonlinear problems with nonlinear viscoelastic or elastoplastic bulk
constitutive models, dynamic fault ruptures, or problems involving
finite strain (small strain formulation).  The lumped solver
(SolverLumped) is a specialized solver used with the lumped system
Jacobian matrix. The options for the PETSc KSP and SNES solvers are
set via the top-level PETSc options (see Section
\vref{sec:petsc:options} and the PETSc documentation
\url{www.mcs.anl.gov/petsc/petsc-as/documentation/index.html}).


\subsection{Time Stepping}
\label{sub:Time-Stepping}

PyLith provides three choices for controlling the time step in time-dependent
simulations. These include (1) a uniform, user-specified time step
(which is the default), (2) user-specified time steps (potentially
nonuniform), and (3) automatically calculated (potentially nonuniform)
time steps. The procedure for automatically selecting time steps requires
that the material models provide a reasonable estimate of the time
step for stable time integration. In general, quasi-static simulations
with viscoelastic materials should use automatically calculated time
steps and dynamic simulations should use a uniform, user-specified
time step. Note that all three of the time stepping schemes make use
of the computed stable time step (see \vref{sec:stable:time:step}).
When using user-specified time steps, the value is checked against
the computed stable time step. The automatically calculated time step
comes from the computed stable time step.

\warning{Varying the time step within a simulation requires
  recomputing the Jacobian of the system whenever the time step
  changes, which can greatly increase the runtime if the time-step
  size changes frequently.}

\subsubsection{Uniform, User-Specified Time Step (\object{TimeStepUniform})}

With a uniform, user-specified time step, the user selects the time
step that is used over the entire duration of the simulation. If this
value exceeds the computed stable time step at any time, PyLith will
terminate with an error. The properties for the uniform, user-specified
time step are:
\begin{inventory}
\propertyitem{total\_time}{Time duration for simulation (default is 0.0 s).}
\propertyitem{start\_time}{Start time for simulation (default is 0.0 s).}
\propertyitem{dt}{Time step for simulation.}
\end{inventory}
An example of setting a uniform, user-specified time step in a \filename{.cfg}
file is:
\begin{cfg}
<h>[pylithapp.problem.formulation]</h>
<p>time_step</p> = pylith.problems.TimeStepUniform ; Default value

<h>[pylithapp.problem.formulation.time_step]</h>
<p>total_time</p> = 1000.0*year
<p>dt</p> = 0.5*year
\end{cfg}

\subsubsection{Nonuniform, User-Specified Time Step (\object{TimeStepUser})}

The nonuniform, user-specified, time-step implementation allows the
user to specify the time steps in an ASCII file (see Section
\vref{sec:format:timestepuser} for the format specification of the
time-step file). If the total duration exceeds the time associated
with the time steps, then a flag determines whether to cycle through
the time steps or to use the last specified time step for the time
remaining. Similar to the uniform time step, if the user-specified
time step size exceeds the computed stable time step at any time,
PyLith will terminate with an error.  The properties for the
nonuniform, user-specified time step are:
\begin{inventory}
\propertyitem{total\_time}{Time duration for simulation.}
\propertyitem{filename}{Name of file with time-step sizes.}
\propertyitem{loop\_steps}{If true, cycle through time steps, otherwise keep
using last time-step size for any time remaining.}
\end{inventory}
An example of setting the properties for nonuniform, user-specified
time steps in a \filename{.cfg} file is:
\begin{cfg}
<h>[pylithapp.problem.formulation]</h>
<f>time_step</f> = pylith.problems.TimeStepUser ; Change the time step algorithm

<h>[pylithapp.problem.formulation.time_step]</h>
<p>total_time</p> = 1000.0*year
<p>filename</p> = timesteps.txt
<p>loop_steps</p> = false ; Default value
\end{cfg}

\subsubsection{Nonuniform, Automatic Time Step (\object{TimeStepAdapt})}

This time-step implementation automatically calculates a time step
size based on the constitutive model and rate of deformation. As a
result, this choice for choosing the time step relies on accurate
calculation of a stable time step within each finite-element cell
by the constitutive models. To provide some control over the time-step
selection, the user can control the frequency with which a new time
step is calculated, the time step to use relative to the value determined
by the constitutive models, and a maximum value for the time step.
Note that the stability factor allows the computed time step size
to exceed the computed stable time step. A stability factor of 1.0
would provide a time step size equal to the stable time step, while
a value of 2.0 (default value) would provide a time step size equal
to 1/2 the stable time step. Caution should be used when adjusting
the stability factor to values less than 1.0, as the large time step
size may result in inaccurate solutions. The properties for controlling
the automatic time-step selection are:
\begin{inventory}
\propertyitem{total\_time}{Time duration for simulation.}
\propertyitem{max\_dt}{Maximum time step permitted.}
\propertyitem{adapt\_skip}{Number of time steps to skip between calculating
new stable time step.}
\propertyitem{stability\_factor}{Safety factor for stable time step (default
is 2.0).}
\end{inventory}
An example of setting the properties for the automatic time step in
a \filename{.cfg} file is:
\begin{cfg}
<h>[pylithapp.problem.formulation]</h>
<p>time_step</p> = pylith.problems.TimeStepAdapt ; Change the time step algorithm

<h>[pylithapp.problem.formulation.time_step]</h>
<p>total_time</p> = 1000.0*year
<p>max_dt</p> = 10.0*year
<p>adapt_skip</p> = 10 ; Default value
<p>stability_factor</p> = 2.0 ; Default value
\end{cfg}

\section{Green's Functions Problem (\object{GreensFns})}

This type of problem applies to computing static Green's functions
for elastic deformation. The \object{GreensFns} problem specializes
the time-dependent facility to the case of static simulations with
slip impulses on a fault. The default formulation is the Implicit
formulation and should not be changed as the other formulations are
not applicable to static Green's functions. In the output files, the
deformation at each ``time step'' is the deformation for a different
slip impulse. The properties provide the ability to select which fault
to use for slip impulses. The only fault component available for use
with the \object{GreensFns} problem is the \object{FaultCohesiveImpulses}
component discussed in Section \vref{sec:fault:cohesive:impulses}.
The \object{GreensFns} properties amd facilities include:
\begin{inventory}
\propertyitem{fault\_id}{Id of fault on which to impose slip impulses.}
\propertyitem{formulation}{Formulation for solving the partial differential
equation.}
\propertyitem{progress\_monitor}{Simple progress monitor via text file.}
\end{inventory}
An example of setting the properties for the GreensFns problem in
a \filename{.cfg} file is:
\begin{cfg}
<h>[pylithapp]</h>
<f>problem</f> = pylith.problems.GreensFns ; Change problem type from the default

<h>[pylithapp.greensfns]</h>
<p>fault_id</p> = 100 ; Default value
<f>formulation</f> = pylith.problems.Implicit ; default
<f>progres_monitor</f> = pylith.problems.ProgressMonitorTime ; default
\end{cfg}

\warning{The \object{GreensFns} problem generates slip impulses on a
  fault. The current version of PyLith requires that impulses can only
  be applied to a single fault and the fault facility must be set to
  \object{FaultCohesiveImpulses}.}

\section{Progress Monitors}
\newfeature{v2.1.0}

The progress monitors make it easy to monitor the general progress of
long simulations, especially on clusters where stdout is not always
easily accessible. The progress monitors update a simulation's current
progress by writing information to a text file. The information
includes time stamps, percent completed, and an estimate of when the
simulation will finish.

\subsection{\object{ProgressMonitorTime}}

This is the default progress monitor for time-stepping problems. The
monitor calculates the percent completed based on the time at the
current time step and the total simulated time of the simulation,
not the total number of time steps (which may be unknown in simulations
with adaptive time stepping). The \object{ProgressMonitorTime} properties
include:
\begin{inventory}
\propertyitem{update\_percent}{Frequency (in percent) of progress updates.}
\propertyitem{filename}{Name of output file.}
\propertyitem{t\_units}{Units for simulation time in output.}
\end{inventory}
An example of setting the properties in a \filename{.cfg} file is:
\begin{cfg}
<h>[pylithapp.problem.progressmonitor]</h>
<p>update_percent</p> = 5.0 ; default
<p>filename</p> = progress.txt ; default
<p>t_units</p> = year ; default
\end{cfg}

\subsection{\object{ProgressMonitorStep}}

This is the default progress monitor for problems with a specified
number of steps, such as Green's function problems. The monitor calculates
the percent completed based on the number of steps (e.g., Green's
function impulses completed). The ProgressMonitorStep propertiles
include:
\begin{inventory}
\propertyitem{update\_percent}{Frequency (in percent) of progress updates.}
\propertyitem{filename}{Name of output file.}
\end{inventory}
An example of setting the properties in a \filename{.cfg} file is:
\begin{cfg}
<h>[pylithapp.problem.progressmonitor]</h>
<p>update_percent</p> = 5.0 ; default
<p>filename</p> = progress.txt ; default
\end{cfg}

% End of file

% ----------------------------------------------------------------------
\section{Elasticity with Infinitesimal Strain and No Faults}

We begin with the elasticity equation including the inertial term,
\begin{gather}
  \label{eqn:elasticity:strong:form}
  \rho \frac{\partial^2\vec{u}}{\partial t^2} - \vec{f}(\vec{x},t) - \tensor{\nabla} \cdot 
\tensor{\sigma}
(\vec{u}) = \vec{0} \text{ in }\Omega, \\
%
  \label{eqn:bc:Neumann}
  \tensor{\sigma} \cdot \vec{n} = \vec{\tau}(\vec{x},t) \text{ on }\Gamma_\tau, \\
%
  \label{eqn:bc:Dirichlet}
  \vec{u} = \vec{u}_0(\vec{x},t) \text{ on }\Gamma_u,
\end{gather}
where $\vec{u}$ is the displacement vector, $\rho$ is the mass
density, $\vec{f}$ is the body force vector, $\tensor{\sigma}$ is the
Cauchy stress tensor, $\vec{x}$ is the spatial coordinate, and $t$ is
time. We specify tractions $\vec{\tau}$ on boundary $\Gamma_\tau$, and
displacements $\vec{u}_0$ on boundary $\Gamma_u$. Because both $\vec{\tau}$
and $\vec{u}$ are vector quantities, there can be some spatial overlap
of boundaries $\Gamma_\tau$ and $\Gamma_u$; however, a degree of freedom at
any location cannot be associated with both prescribed displacements
(Dirichlet) and traction (Neumann) boundary conditions simultaneously.

\begin{table}[htbp]
  \caption{Mathematical notation for elasticity equation with
    infinitesimal strain.}
  \label{tab:notation:elasticity}
  \begin{tabular}{lcp{3in}}
    \toprule
    {\bf Category} & {\bf Symbol} & {\bf Description} \\
    \midrule
    Unknowns & $\vec{u}$ & Displacement field \\
    & $\vec{v}$ & Velocity field \\
    Derived quantities & $\tensor{\sigma}$ & Cauchy stress tensor \\
                   & $\tensor{\epsilon}$ & Cauchy strain tensor \\
    Common constitutive parameters & $\rho$ & Density \\
  & $\mu$ & Shear modulus \\
  & $K$ & Bulk modulus \\
Source terms & $\vec{f}$ & Body force per unit volume, for example $\rho \vec{g}$ \\
    \bottomrule
  \end{tabular}
\end{table}


\subsection{Quastistatic}

If we neglect the inertial term
($\rho \frac{\partial \vec{v}}{\partial t} \approx \vec{0}$), then
time dependence only arises from history-dependent constitutive
equations and boundary conditions. Our solution vector is the
displacement vector and the elasticity equation reduces to
\begin{gather}
  \label{eqn:elasticity:strong:form:quasistatic}
  \vec{f}(\vec{x},t) + \tensor{\nabla} \cdot \tensor{\sigma}(\vec{u}) = \vec{0} \text{ in }\Omega, \\
%
  \tensor{\sigma} \cdot \vec{n} = \vec{\tau}(\vec{x},t) \text{ on }\Gamma_\tau, \\
%
  \vec{u} = \vec{u}_0(\vec{x},t) \text{ on }\Gamma_u.
\end{gather}
Because we will use implicit time stepping, we place all of the terms
in the elasticity equation on the LHS. We create the weak form by
taking the dot product with the trial function $\trialvec[u]$ and
integrating over the domain:
\begin{equation}
    \int_\Omega \trialvec[u] \cdot \left( \vec{f}(t) + \tensor{\nabla}
      \cdot \tensor{\sigma} (\vec{u}) \right) \, d\Omega = 0. 
\end{equation}
Using the divergence theorem and incorporating the Neumann boundary conditions, we have
\begin{equation}
% 
  \int_\Omega \trialvec[u] \cdot \vec{f}(\vec{x},t) + \nabla \trialvec[v] : -\tensor{\sigma}(\vec{u}) \, d\Omega
  + \int_{\Gamma_\tau} \trialvec[v] \cdot \vec{\tau}(\vec{x},t) \, d\Gamma = 0
\end{equation}

\subsubsection{Residual Pointwise Functions}

Identifying $F(t,s,\dot{s})$ and $G(t,s)$, we have
\begin{align}
  % Fu
  F^u(t,s,\dot{s}) &=  \int_\Omega \trialvec[u] \cdot \eqnannotate{\vec{f}(\vec{x},t)}{\vec{f}^u_0} + \nabla \trialvec[u] : \eqnannotate{-\tensor{\sigma}(\vec{u})}{\tensor{f^u_1}} \, d\Omega
  + \int_{\Gamma_\tau} \trialvec[u] \cdot \eqnannotate{\vec{\tau}(\vec{x},t)}{\vec{f}^u_0} \, d\Gamma, \\
  % Gu
  G^u(t,s) &= 0
\end{align}
Note that we have multiple $\vec{f}_0$ functions, each associated with
a trial function and an integral over a different domain or
boundary. Each material and boundary condition (except Dirichlet)
contribute pointwise functions. The integral over the domain $\Omega$
is subdivided into integrals over the materials and the integral over
the boundary $\Gamma_\tau$ is subdivided into integrals over the
Neumann boundaries. Each bulk constitutive model provides a different
pointwise function for the stress tensor
$\tensor{\sigma}(\vec{u})$. With $G=0$ it is clear that we have a
formulation that will use implicit time stepping algorithms.

\subsubsection{Jacobian Pointwise Functions}

We only have a Jacobian for the LHS:
\begin{align}
  J_F^{uu} &= \frac{\partial F^u}{\partial u} = \int_\Omega \nabla \trialvec[u] : 
\frac{\partial}{\partial u}(-\tensor{\sigma}) \, d\Omega 
  = \int_\Omega \nabla \trialvec[u] : -\tensor{C} : \frac{1}{2}(\nabla + \nabla^T)\basisvec[u] 
\, d\Omega 
  = \int_\Omega \trialscalar[u]_{i,k} \, \eqnannotate{\left( -C_{ikjl} \right)}{J_{f3}^{uu}} \, \basisscalar[u]_{j,l}\, d\Omega.
\end{align}


\subsection{Dynamic}

For compatibility with PETSc TS algorithms, we want to turn
the second order equation~\vref{eqn:elasticity:strong:form} into two first order
equations. We introduce the velocity as a unknown,
$\vec{v}=\frac{\partial u}{\partial t}$, which leads to
\begin{align}
  % Displacement-velocity
  \frac{\partial \vec{u}}{\partial t} &= \vec{v} \text{ in } \Omega, \\
  % Elasticity
  \rho(\vec{x}) \frac{\partial\vec{v}}{\partial t} &= \vec{f}(\vec{x},t) + \tensor{\nabla} \cdot \tensor{\sigma}(\vec{u}) \text{ in } \Omega, \\
  % Neumann
  \tensor{\sigma} \cdot \vec{n} &= \vec{\tau}(\vec{x},t) \text{ on } \Gamma_\tau, \\
  % Dirichlet
  \vec{u} &= \vec{u}_0(\vec{x},t) \text{ on } \Gamma_u.
\end{align}
We create the weak form by taking the dot product with the trial
function $\trialvec[u]$ or $\trialvec[v]$ and
integrating over the domain:
\begin{gather}
  % Displacement-velocity
  \int_\Omega \trialvec[u] \cdot \frac{\partial \vec{u}}{\partial t} \, d\Omega = 
  \int_\Omega \trialvec[u] \cdot \vec{v} \, d\Omega, \\
  % Elasticity
    \int_\Omega \trialvec[v] \cdot \rho(\vec{x}) \frac{\partial \vec{v}}{\partial t} \, d\Omega 
 = \int_\Omega \trialvec[v] \cdot \left( \vec{f}(t) + \tensor{\nabla} \cdot \tensor{\sigma} (\vec{u}) \right) \, d\Omega.
\end{gather}
Using the divergence theorem and incorporating the Neumann boundaries, we can rewrite the second equation as
\begin{equation}
% 
  \int_\Omega \trialvec[v] \cdot \rho(\vec{x}) \frac{\partial \vec{v}}{\partial t} \, d\Omega
  = \int_\Omega \trialvec[v] \cdot \vec{f}(\vec{x},t) + \nabla \trialvec[v] : -\tensor{\sigma}(\vec{u}) \, d\Omega
  + \int_{\Gamma_\tau} \trialvec[v] \cdot \vec{\tau}(\vec{x},t) \, d\Gamma.
\end{equation}

For explicit time stepping, we want $F(t,s,\dot{s})=\dot{s}$, so we
solve an augmented system in which we multiply the RHS residual
function by the inversion of the lumped LHS Jacobian,
\begin{gather}
  F^*(t,s,\dot{s}) = G^*(t,s) \text{, where} \\
  F^*(t,s,\dot{s}) = \dot{s} \text{ and} \\
  G^*(t,s) = J_F^{-1} G(t,s).
\end{gather}
With the augmented system, we have
\begin{gather}
  % Displacement-velocity
  \frac{\partial \vec{u}}{\partial t}  = M_u^{-1} \int_\Omega \trialvec[u] \cdot \vec{v} \, d\Omega, \\
  % Elasticity
  \frac{\partial \vec{v}}{\partial t} = M_v^{-1} \int_\Omega \trialvec[v] \cdot \left( \vec{f}(t) + \tensor{\nabla} \cdot \tensor{\sigma} (\vec{u}) \right) \, d\Omega, \\
  % Mu
  M_u = \mathit{Lump}\left( \int_\Omega \trialscalar[u]_i \delta_{ij} \basisscalar[u]_j \, d\Omega \right), \\
  % Mv
  M_v = \mathit{Lump}\left( \int_\Omega \trialscalar[v]_i \rho(\vec{x}) \delta_{ij} \basisscalar[v]_j \, d\Omega \right).
\end{gather}

\subsubsection{Residual Pointwise Functions}

With explicit time stepping the PETSc TS assumes the LHS is
$\dot{s}$, so we only need the RHS residual functions:
\begin{align}
  % Gu
  G^u(t,s) &= \int_\Omega \trialvec[u] \cdot \eqnannotate{\vec{v}}{\vec{g}^u_0} \, d\Omega, \\
  % Gv
  G^v(t,s) &=  \int_\Omega \trialvec[v] \cdot \eqnannotate{\vec{f}(\vec{x},t)}{\vec{g}^v_0} + \nabla \trialvec[v] : \eqnannotate{-\tensor{\sigma}(\vec{u})}{\tensor{g^v_1}} \, d\Omega
  + \int_{\Gamma_\tau} \trialvec[v] \cdot \eqnannotate{\vec{\tau}(\vec{x},t)}{\vec{g}^v_0} \, d\Gamma,
\end{align}
In the second equation these are the same pointwise functions as in the
quasistatic case, only now they are on the RHS instead of the LHS.


\subsubsection{Jacobian Pointwise Functions}

These are the pointwise functions associated with $M_u$ and $M_v$ for
computing the lumped LHS Jacobian. We premultiply the RHS residual
function by the inverse of the lumped LHS Jacobian while
$s_\mathit{tshift}$ remains on the LHS with $\dot{s}$. As a result, we
use LHS Jacobian pointwise functions, but set $s_\mathit{tshift}=1$.
The LHS Jacobians are:
\begin{align}
  % J_F uu
  M_u = J_F^{uu} &= \frac{\partial F^u}{\partial u} + s_\mathit{tshift} \frac{\partial F^u}{\partial \dot{u}} =
             \int_\Omega \trialscalar[u]_i \eqnannotate{s_\mathit{tshift} \delta_{ij}}{J^{uu}_{f0}} \basisscalar[u]_j  \, d\Omega, \\
  % J_F vv
  M_v = J_F^{vv} &= \frac{\partial F^v}{\partial v} + s_\mathit{tshift} \frac{\partial F^v}{\partial \dot{v}} =
             \int_\Omega \trialscalar[v]_i \eqnannotate{\rho(\vec{x}) s_\mathit{tshift} \delta_{ij}}{J ^{vv}_{f0}} \basisscalar[v]_j \, d\Omega
\end{align}


% End of file
% ----------------------------------------------------------------------
\section{Elasticity with Infinitesimal Strain and Prescribed Slip on Faults}

For each fault, which is an internal interface, we add a boundary condition to the elasticity equation prescribing the jump in the displacement field across the fault,
\begin{gather}
  \label{eqn:bc:prescribed_slip}
  \vec{u}^+ - \vec{u}^- - \vec{d}(\vec{x},t) = \vec{0} \text{ on }\Gamma_f,
\end{gather}
where $\vec{u}^+$ is the displacement vector on the ``positive'' side of the fault, $\vec{u}^-$ is the displacement vector on the ``negative'' side of the fault, $\vec{d}$ is the slip vector on the fault, and $\vec{n}$ is the fault normal which points from the negative side of the fault to the positive side of the fault.
We enforce the jump in displacements across the fault using a Lagrange multiplier corresponding to equal and opposite tractions on the two sides of the fault.

We apply conservation of momemtum,
\begin{equation}
  \int_\Omega \rho(\vec{x}) \frac{\partial \vec{v}}{\partial t} \, d\Omega = \int_\Omega \vec{f}(\vec{x},t) \, d\Omega + \int_\Gamma \vec{\tau}(\vec{x},t) \, d\Gamma,
\end{equation}
to a fault interface $\Omega_f$ with boundaries $\Gamma_{f^+}$ and $\Gamma_{f^-}$.
For a fault interface, the body force is zero, $\vec{f}(\vec{x},t) = \vec{0}$.
The tractions on the positive and negative fault faces are
\begin{gather}
  \tau^+(\vec{x},t) = \tensor{\sigma}^+ \cdot \vec{n} + \vec{\lambda} \\
  \tau^-(\vec{x},t) = \tensor{\sigma}^- \cdot \vec{n} - \vec{\lambda},
\end{gather}
where $\vec{\lambda}$ is the Lagrange multiplier that corresponds to the fault traction generating the prescribed slip and $\tensor{\sigma}^+$ and $\tensor{\sigma}^-$ are the stresses in the domain at the positive and negative sides of the fault.
Thus, for a fault interface, we have
\begin{equation}
  \int_{\Omega_f} \rho(\vec{x}) \frac{\partial \vec{v}}{\partial t} \, d\Omega = \int_{\Gamma_{f^+}} \tensor{\sigma} \cdot \vec{n} + \vec{\lambda} \, d\Gamma + \int_{\Gamma_{f^-}} \tensor{\sigma} \cdot \vec{n} - \vec{\lambda} \, d\Gamma.
\end{equation}

\begin{table}[htbp]
  \caption{Mathematical notation for elasticity equation with
    infinitesimal strain and prescribed slip on faults.}
  \label{tab:notation:elasticity:prescribed:slip}
  \begin{tabular}{lcp{3in}}
    \toprule
    {\bf Category} & {\bf Symbol} & {\bf Description} \\
    \midrule
    Unknowns & $\vec{u}$ & Displacement field \\
    & $\vec{v}$ & Velocity field \\
    & $\vec{\lambda}$ & Lagrange multiplier field \\
    Derived quantities & $\tensor{\sigma}$ & Cauchy stress tensor \\
                   & $\tensor{\epsilon}$ & Cauchy strain tensor \\
    Common constitutive parameters & $\rho$ & Density \\
  & $\mu$ & Shear modulus \\
  & $K$ & Bulk modulus \\
Source terms & $\vec{f}$ & Body force per unit volume, for example $\rho \vec{g}$ \\
    & $\vec{d}$ & Slip vector field on the fault corresponding to a
      jump in the displacement field across the fault \\
    \bottomrule
  \end{tabular}
\end{table}

\subsection{Quasistatic}

As in the case of elasticity without faults, we first consider the quasistatic case in which we neglect the inertial term ($\rho \frac{\partial \vec{v}}{\partial t} \approx \vec{0}$).
We place all of the terms in the elasticity equation on the LHS, consistent with implicit time stepping.
Our equation of the conservation of momentum on the fault interface reduces to
\begin{equation}
  \int_{\Gamma_{f^+}} \tensor{\sigma} \cdot \vec{n} + \vec{\lambda} \, d\Gamma + \int_{\Gamma_{f^-}} \tensor{\sigma} \cdot \vec{n} - \vec{\lambda} \, d\Gamma = 0.
\end{equation}
We enforce this equation on each portion of the fault interface along with our prescribed slip constraint, which leads to
\begin{gather}
  \tensor{\sigma} \cdot \vec{n} + \vec{\lambda} = \vec{0} \text{ on } \Gamma_{f^+}, \\
  \tensor{\sigma} \cdot \vec{n} - \vec{\lambda} = \vec{0}\text{ on } \Gamma_{f^-}, \\
  \vec{u}^+ - \vec{u}^- - \vec{d}(\vec{x},t) = \vec{0},  
\end{gather}

Our solution vector consists of both displacements and Lagrange multipliers, and the strong form for the system of equations is
\begin{gather}
  % Solution
  \vec{s}^T = \left( \vec{u} \quad \vec{\lambda} \right)^T \\
  % Elasticity
  \vec{f}(\vec{x},t) + \tensor{\nabla} \cdot \tensor{\sigma}(\vec{u}) = \vec{0} \text{ in }\Omega, \\
  % Neumann
  \tensor{\sigma} \cdot \vec{n} = \vec{\tau}(\vec{x},t) \text{ on }\Gamma_\tau, \\
  % Dirichlet
  \vec{u} = \vec{u}_0(\vec{x},t) \text{ on }\Gamma_u, \\
  % Prescribed slip
  \vec{u}^+ - \vec{u}^- - \vec{d}(\vec{x},t) = \vec{0} \text{ on }\Gamma_f,  \\
  \tensor{\sigma} \cdot \vec{n} = -\vec{\lambda}(\vec{x},t) \text{ on }\Gamma_{f^+}, \text{ and}\\
  \tensor{\sigma} \cdot \vec{n} = +\vec{\lambda}(\vec{x},t) \text{ on }\Gamma_{f^-}.
\end{gather}
We create the weak form by taking the dot product with the trial function $\trialvec[u]$ or $\trialvec[\lambda]$ and integrating over the domain.
After using the divergence theorem and incorporating the Neumann boundary and fault interface conditions, we have
\begin{gather}
  % Elasticity
  \int_\Omega \trialvec[u] \cdot \vec{f}(\vec{x},t) + \nabla \trialvec[v] : -\tensor{\sigma}(\vec{u}) \, d\Omega
  + \int_{\Gamma_\tau} \trialvec[u] \cdot \vec{\tau}(\vec{x},t) \, d\Gamma,
  + \int_{\Gamma_{f}} \trialvec[u^+] \cdot \left(-\vec{\lambda}(\vec{x},t)\right)
  + \trialvec[u^-] \cdot \left(+\vec{\lambda}(\vec{x},t)\right)\, d\Gamma = 0\\
  % Prescribed slip
  \int_{\Gamma_{f}} \trialvec[\lambda] \cdot \left(
    -\vec{u}^+ + \vec{u}^- + \vec{d}(\vec{x},t) \right) \, d\Gamma = 0.
\end{gather}
We have chosen the signs in the last equation so that the Jacobian will be symmetric with respect to the Lagrange multiplier.
We solve the system of equations using implicit time stepping, making use of residuals functions and Jacobians for the LHS.


\subsubsection{Residual Pointwise Functions}

Identifying $F(t,s,\dot{s})$ and $G(t,s)$, we have
\begin{align}
 % Fu
F^u(t,s,\dot{s}) &= \int_\Omega \trialvec[u] \cdot \eqnannotate{\vec{f}(\vec{x},t)}{\vec{f}^u_0} + \nabla \trialvec[u] : \eqnannotate{-\tensor{\sigma}(\vec{u})}{\tensor{f^u_1}} \, d\Omega
  + \int_{\Gamma_\tau} \trialvec[u] \cdot \eqnannotate{\vec{\tau}(\vec{x},t)}{\vec{f}^u_0} \, d\Gamma 
  + \int_{\Gamma_{f}} \trialvec[u^+] \cdot \eqnannotate{\left(-\vec{\lambda}(\vec{x},t)\right)}{\vec{f}^u_0}
  + \trialvec[u^-] \cdot \eqnannotate{\left(+\vec{\lambda}(\vec{x},t)\right)}{\vec{f}^u_0}\, d\Gamma \\
  % Fl
  F^\lambda(t,s,\dot{s}) &= \int_{\Gamma_{f}} \trialvec[\lambda] \cdot \eqnannotate{\left(
    -\vec{u}^+ + \vec{u}^- + \vec{d}(\vec{x},t) \right)}{\vec{f}^\lambda_0} \, d\Gamma, \\
  % Gu
  G^u(t,s) &= 0 \\
  % Gl
  G^\lambda(t,s) &= 0
\end{align}
Compared to the quasistatic elasticity case without a fault, we have simply added additional pointwise functions associated with the fault.
Our fault implementation does not change the formulation for the materials or external Dirichlet or Neumann boundary conditions.

\subsubsection{Jacobian Pointwise Functions}

The LHS Jacobians are:
\begin{align}
  % J_F uu
  J_F^{uu} &= \frac{\partial F^u}{\partial u} + s_\mathit{tshift} \frac{\partial F^u}{\partial \dot{u}}
      = \int_\Omega \nabla \trialvec[u] : -\tensor{C} : \frac{1}{2}(\nabla + \nabla^T)\basisvec[u] 
\, d\Omega 
      = \int_\Omega \trialscalar[u]_{i,k} \, \eqnannotate{\left( -C_{ikjl} \right)}{J_{f3}^{uu}} \, \basisscalar[u]_{j,l}\, d\Omega \\
  % J_F ul
  J_F^{u\lambda} &= \frac{\partial F^u}{\partial \lambda} + s_\mathit{tshift} \frac{\partial F^u}{\partial \dot{\lambda}}
      = \int_{\Gamma_{f}} \trialscalar[u^+]_i \eqnannotate{\left(-\delta_{ij}\right)}{J^{u\lambda}_{f0}} \basisscalar[\lambda]_j
                   + \trialscalar[u^-]_i \eqnannotate{\left(+\delta_{ij}\right)}{J^{u\lambda}_{f0}} \basisscalar[\lambda]_j\, d\Gamma, \\
  % J_F lu
  J_F^{\lambda u} &= \frac{\partial F^\lambda}{\partial u} + s_\mathit{tshift} \frac{\partial F^\lambda}{\partial \dot{u}}
      = \int_{\Gamma_{f}} \trialscalar[\lambda]_i 
                    \eqnannotate{\left(-\delta_{ij}\right)}{J^{\lambda u}_{f0}} \basisscalar[u^+]_j
                    + \trialscalar[\lambda]_i \eqnannotate{\left(+\delta_{ij}\right)}{J^{\lambda u}_{f0}} \basisscalar[u^-]_j \, d\Gamma, \\
  % J_F ll
  J_F^{\lambda \lambda} &= \tensor{0}
\end{align}
This LHS Jacobian has the structure
\begin{equation}
  J_F = \left( \begin{array} {cc} J_F^{uu} & J_F^{u\lambda} \\ J_F^{\lambda u} & 0 \end{array} \right)
      = \left( \begin{array} {cc} J_F^{uu} & C^T \\ C & 0 \end{array} \right),
\end{equation}
where $C$ contains entries of $\pm 1$ for degrees of freedom on the two sides of the fault. The Schur complement of $J$ with respect to $J_F^{uu}$ is $-C\left(J_F^{uu}\right)^{-1}C^T$.


\subsection{Dynamic}

The equation prescribing fault slip is independent of the Lagrange multiplier, so we do not have a system of equations that we can put in
the form $\dot{s} = G^*(t,s)$.
Instead, we have a differential-algebraic set of equations (DAEs), which we solve using an implicit-explicit (IMEX) time integration scheme.
As in the case of dynamic elasticity without faults, we introduce the velocity ($\vec{v}$) as an unknown to turn the elasticity equation into two first order equations.
Our constraint for prescribed slip is
\begin{equation}
  \vec{u}^+ - \vec{u}^- - \vec{d} = \vec{0},
\end{equation}
where $\vec{u}$ is the displacement vector and $\vec{d}$ is the slip vector.
In order to match the order of the time derivative in the elasticity equation, we take the second derivative of the prescribed slip equation with respect to time,
\begin{equation}
  \frac{\partial \vec{v}^+}{\partial t} - \frac{\partial \vec{v}^-}{\partial t} - \frac{\partial^2 \vec{d}}{\partial t^2} = \vec{0}.
\end{equation}
This means that our differential algebraic equations has a differentiation index of 2.

The strong form for our system of equations is:
\begin{gather}
  % Solution
  \vec{s}^T = \left( \vec{u} \quad \vec{v} \quad \vec{\lambda} \right)^T \\
  % Displacement-velocity
  \frac{\partial \vec{u}}{\partial t} = \vec{v}, \\
  % Elasticity
  \rho(\vec{x}) \frac{\partial \vec{v}}{\partial t} = \vec{f}(\vec{x},t) + \nabla \cdot \tensor{\sigma}(\vec{u}), \\
  % Neumann BC
  \tensor{\sigma} \cdot \vec{n} = \vec{\tau} \text{ on } \Gamma_\tau. \\
  % Dirichlet BC
  \vec{u} = \vec{u}_0 \text{ on } \Gamma_u, \\
  % Presribed slip
  \frac{\partial \vec{v}^+}{\partial t} - \frac{\partial \vec{v}^-}{\partial t} - \frac{\partial^2 \vec{d}(\vec{x},t)}{\partial t^2} = \vec{0}, \\
  \int_{\Omega_f} \rho(\vec{x}) \frac{\partial \vec{v}}{\partial t} \, d\Omega = \int_{\Gamma_{f^+}} \tensor{\sigma} \cdot \vec{n} + \vec{\lambda} \, d\Gamma + \int_{\Gamma_{f^-}} \tensor{\sigma} \cdot \vec{n} - \vec{\lambda} \, d\Gamma.
\end{gather}
We generate the weak form in the usual way,
\begin{gather}
  % Displacement-velocity
  \int_{\Omega} \trialvec[u] \cdot \frac{\partial \vec{u}}{\partial t} \, d\Omega =  \int_{\Omega} \trialvec[u] \cdot \vec{v} \, d\Omega, \\
  % Elasticity
  \begin{multlined}
  \int_{\Omega} \trialvec[v] \cdot \rho(\vec{x}) \frac{\partial \vec{v}}{\partial t} \, d\Omega  = \int_\Omega \trialvec[v] \cdot \vec{f}(\vec{x},t) + \nabla \trialvec[v] : -\tensor{\sigma}(\vec{u}) \, d\Omega + \int_{\Gamma_\tau} \trialvec[v] \cdot \vec{\tau}(\vec{x},t) \, d\Gamma \\
  + \int_{\Gamma_{f}} \trialvec[v^+] \cdot \left(-\vec{\lambda}(\vec{x},t)\right) + \trialvec[v^-] \cdot \left(+\vec{\lambda}(\vec{x},t)\right)\, d\Gamma,
  \end{multlined}\\
  % Prescribed slip
  \int_{\Gamma_f} \trialvec[\lambda] \cdot \left(\frac{\partial \vec{v}^+}{\partial t} - \frac{\partial \vec{v}^-}{\partial t} - \frac{\partial^2 \vec{d}(\vec{x},t)}{\partial t^2} \right) \, d\Gamma = 0. \\
  \int_{\Omega_f} \trialvec[\lambda] \cdot \rho(\vec{x}) \frac{\partial \vec{v}}{\partial t} \, d\Omega = \int_{\Gamma_{f^+}} \trialvec[\lambda] \cdot \left( \tensor{\sigma} \cdot \vec{n} + \vec{\lambda} \right) \, d\Gamma + \int_{\Gamma_{f^-}} \trialvec[\lambda] \cdot \left( \tensor{\sigma} \cdot \vec{n} - \vec{\lambda} \right) \, d\Gamma.
\end{gather}

For compatibility with PETSc TS IMEX implementations, we need $\dot{\vec{s}}$ on the LHS for the explicit part (displacement-velocity and elasticity equations) and we need $\vec{\lambda}$ in the equation for the implicit part (prescribed slip equation).
We first focus on the explicit part and select numerical quadrature that yields a lumped mass matrix, $M$, so that we have
\begin{gather}
  % Displacement-velocity
  \label{eqn:displacement:velocity:prescribed:slip:weak:form}
  \frac{\partial \vec{u}}{\partial t} = M_u^{-1} \int_{\Omega} \trialvec[u] \cdot \vec{v} \, d\Omega, \\
  % Elasticity
  \label{eqn:elasticity:prescribed:slip:dynamic:weak:form}
  \begin{multlined}
  \frac{\partial \vec{v}}{\partial t} = M_v^{-1} \int_\Omega \trialvec[v] \cdot \vec{f}(\vec{x},t) + \nabla \trialvec[v] : -\tensor{\sigma}(\vec{u}) \, d\Omega + M_v^{-1} \int_{\Gamma_\tau} \trialvec[v] \cdot \vec{\tau}(\vec{x},t) \, d\Gamma \\
  + M_{v^+}^{-1} \int_{\Gamma_{f}} \trialvec[v^+] \cdot \left(-\vec{\lambda}(\vec{x},t)\right) \, d\Gamma + M_{v^-}^{-1} \int_{\Gamma_{f}}\trialvec[v^-] \cdot \left(+\vec{\lambda}(\vec{x},t)\right) \, d\Gamma,
  \end{multlined}\\
  M_u = \mathit{Lump}\left( \int_\Omega \trialscalar[u]_i \delta_{ij} \basisscalar[u]_j \, d\Omega \right), \\
  M_v = \mathit{Lump}\left( \int_\Omega \trialscalar[v]_i \rho(\vec{x}) \delta_{ij} \basisscalar[v]_j \, d\Omega \right).
\end{gather}
For the implicit part, we can separate the integration of the weak form for negative and positive sides of the fault interface, which yields
\begin{gather}
  M_{v^+} \frac{\partial \vec{v}^+}{\partial t} = \int_{\Gamma_{f^+}} \trialvec[\lambda] \cdot \left( \tensor{\sigma} \cdot \vec{n} + \vec{\lambda} \right) \, d\Gamma, \\
  M_{v^-} \frac{\partial \vec{v}^-}{\partial t} = \int_{\Gamma_{f^-}} \trialvec[\lambda] \cdot \left( \tensor{\sigma} \cdot \vec{n} - \vec{\lambda} \right) \, d\Gamma.
\end{gather}
Using these equations to substitute in the expressions for the time derivative of the velocity on the negative and positive sides of the fault into the prescribed slip constraint equation yields
\begin{equation}
  \label{eqn:elasticity:prescribed:slip:dynamic:DAE:weak:form}
  M_{v^+}^{-1} \int_{\Gamma_f^+} \trialvec[\lambda] \cdot \left(\tensor{\sigma} \cdot \vec{n} + \vec{\lambda}\right) \, d\Gamma + M_{v^-}^{-1} \int_{\Gamma_f^-} \trialvec[\lambda] \cdot \left( -\tensor{\sigma} \cdot \vec{n} + \vec{\lambda} \right) \, d\Gamma - \int_{\Gamma_f} \trialvec[\lambda] \cdot \frac{\partial^2 \vec{d}}{\partial t^2} \, d\Gamma = \vec{0}.
\end{equation}


\subsubsection{Residual Pointwise Functions}

Combining the explicit parts of the weak form in equations~\ref{eqn:displacement:velocity:prescribed:slip:weak:form} and \ref{eqn:elasticity:prescribed:slip:dynamic:weak:form} with the implicit part of the weak form in equation~\ref{eqn:elasticity:prescribed:slip:dynamic:DAE:weak:form} and identifying $F(t,s,\dot{s})$ and $G(t,s)$, we have
\begin{gather}
  % Fu
  F^u(t,s,\dot{s}) = \frac{\partial \vec{u}}{\partial t} \\
  % Fv
  F^v(t,s,\dot{s}) = \frac{\partial \vec{v}}{\partial t} \\
  % Fl
    F^\lambda(t,s,\dot{s}) = \eqnannotate{M_{v^+}^{-1}}{c^+} \int_{\Gamma_f^+} \trialvec[\lambda] \cdot \eqnannotate{\left(\tensor{\sigma} \cdot \vec{n} + \vec{\lambda}\right)}{f^\lambda_0} \, d\Gamma + \eqnannotate{M_{v^-}^{-1}}{c^-} \int_{\Gamma_f^-} \trialvec[\lambda] \cdot \eqnannotate{\left( -\tensor{\sigma} \cdot \vec{n} + \vec{\lambda} \right)}{f^\lambda_0} \, d\Gamma - \int_{\Gamma_f} \trialvec[\lambda] \cdot \eqnannotate{\frac{\partial^2 \vec{d}}{\partial t^2}}{f^\lambda_0} \, d\Gamma = \vec{0}, \\
  % Gu
  G^u(t,s) = \eqnannotate{M_{u}^{-1}}{c} \int_\Omega \trialvec[u] \cdot \eqnannotate{\vec{v}}{\vec{g}^u_0} \, d\Omega, \\
  % Gv
  G^v(t,s) =  \eqnannotate{M_{v}^{-1}}{c} \left( \int_\Omega \trialvec[v] \cdot \eqnannotate{\vec{f}(\vec{x},t)}{\vec{g}^v_0} + \nabla \trialvec[v] : \eqnannotate{-\tensor{\sigma}(\vec{u})}{\tensor{g^v_1}} \, d\Omega
  + \int_{\Gamma_\tau} \trialvec[v] \cdot \eqnannotate{\vec{\tau}(\vec{x},t)}{\vec{g}^v_0} \, d\Gamma,
  + \int_{\Gamma_{f}} \trialvec[v^+] \cdot \eqnannotate{\left(-\vec{\lambda}(\vec{x},t)\right)}{\vec{g}^v_0}
             + \trialvec[v^-] \cdot \eqnannotate{\left(+\vec{\lambda}(\vec{x},t)\right)}{\vec{g}^v_0} \, d\Gamma \right), \\
  % Gl
  G^\lambda(t,s) = 0
\end{gather}
The integrals for the explicit part are all weighted by the inverse of the lumped mass matrix.
For the implicit part, only the integrals over the positive and negative sides of the fault are weighted by the inverse of the lumped mass matrix.

\subsubsection{Jacobian Pointwise Functions}

For the explicit part we have pointwise functions for computing the lumped LHS Jacobian. These are exactly the same pointwise functions as in the dynamic case without a fault,
\begin{align}
  % J_F uu
  J_F^{uu} &= \frac{\partial F^u}{\partial u} + s_\mathit{tshift} \frac{\partial F^u}{\partial \dot{u}} =
             \int_\Omega \trialscalar[u]_i \eqnannotate{s_\mathit{tshift} \delta_{ij}}{J^{uu}_{f0}} \basisscalar[u]_j  \, d\Omega, \\
  % J_F vv
  J_F^{vv} &= \frac{\partial F^v}{\partial v} + s_\mathit{tshift} \frac{\partial F^v}{\partial \dot{v}} =
             \int_\Omega \trialscalar[v]_i \eqnannotate{\rho(\vec{x}) s_\mathit{tshift} \delta_{ij}}{J ^{vv}_{f0}} \basisscalar[v]_j \, d\Omega
\end{align}
For the implicit part, we have pointwise functions for the LHS Jacobians associated with the prescribed slip,
\begin{gather}
  \begin{multlined}
  % J_F lu
  J_F^{\lambda u} = \frac{\partial F^\lambda}{\partial u} + s_\mathit{tshift} \frac{\partial F^\lambda}{\partial \dot{u}} = 
  \eqnannotate{M_{v^+}^{-1}}{c^+} \int_{\Gamma_{f^+}} \trialscalar[\lambda]_i \eqnannotate{ C_{kijl} n_k}{J^{\lambda u}_{f1}} \basisscalar[u]_{j,l} \, d\Gamma + \eqnannotate{M_{v^-}^{-1}}{c^-} \int_{\Gamma_{f^-}} \trialscalar[\lambda]_i \eqnannotate{- C_{kijl} n_k}{J^{\lambda u}_{f1}} \basisscalar[u]_{j,l} \, d\Gamma
                  \end{multlined} \\
  % J_F ll
  J_F^{\lambda \lambda} = \frac{\partial F^\lambda}{\partial \lambda} + s_\mathit{tshift} \frac{\partial F^\lambda}{\partial \dot{\lambda}} =
  \eqnannotate{M_{v^+}^{-1}}{c^+} \int_{\Gamma_{f^+}} \trialscalar[\lambda]_i \eqnannotate{ \delta_{ij}}{J^{\lambda\lambda}_{f0}} \basisscalar[\lambda]_j \, d\Gamma
            + \eqnannotate{M_{v^-}^{-1}}{c^-} \int_{\Gamma_{f^-}} \trialscalar[\lambda]_i \eqnannotate{ \delta_{ij}}{J^{\lambda\lambda}_{f0}} \basisscalar[\lambda]_j \, d\Gamma
\end{gather}



% End of file
% ----------------------------------------------------------------------
\section{Incompressible Isotropic Elasticity with Infinitesimal Strain (Bathe)}

Building from the elasticity equation
(equation~\ref{eqn:elasticity:order1:strong:form}), we consider an
incompressible material. As the bulk modulus ($K$) approaches
infinity, the volumetric strain ($\Tr(\epsilon)$) approaches zero and
the pressure remains finite, $p = -K \Tr(\epsilon)$. We consider
pressure $p$ as an independent variable and decompose the stress into the
pressure and deviatoric components. As a result, we write the stress tensor in terms of both the displacement and pressure fields,
\begin{equation}
  \tensor{\sigma}(\vec{u},p) = \tensor{\sigma}^\mathit{dev}(\vec{u}) - p\tensor{I}.
\end{equation}

We only consider the case of an incompressible material while
neglecting inertia. The time dependence arises from
history-dependent constitutive equations and boundary conditions. We
have
\begin{gather}
  % Solution
  \vec{s}^T = \left( \vec{u} \quad \ p \right)^T, \\
  % Elasticity
  \vec{0} = \vec{f}(t) + \tensor{\nabla} \cdot \left(\tensor{\sigma}^\mathit{dev}(\vec{u}) - p\tensor{I}\right) \text{ in }\Omega, \\
  % Pressure
  0 = \vec{\nabla} \cdot \vec{u} + \frac{p}{K}, \\
  % Neumann
  \tensor{\sigma} \cdot \vec{n} = \vec{\tau} \text{ on }\Gamma_\tau, \\
  % Dirichlet
  \vec{u} = \vec{u}_0 \text{ on }\Gamma_u, \\
  p = p_0 \text{ on }\Gamma_p.
\end{gather}

\begin{table}[htbp]
  \caption{Mathematical notation for incompressible elasticity with
    infinitesimal strain.}
  \label{tab:notation:incompressible:elasticity}
  \begin{tabular}{lcp{3.5in}}
    \toprule
    {\bf Category} & {\bf Symbol} & {\bf Description} \\
    \midrule
    Unknowns & $\vec{u}$ & Displacement field \\
    & $p$ & Pressure field ($p>0$ corresponds to negative mean stress \\
    Derived quantities & $\tensor{\sigma}$ & Cauchy stress tensor \\
                   & $\tensor{\epsilon}$ & Cauchy strain tensor \\
    Common constitutive parameters & $\rho$ & Density \\
  & $\mu$ & Shear modulus \\
  & $K$ & Bulk modulus \\
Source terms & $\vec{f}$ & Body force per unit volume, for example $\rho \vec{g}$ \\
    \bottomrule
  \end{tabular}
\end{table}

Using trial functions $\trialvec[u]$ and $\trialscalar[p]$ and
incorporating the Neumann boundary conditions, we write the weak form
as
\begin{gather}
  % Displacement
  0 = 
  \int_\Omega \trialvec[u] \cdot \vec{f}(t) + \nabla \trialvec[u] : \left(-\tensor{\sigma}^\mathit{dev}(\vec{u}) + p\tensor{I}
  \right)\, d\Omega + \int_{\Gamma_\tau} \trialvec[u] \cdot \vec{\tau}(t) \, d\Gamma, \\
  % Pressure
  0 = \int_\Omega \trialscalar[p] \cdot \left(\vec{\nabla} \cdot \vec{u} + \frac{p}{K} \right) 
\, d\Omega.
\end{gather}

\subsection{Residual Pointwise Kernels}

Identifying $G(t,s)$, we have
\begin{gather}
  \label{eqn:incompressible:elasticity:displacement}
  0 = \int_\Omega \trialvec[u] \cdot \eqnannotate{\vec{f}(t)}{g_0^u} + \nabla \trialvec[u] :
  \eqnannotate{\left(-\tensor{\sigma}^\mathit{dev}(\vec{u}) + p\tensor{I}\right)}{g_1^u}  \, d\Omega
  + \int_{\Gamma_\tau} \trialvec[u] \cdot \eqnannotate{\vec{\tau}(t)}{g_0^u} \, d\Gamma, \\
%
  \label{eqn:incompressible:elasticity:pressure}
  0 = \int_\Omega \trialscalar[p] \cdot \eqnannotate{\left(\vec{\nabla} \cdot \vec{u} + 
\frac{p}{K} \right)}{g_0^p} \, d\Omega.
\end{gather}

\subsection{Jacobians Pointwisde Kernels}

With two fields we have four Jacobian pointwise kernels for the RHS:
\begin{align}
  J_G^{uu} &= \frac{\partial G^u}{\partial u} = \int_\Omega \nabla \trialvec[u] : 
\frac{\partial}{\partial u}(-
\tensor{\sigma}^\mathit{dev}) \, d\Omega 
  = \int_\Omega \trialscalar[u]_{i,k} \, \eqnannotate{\left(-C^\mathit{dev}_{ikjl}\right)}
{J_{g3}^{uu}}  \, 
\basisscalar[u]_{j,l}\, d\Omega \\
  J_G^{up} &= \frac{\partial G^u}{\partial p} = \int_\Omega \nabla\trialvec[u] : \tensor{I} 
\basisscalar[p] \,  d\Omega = \int_\Omega \trialscalar[u]_{i,k} \eqnannotate{\delta_{ik}}{J_{g2}^{up}} \, 
\basisscalar[p] \, d\Omega \\
%
  J_G^{pu} &= \frac{\partial G^p}{\partial u} = \int_\Omega \trialscalar[p] \left(\vec{\nabla} 
\cdot \basisvec[u]\right) \, d\Omega = \int_\Omega \trialscalar[p] \eqnannotate{\delta_{jl}}{J_{g1}^{pu}} 
\basisscalar[u]_{j,l} \, d\Omega\\
  J_G^{pp} &= \frac{\partial G^p}{\partial p} = \int_\Omega \trialscalar[p] \eqnannotate{\frac{1}
{K}}{J_{g0}^{pp}} \basisscalar[p] \, d\Omega
\end{align}

For isotropic, linear incompressible elasticity, the deviatoric elastic constants are:
\begin{align}
    C_{1111} &= C_{2222} = C_{3333} = +\frac{4}{3} \mu \\
    C_{1122} &= C_{1133} = C_{2233} = -\frac{2}{3} \mu \\
    C_{1212} &= C_{1313} = C_{2323} = \mu
\end{align}

%\chapter{Multiphysics Finite-Element Formulation}
\label{cha:multiphysics:formulation}

This chapter will become part of the governing equations chapter in
the PyLith Manual.

\section{General Finite-Element Formulation}

% ----------------------------------------------------------------------
% ----------------------------------------------------------------------
\section{Poroelasticity with Infinitesimal Strain and No Faults or Inertia}

\todo{poroelasticity group}{Update this with revised formulation from
  the poroelasticity group.}

Formulation based on Zheng et al. and Detournay and Cheng (1993).

In this poroelasticity formulation we assume a compressible fluid
completely saturates a porous solid undergoing infinitesimal
strain. We neglect the inertial effects and do not consider faults.

We begin with the elasticity equilibrium equation neglecting the inertial term,
\begin{gather}
  - \vec{f}(\vec{x},t) - \tensor{\nabla} \cdot \tensor{\sigma}(\vec{u},p_f) = \vec{0} 
\text{ in }\Omega, \\
%
  \tensor{\sigma} \cdot \vec{n} = \vec{\tau}(\vec{x},t) \text{ on }\Gamma_\tau, \\
%
  \vec{u} = \vec{u}_0(\vec{x},t) \text{ on }\Gamma_u,
\end{gather}
where $\vec{u}$ is the displacement vector, $\vec{f}$ is the body
force vector, $\tensor{\sigma}$ is the Cauchy stress tensor, and $t$
is time. We specify tractions $\vec{\tau}$ on boundary $\Gamma_\tau$, and
displacements $\vec{u}_0$ on boundary $\Gamma_u$. If gravity is included in
the problem, then usually $\vec{f} = \rho \vec{g}$, where $\rho$ is
the average density $\rho = (1-\phi)\rho_s + \phi \rho_f$, $\phi$ is
the porosity of the solid, $\rho_s$ is the density of the solid, and
$\rho_f$ is the density of the fluid.

Enforcing mass balance of the fluid gives
\begin{gather}
  \frac{\partial \zeta(\vec{u},p_f)}{\partial t} + \nabla \cdot \vec{q}(p_f) = 
\gamma(\vec{x},t) \text{ in }
\Omega, \\
%
  \vec{q} \cdot \vec{n} = q_0(\vec{x},t) \text{ on }\Gamma_q, \\
%
  p_f = p_0(\vec{x},t) \text{ on }\Gamma_p,
\end{gather}
where $\zeta$ is the variation in fluid content, $\vec{q}$ is the rate
of fluid volume crossing a unit area of the porous solid, $\gamma$ is
the rate of injected fluid per unit volume of the porous solid, $q_0$
is the outward fluid velocity normal to the boundary $\Gamma_q$, and
$p_0$ is the fluid pressure on boundary $\Gamma_p$.

We require the fluid flow to follow Darcy's law (Navier-Stokes equation neglecting inertial 
effects),
\begin{gather}
  \vec{q}(p_f) = -\kappa (\nabla p_f - \vec{f}_f), \\
%
  \kappa = \frac{k}{\eta_f}
\end{gather}
where $\kappa$ is the permeability coefficient (Darcy conductivity),
$k$ is the intrinsic permeability, $\eta_f$ is the viscosity of the
fluid, $p_f$ is the fluid pressure, and $\vec{f}_f$ is the body force
in the fluid. If gravity is included in a problem, then usually
$\vec{f}_f = \rho_f \vec{g}$, where $\rho_f$ is the density of the
fluid and $\vec{g}$ is the gravitational acceleration vector.

We assume linear elasticity for the solid phase, so the constitutive behavior can be expressed 
as
\begin{gather}
  \tensor{\sigma}(\vec{u},p_f) = \tensor{C} : \tensor{\epsilon} - \alpha p_f \tensor{I},
\end{gather}
where $\tensor{\sigma}$ is the stress tensor, $\tensor{C}$ is the
tensor of elasticity constants, $\alpha$ is the Biot coefficient
(effective stress coefficient), $\tensor{\epsilon}$ is the strain
tensor, and $\tensor{I}$ is the identity tensor.

For the constitutive behavior of the fluid, we use the volumetric strain to couple the fluid-
solid behavior,
\begin{gather}
  \zeta(\vec{u},p_f) = \alpha \Tr({\tensor{\epsilon}}) + \frac{p_f}{M}, \\
%
  \frac{1}{M} = \frac{\alpha-\phi}{K_s} + \frac{\phi}{K_f},
\end{gather}
where $1/M$ is the specific storage coefficient at constant strain,
$K_s$ is the bulk modulus of the solid, and $K_f$ is the bulk modulus
of the fluid. We can write the trace of the strain tensor as the dot product of the gradient 
and displacement 
field, so we have
\begin{gather}
  \zeta(\vec{u},p_f) = \alpha (\nabla \cdot \vec{u}) + \frac{p_f}{M}.
\end{gather}

\subsection{Notation}
\begin{itemize}
\item Unknowns
  \begin{description}
  \item[$\vec{u}$] Displacement field
  \item[$p_f$] Fluid pressure
  \end{description}
\item Derived quantities
  \begin{description}
    \item[$\tensor{\sigma}$] Stress tensor
    \item[$\tensor{\epsilon}$] Strain tensor
    \item[$\zeta$] Variation of fluid content; variation of fluid volumer per unit volume of 
porous material
    \item[$q$] rate of fluid volume crossing a unit area of the porous
      solid; fluid flux
    \item[$1/M$] Specific storage coefficient at constant strain
    \item[$\kappa$] permability coefficient; Darcy conductivity; $\kappa = k/\eta_f$
    \item[$\rho$] Average density; $\rho = (1-\phi)\rho_s + \phi \rho_f$
  \end{description}
\item Constitutive parameters
  \begin{description}
  \item[$\mu$] Shear modulus of solid
  \item[$K_s$] Bulk modulus of solid
  \item[$K_f$] Bulk modulus of fluid
  \item[$\alpha$] Biot coefficient; effective stress coefficient
  \item[$k$] Intrinsic permeability
  \item[$\eta_f$] Fluid viscosity
  \item[$\rho_s$] Density of solid
  \item[$\rho_f$] Density of fluid
  \item[$\phi$] Porosity; $\frac{V_p}{V}$ ($V_p$ is the volume of the pore space)
  \end{description}
\item Source terms
  \begin{description}
    \item[$\vec{f}$] Body force, for example $\rho \vec{g} = (1-\phi)\rho_s + \phi \rho_f$
    \item[$\vec{f}_f$] Body force in fluid, for example $\rho_f \vec{g}$
    \item[$\gamma$] Source density; rate of injected fluid per unit volume of the porous solid
  \end{description}
\end{itemize}

We consider the displacement $\vec{u}$ and fluid pressure $p_f$ as unknowns,
\begin{align}
  \vec{s}^T &= (\vec{u} \quad p_f)^T, \\
%
% elasticity equilibrium equation
  \vec{0} &= \vec{f}(\vec{x},t) + \tensor{\nabla} \cdot \tensor{\sigma}(\vec{u},p_f) 
\text{ in }\Omega, \\
%
% fluid mass balance
  \frac{\partial \zeta(\vec{u},p_f)}{\partial t} &= \gamma(\vec{x},t) - \nabla \cdot \vec{q}
(p_f) \text{ in }
\Omega, \\
%
% Darcy's law
  \vec{q}(p_f) &= -\kappa (\nabla p_f - \vec{f}_f), \\
%
  \tensor{\sigma} \cdot \vec{n} &= \vec{\tau}(\vec{x},t) \text{ on }\Gamma_\tau, \\
%
  \zeta(\vec{u},p_f) &= \alpha (\nabla \cdot \vec{u}) + \frac{p_f}{M}, \\
%
  \vec{u} &= \vec{u}_0(\vec{x},t) \text{ on }\Gamma_u, \\
%
  \vec{q} \cdot \vec{n} &= q_0(\vec{x},t) \text{ on }\Gamma_q, \\
%
  p_f &= p_0(\vec{x},t) \text{ on }\Gamma_p.
\end{align}
For trial functions $\trialvec[u]$ and $\trialscalar[p]$ we write the weak form
using the elasticity equilibrium and the fluid mass balance equations,
\begin{align}
  0 &= \int_\Omega \trialvec[u] \cdot \left( \vec{f}(\vec{x},t) + \tensor{\nabla} \cdot 
\tensor{\sigma}
(\vec{u},p_f) \right) \, d\Omega, \\
%
 \int_\Omega  \trialscalar[p] \frac{\partial \zeta(\vec{u},p_f)}{\partial t} \, d\Omega &= 
\int_\Omega 
\trialscalar[p] \left(\gamma(\vec{x},t) - \nabla \cdot \vec{q}(p_f)\right) \, d\Omega.
\end{align}
Applying the divergence theorem to each of these two equations and incorporating the Neumann 
boundary conditions 
yields
\begin{align}
  0 &= \int_\Omega \trialvec[u] \cdot \vec{f}(\vec{x},t) + \nabla \trialvec[u] : -
\tensor{\sigma}(\vec{u},p_f) \, 
d\Omega + \int_{\Gamma_\tau} \trialvec[u] \cdot \vec{\tau}(\vec{x},t) \, d\Gamma, \\
%
 \int_\Omega  \trialscalar[p] \frac{\partial \zeta(\vec{u},p_f)}{\partial t} \, d\Omega &= 
 \int_\Omega \trialscalar[p] \gamma(\vec{x},t) + \nabla \trialscalar[p] \cdot \vec{q}(p_f) \, 
d\Omega
 + \int_{\Gamma_q} \trialscalar[p] (-q_0(\vec{x},t)) \, d\Gamma.
\end{align}
Identifying $F(t,s,\dot{s})$ and $G(t,s)$ we have
\begin{alignat}{2}
  F^u(t,s,\dot{s}) &= \vec{0},
  & \quad
  G^u(t,s) &= \int_\Omega \trialvec[u] \cdot \eqnannotate{\vec{f}(\vec{x},t)}{g^u_0} + \nabla 
\trialvec[u] : 
\eqnannotate{-\tensor{\sigma}(\vec{u},p_f)}{g^u_1} \, d\Omega + \int_{\Gamma_\tau} 
\trialvec[u] \cdot 
\eqnannotate{\vec{\tau}(\vec{x},t)}{g^u_0} \, d\Gamma, \\
  F^p(t,s,\dot{s}) &= \int_\Omega  \trialscalar[p] \eqnannotate{\frac{\partial 
\zeta(\vec{u},p_f)}{\partial t}}
{f^p_0} \, d\Omega
  & \quad
  G^p(t,s) &= \int_\Omega \trialscalar[p] \eqnannotate{\gamma(\vec{x},t)}{g^p_0} + \nabla 
\trialscalar[p] \cdot 
\eqnannotate{\vec{q}(p_f)}{g^p_1} \, d\Omega
 + \int_{\Gamma_q} \trialscalar[p] (\eqnannotate{-q_0(\vec{x},t)}{g^p_0}) \, d\Gamma
\end{alignat}
