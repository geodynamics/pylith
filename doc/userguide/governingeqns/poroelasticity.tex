\chapter{Multiphysics Finite-Element Formulation}
\label{cha:multiphysics:formulation}

This chapter will become part of the governing equations chapter in
the PyLith Manual.

\section{General Finite-Element Formulation}

% ----------------------------------------------------------------------
% ----------------------------------------------------------------------
\section{Poroelasticity with Infinitesimal Strain and No Faults or Inertia}

\todo{poroelasticity group}{Update this with revised formulation from
  the poroelasticity group.}

Formulation based on Zheng et al. and Detournay and Cheng (1993).

In this poroelasticity formulation we assume a compressible fluid
completely saturates a porous solid undergoing infinitesimal
strain. We neglect the inertial effects and do not consider faults.

We begin with the elasticity equilibrium equation neglecting the inertial term,
\begin{gather}
  - \vec{f}(\vec{x},t) - \tensor{\nabla} \cdot \tensor{\sigma}(\vec{u},p_f) = \vec{0} 
\text{ in }\Omega, \\
%
  \tensor{\sigma} \cdot \vec{n} = \vec{\tau}(\vec{x},t) \text{ on }\Gamma_\tau, \\
%
  \vec{u} = \vec{u}_0(\vec{x},t) \text{ on }\Gamma_u,
\end{gather}
where $\vec{u}$ is the displacement vector, $\vec{f}$ is the body
force vector, $\tensor{\sigma}$ is the Cauchy stress tensor, and $t$
is time. We specify tractions $\vec{\tau}$ on boundary $\Gamma_\tau$, and
displacements $\vec{u}_0$ on boundary $\Gamma_u$. If gravity is included in
the problem, then usually $\vec{f} = \rho \vec{g}$, where $\rho$ is
the average density $\rho = (1-\phi)\rho_s + \phi \rho_f$, $\phi$ is
the porosity of the solid, $\rho_s$ is the density of the solid, and
$\rho_f$ is the density of the fluid.

Enforcing mass balance of the fluid gives
\begin{gather}
  \frac{\partial \zeta(\vec{u},p_f)}{\partial t} + \nabla \cdot \vec{q}(p_f) = 
\gamma(\vec{x},t) \text{ in }
\Omega, \\
%
  \vec{q} \cdot \vec{n} = q_0(\vec{x},t) \text{ on }\Gamma_q, \\
%
  p_f = p_0(\vec{x},t) \text{ on }\Gamma_p,
\end{gather}
where $\zeta$ is the variation in fluid content, $\vec{q}$ is the rate
of fluid volume crossing a unit area of the porous solid, $\gamma$ is
the rate of injected fluid per unit volume of the porous solid, $q_0$
is the outward fluid velocity normal to the boundary $\Gamma_q$, and
$p_0$ is the fluid pressure on boundary $\Gamma_p$.

We require the fluid flow to follow Darcy's law (Navier-Stokes equation neglecting inertial 
effects),
\begin{gather}
  \vec{q}(p_f) = -\kappa (\nabla p_f - \vec{f}_f), \\
%
  \kappa = \frac{k}{\eta_f}
\end{gather}
where $\kappa$ is the permeability coefficient (Darcy conductivity),
$k$ is the intrinsic permeability, $\eta_f$ is the viscosity of the
fluid, $p_f$ is the fluid pressure, and $\vec{f}_f$ is the body force
in the fluid. If gravity is included in a problem, then usually
$\vec{f}_f = \rho_f \vec{g}$, where $\rho_f$ is the density of the
fluid and $\vec{g}$ is the gravitational acceleration vector.

We assume linear elasticity for the solid phase, so the constitutive behavior can be expressed 
as
\begin{gather}
  \tensor{\sigma}(\vec{u},p_f) = \tensor{C} : \tensor{\epsilon} - \alpha p_f \tensor{I},
\end{gather}
where $\tensor{\sigma}$ is the stress tensor, $\tensor{C}$ is the
tensor of elasticity constants, $\alpha$ is the Biot coefficient
(effective stress coefficient), $\tensor{\epsilon}$ is the strain
tensor, and $\tensor{I}$ is the identity tensor.

For the constitutive behavior of the fluid, we use the volumetric strain to couple the fluid-
solid behavior,
\begin{gather}
  \zeta(\vec{u},p_f) = \alpha \Tr({\tensor{\epsilon}}) + \frac{p_f}{M}, \\
%
  \frac{1}{M} = \frac{\alpha-\phi}{K_s} + \frac{\phi}{K_f},
\end{gather}
where $1/M$ is the specific storage coefficient at constant strain,
$K_s$ is the bulk modulus of the solid, and $K_f$ is the bulk modulus
of the fluid. We can write the trace of the strain tensor as the dot product of the gradient 
and displacement 
field, so we have
\begin{gather}
  \zeta(\vec{u},p_f) = \alpha (\nabla \cdot \vec{u}) + \frac{p_f}{M}.
\end{gather}

\subsection{Notation}
\begin{itemize}
\item Unknowns
  \begin{description}
  \item[$\vec{u}$] Displacement field
  \item[$p_f$] Fluid pressure
  \end{description}
\item Derived quantities
  \begin{description}
    \item[$\tensor{\sigma}$] Stress tensor
    \item[$\tensor{\epsilon}$] Strain tensor
    \item[$\zeta$] Variation of fluid content; variation of fluid volumer per unit volume of 
porous material
    \item[$q$] rate of fluid volume crossing a unit area of the porous
      solid; fluid flux
    \item[$1/M$] Specific storage coefficient at constant strain
    \item[$\kappa$] permability coefficient; Darcy conductivity; $\kappa = k/\eta_f$
    \item[$\rho$] Average density; $\rho = (1-\phi)\rho_s + \phi \rho_f$
  \end{description}
\item Constitutive parameters
  \begin{description}
  \item[$\mu$] Shear modulus of solid
  \item[$K_s$] Bulk modulus of solid
  \item[$K_f$] Bulk modulus of fluid
  \item[$\alpha$] Biot coefficient; effective stress coefficient
  \item[$k$] Intrinsic permeability
  \item[$\eta_f$] Fluid viscosity
  \item[$\rho_s$] Density of solid
  \item[$\rho_f$] Density of fluid
  \item[$\phi$] Porosity; $\frac{V_p}{V}$ ($V_p$ is the volume of the pore space)
  \end{description}
\item Source terms
  \begin{description}
    \item[$\vec{f}$] Body force, for example $\rho \vec{g} = (1-\phi)\rho_s + \phi \rho_f$
    \item[$\vec{f}_f$] Body force in fluid, for example $\rho_f \vec{g}$
    \item[$\gamma$] Source density; rate of injected fluid per unit volume of the porous solid
  \end{description}
\end{itemize}

We consider the displacement $\vec{u}$ and fluid pressure $p_f$ as unknowns,
\begin{align}
  \vec{s}^T &= (\vec{u} \quad p_f)^T, \\
%
% elasticity equilibrium equation
  \vec{0} &= \vec{f}(\vec{x},t) + \tensor{\nabla} \cdot \tensor{\sigma}(\vec{u},p_f) 
\text{ in }\Omega, \\
%
% fluid mass balance
  \frac{\partial \zeta(\vec{u},p_f)}{\partial t} &= \gamma(\vec{x},t) - \nabla \cdot \vec{q}
(p_f) \text{ in }
\Omega, \\
%
% Darcy's law
  \vec{q}(p_f) &= -\kappa (\nabla p_f - \vec{f}_f), \\
%
  \tensor{\sigma} \cdot \vec{n} &= \vec{\tau}(\vec{x},t) \text{ on }\Gamma_\tau, \\
%
  \zeta(\vec{u},p_f) &= \alpha (\nabla \cdot \vec{u}) + \frac{p_f}{M}, \\
%
  \vec{u} &= \vec{u}_0(\vec{x},t) \text{ on }\Gamma_u, \\
%
  \vec{q} \cdot \vec{n} &= q_0(\vec{x},t) \text{ on }\Gamma_q, \\
%
  p_f &= p_0(\vec{x},t) \text{ on }\Gamma_p.
\end{align}
For trial functions $\trialvec[u]$ and $\trialscalar[p]$ we write the weak form
using the elasticity equilibrium and the fluid mass balance equations,
\begin{align}
  0 &= \int_\Omega \trialvec[u] \cdot \left( \vec{f}(\vec{x},t) + \tensor{\nabla} \cdot 
\tensor{\sigma}
(\vec{u},p_f) \right) \, d\Omega, \\
%
 \int_\Omega  \trialscalar[p] \frac{\partial \zeta(\vec{u},p_f)}{\partial t} \, d\Omega &= 
\int_\Omega 
\trialscalar[p] \left(\gamma(\vec{x},t) - \nabla \cdot \vec{q}(p_f)\right) \, d\Omega.
\end{align}
Applying the divergence theorem to each of these two equations and incorporating the Neumann 
boundary conditions 
yields
\begin{align}
  0 &= \int_\Omega \trialvec[u] \cdot \vec{f}(\vec{x},t) + \nabla \trialvec[u] : -
\tensor{\sigma}(\vec{u},p_f) \, 
d\Omega + \int_{\Gamma_\tau} \trialvec[u] \cdot \vec{\tau}(\vec{x},t) \, d\Gamma, \\
%
 \int_\Omega  \trialscalar[p] \frac{\partial \zeta(\vec{u},p_f)}{\partial t} \, d\Omega &= 
 \int_\Omega \trialscalar[p] \gamma(\vec{x},t) + \nabla \trialscalar[p] \cdot \vec{q}(p_f) \, 
d\Omega
 + \int_{\Gamma_q} \trialscalar[p] (-q_0(\vec{x},t)) \, d\Gamma.
\end{align}
Identifying $F(t,s,\dot{s})$ and $G(t,s)$ we have
\begin{alignat}{2}
  F^u(t,s,\dot{s}) &= \vec{0},
  & \quad
  G^u(t,s) &= \int_\Omega \trialvec[u] \cdot \eqnannotate{\vec{f}(\vec{x},t)}{g^u_0} + \nabla 
\trialvec[u] : 
\eqnannotate{-\tensor{\sigma}(\vec{u},p_f)}{g^u_1} \, d\Omega + \int_{\Gamma_\tau} 
\trialvec[u] \cdot 
\eqnannotate{\vec{\tau}(\vec{x},t)}{g^u_0} \, d\Gamma, \\
  F^p(t,s,\dot{s}) &= \int_\Omega  \trialscalar[p] \eqnannotate{\frac{\partial 
\zeta(\vec{u},p_f)}{\partial t}}
{f^p_0} \, d\Omega
  & \quad
  G^p(t,s) &= \int_\Omega \trialscalar[p] \eqnannotate{\gamma(\vec{x},t)}{g^p_0} + \nabla 
\trialscalar[p] \cdot 
\eqnannotate{\vec{q}(p_f)}{g^p_1} \, d\Omega
 + \int_{\Gamma_q} \trialscalar[p] (\eqnannotate{-q_0(\vec{x},t)}{g^p_0}) \, d\Gamma
\end{alignat}
