% ----------------------------------------------------------------------
\section{Poroelasticity with Infinitesimal Strain and No Faults}

We base this formulation for poroelsticity on Zheng et al. and
Detournay and Cheng (1993). We assume a slightly compressible fluid
that completely saturates a porous solid, undergoing infinitesimal
strain.

We begin with the conservation of linear momentum, including inertia,
borrowed from linear elasticity:
\begin{equation}
    \rho_s\frac{\partial^2 \vec{u}}{\partial t^2} = \vec{f}(t) + \nabla \cdot \tensor{\sigma}(\vec{u},p),
\end{equation}
Enforcing mass balance of the fluid gives
\begin{gather}
  \frac{\partial \zeta(\vec{u},p)}{\partial t} + \nabla \cdot \vec{q}(p) =
  \gamma(\vec{x},t) \text{ in } \Omega, \\
%
  \vec{q} \cdot \vec{n} = q_0(\vec{x},t) \text{ on }\Gamma_q, \\
%
  p = p_0(\vec{x},t) \text{ on }\Gamma_p,
\end{gather}
where $\zeta$ is the variation in fluid content, $\vec{q}$ is the rate
of fluid volume crossing a unit area of the porous solid, $\gamma$ is
the rate of injected fluid per unit volume of the porous solid, $q_0$
is the outward fluid velocity normal to the boundary $\Gamma_q$, and
$p_0$ is the fluid pressure on boundary $\Gamma_p$.\\

We require the fluid flow to follow Darcy's law (Navier-Stokes
equation neglecting inertial effects),
\begin{equation}
  \vec{q}(p) = -\frac{\tensor{k}}{\mu_{f}}(\nabla p - \vec{f}_f), \\
\end{equation}
where $\tensor{k}$ is the intrinsic permeability, $\mu_f$ is the viscosity of the
fluid, $p$ is the fluid pressure, and $\vec{f}_f$ is the body force
in the fluid. If gravity is included in a problem, then usually
$\vec{f}_f = \rho_f \vec{g}$, where $\rho_f$ is the density of the
fluid and $\vec{g}$ is the gravitational acceleration vector.\\

\subsection{Constitutive Behavior}

We assume linear elasticity for the solid phase, so the constitutive behavior can be expressed
as
\begin{equation}
  \tensor{\sigma}(\vec{u},p) = \tensor{C} : \tensor{\epsilon} - \alpha p \tensor{I},
\end{equation}
where $\tensor{\sigma}$ is the stress tensor, $\tensor{C}$ is the
tensor of elasticity constants, $\alpha$ is the Biot coefficient
(effective stress coefficient), $\tensor{\epsilon}$ is the strain
tensor, and $\tensor{I}$ is the identity tensor.\\
For this case, we will assume that the material properties are isotropic, resulting
in the following formulation for the stress tensor:
\begin{equation}
    \tensor{\sigma}(\vec{u},p) = \tensor{C}:\tensor{\epsilon} - \alpha p \tensor{I}
                                           = \lambda \tensor{I} \epsilon_{v} + 2 \mu - \alpha \tensor{I} p
\end{equation}
where $\lambda$ and $\mu$ are Lam\'e's parameters,
$\lambda = K_{d} - \frac{2 \mu}{3}$, $\mu$ is the shear modulus, and
the volumetric strain is defined as
$\epsilon_{v} = \nabla \cdot \vec{u}$.

For the constitutive behavior of the fluid, we use the volumetric
strain to couple the fluid-solid behavior,
\begin{gather}
  \zeta(\vec{u},p) = \alpha \Tr({\tensor{\epsilon}}) + \frac{p}{M}, \\
%
  \frac{1}{M} = \frac{\alpha-\phi}{K_s} + \frac{\phi}{K_f},
\end{gather}
where $1/M$ is the specific storage coefficient at constant strain,
$K_s$ is the bulk modulus of the solid, and $K_f$ is the bulk modulus
of the fluid. We can write the trace of the strain tensor as the dot
product of the gradient and displacement field, so we have
\begin{equation}
  \zeta(\vec{u},p) = \alpha (\nabla \cdot \vec{u}) + \frac{p}{M}.
\end{equation}

\begin{table}[htbp]
  \caption{Mathematical notation for poroelasticity with
    infinitesimal strain.}
  \label{tab:notation:poroelasticity}
  \begin{tabular}{lcp{3.5in}}
    \toprule
    {\bf Category} & {\bf Symbol} & {\bf Description} \\
    \midrule
    Unknowns           & $\vec{u}$ & Displacement field \\
                       & $\vec{v}$ & Velocity field \\
                       & $p$       & Pressure field (corresponds to pore fluid pressure) \\
                       & $\epsilon_{v}$ & Volumetric (trace) strain \\
    \hline
    Derived quantities & $\tensor{\sigma}$ & Cauchy stress tensor \\
                       & $\tensor{\epsilon}$ & Cauchy strain tensor \\
                       & $\zeta$ & Variation of fluid content (variation of fluid vol. per unit vol. of PM), $\alpha \epsilon_{v} + \frac{p}{M}$ \\
                       & $\rho_{b}$ & Bulk density, $\left(1 - \phi\right) \rho_{s} + \phi \rho_{f}$ \\
                       & $\vec{q}$ & Darcy flux, $-\frac{\tensor{k}}{\mu_{f}} \cdot \left(\nabla p - \vec{f}_{f}\right)$ \\
                       & $M$ & Biot Modulus, $\frac{K_{f}}{\phi} + \frac{K_{s}}{\alpha - \phi}$ \\
    \hline
    Common constitutive parameters & $\rho_{f}$ & Fluid density \\
                       & $\rho_{s}$ & Solid (matrix) density \\
                       & $\phi$ & Porosity \\
                       & $\tensor{k}$ & Permeability \\
                       & $\mu_{f}$ & Fluid viscosity \\
                       & $K_{s}$ & Solid grain bulk modulus \\
                       & $K_{f}$ & Fluid bulk modulus \\
                       & $K_{d}$ & Drained bulk modulus \\
                       & $\alpha$ & Biot coefficient, $1 - \frac{K_{d}}{K_{s}}$ \\
    \hline
    Source terms       & $\vec{f}$ & Body force per unit volume, for example: $\rho_{b} \vec{g}$ \\
                       & $\vec{f}_{f}$ & Fluid body force, for example: $\rho_{f} \vec{g}$ \\
                       & $\gamma$ & Source density; rate of injected fluid per unit volume of the porous solid \\
    \bottomrule
  \end{tabular}
\end{table}


\subsection{Quasistatic}

For ease of solution in the quasistatic case, we introduce a third
variable in the form of volumetric strain ($\epsilon_v$).  The
strong form of the problem may be expressed as
\begin{gather}
% Solution
\vec{s}^{T} = \left(\vec{u} \quad p \quad \epsilon_v\right), \\
% Elasticity
\vec{f}(t) + \nabla \cdot \tensor{\sigma}(\vec{u},p) = \vec{0} \text{ in } \Omega, \\
% Pressure
\frac{\partial \zeta(\vec{u},p)}{\partial t} - \gamma(\vec{x},t) + \nabla \cdot \vec{q}(p) = 0 \text{ in } \Omega, \\
% Vol. Strain
\nabla \cdot \vec{u} - \epsilon_{v} = 0 \text{ in } \Omega, \\
% Neumann traction
\tensor{\sigma} \cdot \vec{n} = \vec{\tau}(\vec{x},t) \text{ on } \Gamma_{\tau}, \\
% Dirichlet displacement
\vec{u} = \vec{u}_0(\vec{x}, t) \text{ on } \Gamma_{u}, \\
% Neumann flow
\vec{q} \cdot \vec{n} = q_0(\vec{x}, t) \text{ on } \Gamma_{q}, \text{ and } \\
% Dirichlet pressure
p = p_0(\vec{x},t) \text{ on } \Gamma_{p}.
\end{gather}
%
We create the weak form by taking the dot product with the trial
functions $\trialvec[u]$, $\trialscalar[p]$, and
$\trialscalar[\epsilon_{v}]$ and
integrating over the domain:
\begin{gather}
% Weak conservation of momentum
\int_\Omega \trialvec[u] \cdot \left( \vec{f}(\vec{x},t) + \tensor{\nabla} \cdot \tensor{\sigma} (\vec{u},p) \right) \, d\Omega = 0, \\
% Weak conservation of mass
\int_\Omega  \trialscalar[p] \left( \frac{\partial \zeta(\vec{u},p)}{\partial t} - \gamma(\vec{x},t) + \nabla \cdot \vec{q}(p)\right) \, d\Omega = 0,\\
% Weak vol. strain
\int_{\Omega} \trialscalar[\epsilon_{v}]\cdot \left( \nabla \cdot \vec{u} - \epsilon_v \right) \, d\Omega.
\end{gather}
%
Applying the divergence theorem to the first two equations and
incorporating the Neumann boundary conditions yields
\begin{gather}
% Weak conservation of momentum
\int_\Omega \trialvec[u] \cdot \vec{f}(\vec{x},t) + \nabla \trialvec[u] : -\tensor{\sigma}(\vec{u},p_f) \,
d\Omega + \int_{\Gamma_\tau} \trialvec[u] \cdot \vec{\tau}(\vec{x},t) \, d\Gamma = 0, \\
% Weak conservation of mass
\int_\Omega  \trialscalar[p] \left( \frac{\partial \zeta(\vec{u},p_f)}{\partial t} - \gamma(\vec{x},t)\right)
+ \nabla \trialscalar[p] \cdot \left(-\vec{q}(p_f)\right) \, d\Omega + \int_{\Gamma_q} \trialscalar[p] q_0(\vec{x},t))\, d\Gamma = 0, \text{ and } \\
% Weak vol. strain
\int_{\Omega} \trialscalar[\epsilon_{v}] \cdot \left(\nabla \cdot \vec{u} - \epsilon_{v} \right) d\Omega = 0
\end{gather}

\subsubsection{Residual Pointwise Functions}

Identifying $F(t,s,\dot{s})$ and $G(t,s)$we have
\begin{align}
    % Displacement
  F^u(t,s,\dot{s}) &= \int_\Omega \trialvec[u] \cdot \eqnannotate{\vec{f}(\vec{x},t)}{g^u_0}
                     + \nabla \trialvec[u] : \eqnannotate{-\tensor{\sigma}(\vec{u},p_f)}{g^u_1} \, d\Omega
                     + \int_{\Gamma_\tau} \trialvec[u] \cdot \eqnannotate{\vec{\tau}(\vec{x},t)}{g^u_0} \, d\Gamma, \\
% Pressure
  F^p(t,s,\dot{s}) &= \int_\Omega  \trialscalar[p] \eqnannotate{\frac{\partial \zeta(\vec{u},p_f)}{\partial t} - \gamma(\vec{x},t)} {f^p_0} \, d\Omega
                     + \nabla \trialscalar[p] \cdot \eqnannotate{-\vec{q}(p_f)}{f^p_1} \, d\Omega
                     + \int_{\Gamma_q} \trialscalar[p] (\eqnannotate{q_0(\vec{x},t)}{f^p_0}) \, d\Gamma, \\
 % Volumetric Strain
  F^{\epsilon_{v}}(t,s,\dot{s}) &= \trialscalar[\epsilon_{v}] \cdot \eqnannotate{\left(\nabla \cdot \vec{u} - \epsilon_{v} \right)}{g^{\epsilon_{v}}_{0}} \, d\Omega. \\
  G^u(t,s) &= 0, \\
             G^p(t,s) &= 0, \\
 G^{\epsilon_v} &= 0. 
\end{align}

\subsubsection{Jacobian Pointwise Functions}

The LHS Jacobians are:

\begin{align}
    %
J_F^{uu} &= \frac{\partial F^u}{\partial u} + t_{shift} \frac{\partial F^u}{\partial \dot{u}} = 0 \\
%
J_F^{up} &= \frac{\partial F^u}{\partial p} + t_{shift} \frac{\partial F^u}{\partial \dot{p}} = 0 \\
%
J_F^{u \epsilon_{v}} &= \frac{\partial F^u}{\partial \epsilon_{v}} + t_{shift} \frac{\partial F^u}{\partial \dot{\epsilon_{v}}} = 0 \\
%
J_F^{pu} &= \frac{\partial F^p}{\partial u} + t_{shift} \frac{\partial F^p}{\partial \dot{u}} = 0 \\
% Jf0pp
J_F^{pp} &= \frac{\partial F^p}{\partial p} + t_{shift} \frac{\partial F^p}{\partial \dot{p}} = \int_{\Omega} \trialscalar[p]
\eqnannotate{\left(t_{shift} \frac{1}{M}\right)}{J_{f0}^{pp}} \basisscalar[p] \ d\Omega \\
% Jf0pe
J_F^{p\epsilon_{v}} &= \frac{\partial F^p}{\partial \epsilon_{v}} + t_{shift} \frac{\partial
F^p}{\partial \dot{\epsilon_{v}}} = \int_{\Omega} \trialscalar[p] \eqnannotate{\left(t_{shift} \alpha \right)}{J_{f0}^{p\epsilon_{v}}}
\basisscalar[\epsilon_{v}] \ d\Omega \\
%
J_F^{\epsilon_{v}u} &= \frac{\partial F^{\epsilon_{v}}}{\partial u} + t_{shift} \frac{\partial F^{\epsilon_{v}}}{\partial \dot{u}} = 0 \\
%
J_F^{\epsilon_{v}p} &= \frac{\partial F^{\epsilon_{v}}}{\partial p} + t_{shift} \frac{\partial F^{\epsilon_{v}}}{\partial \dot{p}} = 0 \\
%
J_F^{\epsilon_{v}\epsilon_{v}} &= \frac{\partial F^\epsilon_{v}}{\epsilon_{v}} + t_{shift} \frac{\partial F^{\epsilon_{v}}}{\partial \dot{\epsilon_{v}}} = 0
\end{align}

The RHS Jacobians are:

\begin{align}
%% Jg3uu
J_G^{uu} &= \frac{\partial G^u}{\partial u} = \int_{\Omega} \nabla \trialvec[u] : \frac{\partial}{\partial u} (- \sigma(\vec{u},p,\epsilon_{v})) \
d\Omega = \int_{\Omega} \nabla \trialvec[u] : \frac{\partial}{\partial u} (-(\tensor{C}:\tensor{\varepsilon} -\alpha p \tensor{I})) \ d\Omega \\
&= \int_{\Omega} \nabla \trialvec[u] : -\tensor{C}: \frac{1}{2} (\nabla + \nabla^T) \basisvec[u] \ d\Omega = \int_{\Omega} \trialscalar[u]_{i,k}
\eqnannotate{\left(-C_{ikjl}\right)}{J_{g3}^{uu}} \basisscalar[u]_{j,l} \ d\Omega
\\
%
%% Jg2up
J_G^{up} &= \frac{\partial G^u}{\partial p} = \int_{\Omega} \nabla \trialvec[u] : \frac{\partial}{\partial p}(-(\tensor{C}:\tensor{\varepsilon} -\alpha p \tensor{I})) \ d\Omega =
\int_{\Omega} \trialscalar[u]_{i,j} \eqnannotate{\left(\alpha \delta_{ij}\right)}{J_{g2}^{up}} \basisscalar[p] \ d\Omega \\
%
%% Jg2ue
J_G^{u\epsilon_{v}} &= \frac{\partial G^u}{\partial \epsilon_{v}} = \int_{\Omega} \nabla \trialvec[u] : \frac{\partial}{\partial \epsilon_{v}}
(-\sigma(\vec{u},p,\epsilon_{v})) \ d\Omega = \int_{\Omega} \nabla \trialvec[u] :
\frac{\partial}{\partial \epsilon_{v}} (-(\tensor{C}:\tensor{\varepsilon} -\alpha p \tensor{I})) \ d\Omega \\
&= \int_{\Omega} \nabla \trialvec[u] : \frac{\partial}{\partial \epsilon_{v}} \
\left[-\left(2 \mu \tensor{\epsilon} + \lambda \tensor{I} \epsilon_{v} - \alpha \tensor{I} p \right) \right] d\Omega =
\int_{\Omega} \trialscalar[u]_{i,j} \eqnannotate{\left(-\lambda \delta_{ij} \right)}{J_{g2}^{u \epsilon_{v}}} \basisscalar[\epsilon_{v}] d\Omega  \\
%
J_G^{pu} &= \frac{\partial G^p}{\partial u} = 0 \\
%
%% Jg3pp
J_G^{pp} &= \frac{\partial G^p}{\partial p} = \int_{\Omega} \nabla
\trialscalar[p] \cdot \frac{\partial}{\partial p} \left[-\frac{\tensor{k}}{\mu_{f}} \left(\nabla p -
\vec{f} \right) \right] \ d\Omega \\
&= \int_{\Omega} \nabla \psi_{trial}^ p \left(-\frac{\tensor{k}}{\mu_{f}}
\nabla \cdot \psi_{basis}^p \right) \ d\Omega \\
&= \int_{\Omega} \psi_{trial,k}^p \eqnannotate{\left(-\frac{\tensor{k}}{\mu_{f}}
\delta_{kl}\right)}{J_{g3}^{pp}}
\psi_{basis,l}^p \ d\Omega \\
%
J_G^{p\epsilon_{v}} &= \frac{\partial G^p}{\partial \epsilon_{v}} = 0 \\
% Jg1eu
J_G^{\epsilon_{v}u} &= \frac{\partial G^{\epsilon_{v}}}{\partial u} =
\int_{\Omega} \psi_{trial}^{\epsilon_{v}} \nabla \cdot \vec{\psi}_{basis}^u \ d\Omega = \int_{\Omega}
\psi_{trial}^{\epsilon_{v}}
\eqnannotate{\left(\delta_{ij}\right)}{J_{g1}^{\epsilon_{v}u}}
\psi_{basis
i,j}^u \
d\Omega\\
%
J_G^{\epsilon_{v}p} &= \frac{\partial G^{\epsilon_{v}}}{\partial p} = 0 \\
% Jg0ee
J_G^{\epsilon_{v} \epsilon_{v}} &= \frac{\partial G^{\epsilon_{v}}}{\partial
\epsilon_{v}} = \int_{\Omega} \psi_{trial}^{\epsilon_{v}}
\eqnannotate{\left(-1\right)}{J_{g0}^{\epsilon_{v}\epsilon_{v}}}
\psi_{basis}^{\epsilon_{v}} \ d\Omega
\end{align} \\


\subsection{Dynamic}

For compatibility with PETSc TS algorithms, we want to turn the second
order equation elasticity equation into two first order equations. We
introduce the velocity as a unknown,
$\vec{v}=\frac{\partial u}{\partial t}$, which leads to a slightly
different three field problem,
\begin{gather}
% Solution
\vec{s}^{T} = \left(\vec{u} \quad p \quad \vec{v}\right) \\
% Displacement
\frac{\partial \vec{u}}{\partial t} = \vec{v} \text{ in } \Omega \\
% Pressure
\zeta(\vec{u},p) = \gamma(\vec{x},t) + \nabla \cdot \vec{q}(p) \text{ in } \Omega \\
% Velocity
\rho_{b} \frac{\partial \vec{v}}{\partial t} = \vec{f}(\vec{x},t) + \nabla \cdot \tensor{\sigma}(\vec{u},p) \text{ in } \Omega \\
% Neumann traction
\tensor{\sigma} \cdot \vec{n} = \vec{\tau}(\vec{x},t) \text{ on } \Gamma_{\tau} \\
% Dirichlet displacement
\vec{u} = \vec{u}_{0}(\vec{x}, t) \text{ on } \Gamma_{u} \\
% Neumann flow
\vec{q} \cdot \vec{n} = q_{0}(\vec{x}, t) \text{ on } \Gamma_{q} \\
% Dirichlet pressure
p = p_{0}(\vec{x},t) \text{ on } \Gamma_{p}
\end{gather}



Using trial functions $\trialvec[u]$, $\trialscalar[p]$, and $\trialvec[v]$, and incorporating the
Neumann boundary conditions, the weak form may be written as:

\begin{align}
    % Displacement
    \int_{\Omega} \trialvec[u] \cdot \left( \frac{\partial \vec{u}}{\partial t} \right)d \Omega &= \int_{\Omega} \trialvec[u] \cdot \left( \vec{v} \right) d \Omega \\
    % Pressure
    \int_{\Omega} \trialscalar[p] \left( \frac{\partial \zeta (\vec{u},p_f)}{\partial t} \right) d\Omega &=
    \int_{\Omega} \trialscalar[p] \gamma(\vec{x},t) + \nabla \trialscalar[p] \cdot \vec{q}(p_f) \ d\Omega +
   \int_{\Gamma_q} \trialscalar[p] (-q_0(\vec{x},t)) \ d\Gamma, \\
   % Velocity
   \int_\Omega \trialvec[v] \cdot \left( \rho_{b} \frac{\partial
   \vec{v}}{\partial t} \right) \,
   d\Omega &= \int_\Omega \trialvec[v] \cdot \vec{f}(\vec{x},t) + \nabla \trialvec[v] :
   -\tensor{\sigma} (\vec{u},p_f) \, d\Omega + \int_{\Gamma_\tau} \trialvec[u]
   \cdot \vec{\tau}(\vec{x},t) \, d\Gamma.
\end{align}

\subsubsection{Residual Pointwise Kernels}

Identifying $F(t,s,\dot{s})$ and $G(t,s)$ we have

\begin{alignat}{2}
% Displacement
  F^u(t,s,\dot{s}) &= \int_{\Omega} \trialvec[u] \cdot \eqnannotate{\left( \frac{\partial \vec{u}}{\partial t} \right)}{f_{0}^{u}} d \Omega;
  & \quad
  G^u(t,s) &= \int_{\Omega} \trialvec[u] \cdot \eqnannotate{\left( \vec{v} \right)}{g_{0}^{u}} d \Omega, \\
% Pressure
  F^p(t,s,\dot{s}) &= \int_\Omega  \trialscalar[p] \eqnannotate{\frac{\partial \zeta(\vec{u},p_f)}{\partial t}} {f^p_0} \, d\Omega;
  & \quad
  G^p(t,s) &= \int_\Omega \trialscalar[p] \eqnannotate{\gamma(\vec{x},t)}{g^p_0} + \nabla
\trialscalar[p] \cdot \eqnannotate{\vec{q}(p_f)}{g^p_1} \, d\Omega
 + \int_{\Gamma_q} \trialscalar[p] (\eqnannotate{-q_0(\vec{x},t)}{g^p_0}) \, d\Gamma, \\
 % Velocity
  F^v(t,s,\dot{s}) &= \int_\Omega \trialvec[v] \cdot \eqnannotate{\left( \rho_{b} \frac{\partial \vec{v}}{\partial t} \right)}{f^{v}_{0}} \, d\Omega;
 & \quad
 G^v(t,s) &= \int_\Omega \trialvec[v] \cdot \eqnannotate{\vec{f}(\vec{x},t)}{g^{v}_{0}} + \nabla \trialvec[v] :\eqnannotate{-\tensor{\sigma} (\vec{u},p_f)}{g^{v}_{1}} \, d\Omega + \int_{\Gamma_\tau} \trialvec[u] \cdot \eqnannotate{\vec{\tau}(\vec{x},t)}{g^{v}_{0}} \, d\Gamma.
\end{alignat}

\subsubsection{Jacobians Pointwise Kernels}

The LHS Jacobians are:

\begin{align}
% Jf0uu
J_F^{uu} &= \frac{\partial F^u}{\partial u} + t_{shift} \frac{\partial F^u}{\partial \dot{u}} =
\int_{\Omega} \trialscalar[u]_{i} \eqnannotate{t_{shift} \delta_{ij}}{J^{uu}_{f0}} \basisscalar[u]_{j} \, d \Omega \\
%
J_F^{up} &= \frac{\partial F^u}{\partial p} + t_{shift} \frac{\partial F^u}{\partial \dot{p}} = 0 \\
%
J_F^{uv} &= \frac{\partial F^u}{\partial v} + t_{shift} \frac{\partial F^u}{\partial \dot{v}} = 0 \\
%
J_F^{pu} &= \frac{\partial F^p}{\partial u} + t_{shift} \frac{\partial F^p}{\partial \dot{u}} = 0 \\
% Jf0pp
J_F^{pp} &= \frac{\partial F^p}{\partial p} + t_{shift} \frac{\partial F^p}{\partial \dot{p}} = \int_{\Omega} \trialscalar[p]
\eqnannotate{\left(t_{shift} \frac{1}{M}\right)}{J_{f0}^{pp}} \basisscalar[p] \ d\Omega \\
%
J_F^{pv} &= \frac{\partial F^p}{\partial v} + t_{shift} \frac{\partial F^p}{\partial \dot{v}} = 0 \\
%
J_F^{vu} &= \frac{\partial F^v}{\partial u} + t_{shift} \frac{\partial F^v}{\partial \dot{u}} = 0 \\
%
J_F^{vp} &= \frac{\partial F^v}{\partial p} + t_{shift} \frac{\partial F^v}{\partial \dot{p}} = 0 \\
% Jf0vv
J_F^{vv} &= \frac{\partial F^v}{\partial v} + t_{shift} \frac{\partial F^v}{\partial \dot{v}} =
\int_{\Omega} \trialscalar[v]_{i}\eqnannotate{\rho_{b}(\vec{x}) t_{shift} \delta_{ij}}{J^{vv}_{f0}} \basisscalar[v]_{j} \ , d \Omega
\end{align}

The RHS Jacobians are:

\begin{align}
%%
J_G^{uu} = \frac{\partial G^u}{\partial u} &= 0 \\
%
J_G^{up} = \frac{\partial G^u}{\partial p} &= 0 \\
% Jg0uv
J_G^{uv} = \frac{\partial G^u}{\partial v} &= \int_{\Omega} \trialscalar[u]_{i} \eqnannotate{\delta_{ij}}{J^{uv}_{g0}} \basisscalar[v]_{j} \, d\Omega \\
J_G^{pu} = \frac{\partial G^p}{\partial u} &= 0 \\
% Jg3pp
J_G^{pp} = \frac{\partial G^p}{\partial p} &= \int_{\Omega} \nabla
\trialscalar[p] \cdot \frac{\partial}{\partial p} \left[-\frac{\tensor{k}}{\mu_{f}} \left(\nabla p -
\vec{f} \right) \right] \ d\Omega \\
&= \int_{\Omega} \nabla \psi_{trial}^ p \left(-\frac{\tensor{k}}{\mu_{f}}
\nabla \cdot \psi_{basis}^p \right) \ d\Omega \\
&= \int_{\Omega} \trialscalar[p]_{k} \eqnannotate{\left(-\frac{\tensor{k}}{\mu_{f}}
\delta_{kl}\right)}{J_{g3}^{pp}}
\basisscalar[p]_{l} \ d\Omega \\
%
J_G^{pv} = \frac{\partial G^p}{\partial v} &= 0 \\
% Jg3vu
J_G^{vu} = \frac{\partial G^v}{\partial u} &= \int_{\Omega} \nabla \trialvec[v] : \frac{\partial}{\partial u} \left(- \sigma(\vec{u},p)\right) \, d\Omega
= \int_{\Omega} \nabla \trialvec[v] : \frac{\partial}{\partial u} \left[ -\left(\tensor{C} : \tensor{\varepsilon} -\alpha p \tensor{I} \right)\right] \ d\Omega \\
&= \int_{\Omega} \nabla \trialvec[v] : -\tensor{C}: \frac{1}{2} (\nabla + \nabla^T) \basisvec[u] \, d\Omega = \int_{\Omega} \trialscalar[v]_{i,k} \eqnannotate{ \left(-C_{ikjl} \right) }{J_{g3}^{vu} } \basisscalar[u]_{j,l} \, d \Omega \\
% Jg2vp
J_G^{vp} = \frac{\partial G^u}{\partial p} &= \int_{\Omega} \nabla \trialvec[u] : \frac{\partial}{\partial p}(-(\tensor{C}:\tensor{\varepsilon} -\alpha p \tensor{I})) \ d\Omega = \int_{\Omega} \trialscalar[u]_{i,j} \eqnannotate{\left(\alpha \delta_{ij}\right)}{J_{g2}^{up}} \basisscalar[p] \ d\Omega \\
%
J_G^{vv} = \frac{\partial G^v}{\partial v} &= 0
\end{align}
