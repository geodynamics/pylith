% ----------------------------------------------------------------------
\section{Incompressible Isotropic Elasticity with Infinitesimal Strain (Bathe)}

In this section we apply a similar approach to the one we use for the
elasticity equation to the case of an incompressible material. We only
consider the quasistatic case (neglect inertia) without faults. As the
bulk modulus ($K$) approaches infinity, the volumetric strain
($\Tr(\epsilon)$) approaches zero and the pressure remains finite,
$p = -K \Tr(\epsilon)$. We consider pressure $p$ as an independent
variable and decompose the stress into the pressure and deviatoric
components. As a result, we write the stress tensor in terms of both
the displacement and pressure fields,
\begin{equation}
  \tensor{\sigma}(\vec{u},p) = \tensor{\sigma}^\mathit{dev}(\vec{u}) - p\tensor{I}.
\end{equation}

The strong form is
\begin{gather}
  % Solution
  \vec{s}^T = \left( \vec{u} \quad \ p \right)^T, \\
  % Elasticity
  \vec{f}(t) + \tensor{\nabla} \cdot \left(\tensor{\sigma}^\mathit{dev}(\vec{u}) - p\tensor{I}\right) = \vec{0} \text{ in }\Omega, \\
  % Pressure
  \vec{\nabla} \cdot \vec{u} + \frac{p}{K} = 0 \text{ in }\Omega, \\
  % Neumann
  \tensor{\sigma} \cdot \vec{n} = \vec{\tau} \text{ on }\Gamma_\tau, \\
  % Dirichlet
  \vec{u} = \vec{u}_0 \text{ on }\Gamma_u, \\
  p = p_0 \text{ on }\Gamma_p.
\end{gather}

\begin{table}[htbp]
  \caption{Mathematical notation for incompressible elasticity with
    infinitesimal strain.}
  \label{tab:notation:incompressible:elasticity}
  \begin{tabular}{lcp{3.5in}}
    \toprule
    {\bf Category} & {\bf Symbol} & {\bf Description} \\
    \midrule
    Unknowns & $\vec{u}$ & Displacement field \\
    & $p$ & Pressure field ($p>0$ corresponds to negative mean stress \\
    Derived quantities & $\tensor{\sigma}$ & Cauchy stress tensor \\
                   & $\tensor{\epsilon}$ & Cauchy strain tensor \\
    Common constitutive parameters & $\rho$ & Density \\
  & $\mu$ & Shear modulus \\
  & $K$ & Bulk modulus \\
Source terms & $\vec{f}$ & Body force per unit volume, for example $\rho \vec{g}$ \\
    \bottomrule
  \end{tabular}
\end{table}

Using trial functions $\trialvec[u]$ and $\trialscalar[p]$ and
incorporating the Neumann boundary conditions, we write the weak form
as
\begin{gather}
  % Displacement
  \int_\Omega \trialvec[u] \cdot \vec{f}(t) + \nabla \trialvec[u] : \left(-\tensor{\sigma}^\mathit{dev}(\vec{u}) + p\tensor{I}
  \right)\, d\Omega + \int_{\Gamma_\tau} \trialvec[u] \cdot \vec{\tau}(t) \, d\Gamma, = 0 \\
  % Pressure
  \int_\Omega \trialscalar[p] \cdot \left(\vec{\nabla} \cdot \vec{u} + \frac{p}{K} \right) \, d\Omega = 0.
\end{gather}

\subsection{Residual Pointwise Functions}

Identifying $F(t,s,\dot{s})$, we have
\begin{gather}
  \label{eqn:incompressible:elasticity:displacement}
  F^u(t,s,\dot{s}) = \int_\Omega \trialvec[u] \cdot \eqnannotate{\vec{f}(t)}{f_0^u} + \nabla \trialvec[u] :
  \eqnannotate{\left(-\tensor{\sigma}^\mathit{dev}(\vec{u}) + p\tensor{I}\right)}{f_1^u}  \, d\Omega
  + \int_{\Gamma_\tau} \trialvec[u] \cdot \eqnannotate{\vec{\tau}(t)}{f_0^u} \, d\Gamma, \\
%
  \label{eqn:incompressible:elasticity:pressure}
  F^p(t,s,\dot{s}) = \int_\Omega \trialscalar[p] \cdot \eqnannotate{\left(\vec{\nabla} \cdot \vec{u} + 
\frac{p}{K} \right)}{f_0^p} \, d\Omega.
\end{gather}

\subsection{Jacobians Pointwise Functions}

With two fields we have four Jacobian pointwise functions for the LHS:
\begin{align}
  % JF uu
  J_F^{uu} &= \frac{\partial F^u}{\partial u} + s_\mathit{tshift} \frac{\partial F^u}{\partial \dot{u}} =
             \int_\Omega \nabla \trialvec[u] : \frac{\partial}{\partial u}(-\tensor{\sigma}^\mathit{dev}) \, d\Omega 
             = \int_\Omega \trialscalar[u]_{i,k} \, \eqnannotate{\left(-C^\mathit{dev}_{ikjl}\right)} {J_{f3}^{uu}}  \, \basisscalar[u]_{j,l}\, d\Omega \\
  % JF up
  J_F^{up} &= \frac{\partial F^u}{\partial p} + s_\mathit{tshift} \frac{\partial F^u}{\partial \dot{p}} =
             \int_\Omega \nabla\trialvec[u] : \tensor{I} \basisscalar[p] \,  d\Omega
             = \int_\Omega \trialscalar[u]_{i,k} \eqnannotate{\delta_{ik}}{J_{f2}^{up}} \, \basisscalar[p] \, d\Omega \\
  % JF pu
  J_F^{pu} &= \frac{\partial F^p}{\partial u} + s_\mathit{tshift} \frac{\partial F^p}{\partial \dot{u}} =
             \int_\Omega \trialscalar[p] \left(\vec{\nabla}  \cdot \basisvec[u]\right) \, d\Omega
             = \int_\Omega \trialscalar[p] \eqnannotate{\delta_{jl}}{J_{f1}^{pu}} \basisscalar[u]_{j,l} \, d\Omega\\
  % JF pp
  J_F^{pp} &= \frac{\partial F^p}{\partial p}  + s_\mathit{tshift} \frac{\partial F^p}{\partial \dot{p}} =
             \int_\Omega \trialscalar[p] \eqnannotate{\frac{1} {K}}{J_{f0}^{pp}} \basisscalar[p] \, d\Omega
\end{align}

For isotropic, linear incompressible elasticity, the deviatoric elastic constants are:
\begin{align}
    C_{1111} &= C_{2222} = C_{3333} = +\frac{4}{3} \mu \\
    C_{1122} &= C_{1133} = C_{2233} = -\frac{2}{3} \mu \\
    C_{1212} &= C_{1313} = C_{2323} = \mu
\end{align}
