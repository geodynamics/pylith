\documentclass[12pt]{article}

\usepackage{graphicx}
\usepackage{amssymb}
\usepackage{amsmath}
\usepackage{color}
\usepackage{url}

\title{Sieve Notes}
\author{Matthew G. Knepley\\
\small Computation Institute\\[-0.8ex]
\small University of Chicago, Chicago, IL\\
\small \texttt{knepley@ci.uchicago.edu}\\
}

\begin{document}
\maketitle

\section{Point Numbering}

In PyLith, we choose to use integers to refer to Sieve points. Thus,
we are left with the task of assigning semantics to certain
ranges. Without this, debugging becomes quite difficult and quick
range checks qould be replaced by costly depth checks in many
algorithms.

The first assumption, also present in much of Sieve itself, is that
cells are numbered first, followed by vertices, and then all other
intermediate topological pieces, such as edges and faces. This simple
scheme is enough for most applications, however the introduction of
faults and cohesive cells in PyLith poses additional problems.

When faults are introduced, we must also assign numbers to three more
types of topological pieces:
\begin{itemize}
  \item cohesive cells
  \item new fault vertices, which are duplicate vertices on the far
    side of the fault
  \item Lagrange vertices, which implement Lagrange multiplier constraints
\end{itemize}
Before creating the faults, we know only the number of vertices in
each fault group. Currently, we number new fault vertices in a block,
then the Lagrange vertices, and then cohesive cells. In each division,
the new points for each fault are numbered contiguously. Sieve viewers
use a variant depth label, ``censored depth'', to exclude some points
from the output. The censored depth is computed by rejecting both the
Lagrange vertices and cohesive cells.

\section{Point Depth and Height}

In general, vertices are at a depth of 0 and cells are at the maximum
depth. Similarly, cells are at a height of 0 and vertices are at the
maximum depth. Notice that depth and height correspond the the usual
dimension and codimension. For a 3-D mesh with vertices, edges, faces,
and cells:

\begin{table}
\begin{center}
\begin{tabular}{|l|c|c|}
\hline
               & Depth & Height \\
\hline
    vertices   &   0   &   3 \\
    edges      &   1   &   2 \\
    faces      &   2   &   1 \\
    cells      &   3   &   0 \\
\hline
\end{tabular}
\end{center}
\caption{Depth and height of various topological pieces.}
\end{table}

  For a boundary mesh, we currently store the full set (vertices,
  edges, faces, and cells). Obviously for 2-D meshes, the boundary
  mesh doesn't contain "volume" cells, but just vertices, edges, and
  faces. This means the "boundary" cells are at a height of 1 and
  maximum depth - 1 .

  The fault mesh is generated with its own numbering, so it is
  different and it only contains vertices and "faces" where "faces"
  means faces for a 3-D mesh and edges for a 2-D mesh.

\section{Sections}

\begin{description}
\item[Fiber dimension] Fiber dimension refers to the number of values at a point.
\item[Section/Field] Sections generally refer to the Sieve data
  structure, whereas Fields refer to a vector field. The PyLith
  Field class stores a vector field over vertices or cells.
\item[Chart] Describes the points associated with a section.
\item[Atlas] Describes the fiber dimension at each point in a section.
\end{itemize}

\begin{description}
\item[IConstantSection] All values are identical. The chart is an
  interval (min, max), the fiber dimension is 1, and the values are
  stored as an array of size 2 (value, defaultValue).
\item[IUniformSection] Fiber dimension is uniform and set at compile
  time via a template parameter. The chart is an interval (min, max),
  and the values are storeed as an array of size
  (max-min)*fiberDim. The atlas is an
  IConstantSection. restrictPoint() simply involves calculating the
  offset from the first value via (p-min)*fiberDim.
\item[ISection] The fiber dimension is independent at each point. The
  chart is an interval (min, max) and the values are stored as an
  array with a size that exactly matches the sum of the fiber
  dimension at the points. The atlas is an
  IUniformSection. restrictPoint() involves looking up the offset in
  the atlas via atlas.restrictPoint().
\item[IGeneralSection] Adds constraints and spaces to an ISection. The
  atlas is an IUniformSection. Constraints use an additional atlas and
  spaces use two additional atlases (one for the values and one for
  the constraints).
\end{description}

\end{document}
