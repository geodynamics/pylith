\section{Mesh Memory}

    The \code{ISieve} maintains arrays for both cones and supports, and also for cone orientations.
$$
  2 (K C + V+C) + K C
$$
where $K$ is the number of vertices per cell, $C$ is the number of cells, and $V$ is the number of vertices. We also
store coordinates, which takes
$$
  D V
$$

    Stratification uses a \code{Sieve} as a label to associate depths and heights with sieve points. There is an arrow
for each mesh point for both height and depth. Thus we get
$$
  2 A (V + C)
$$
where $A$ is the size of an arrow in bytes. Materials are also stored this way, but in PyLith are mapped only to cells.

    For an IGeneralSection, the data take up only the minimum size. The atlas is an IUniformSection so that the data
take up $2 max(V, C)$ integers, and its atlas is an IConstantSection. We also have a BC ISection, which has an
IUniformSection Atlas. Thus, for a group we expect
$$
  2 max(V, C) + 2 max(V, C) + max(V, C)
$$

\subsection{Savings}

\begin{itemize}
  \item We could eliminate the redundant classification of height since we can always use depth in its place.

  \item Groups could be IUniformSections since they have only 1 value per point.
\end{itemize}

\subsection{Strike-slip Benchmark}
    For the 1000m benchmark, we have
\begin{itemize}
  \item $D = 3$
  \item $K = 4$
  \item $C = 13824$
  \item $V = 15625$
  \item $A = 40$
  \item $G = 6$
\end{itemize}
so we get 1,274,144 bytes for the mesh, 2,355,920 for stratification, 552,960 for materials, and 1,500,000 for groups.