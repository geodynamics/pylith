\documentclass{article}[10pt]
\usepackage{amsmath}

\renewcommand{\matrix}[1]{\bar{#1}}
\newcommand{\Adiag}{\ensuremath{A_\mathit{diag}}}

% ------------------------------------------------------------------
% Basic page layout
\setlength{\textheight}{9.0in}
\setlength{\textwidth}{7.0in}
\setlength{\topmargin}{-20pt}
\setlength{\headheight}{16pt}
\setlength{\headsep}{4pt}
\setlength{\oddsidemargin}{-0.25in}
\setlength{\footskip}{24pt}
%
\setlength{\floatsep}{12pt}
\setlength{\textfloatsep}{12pt}
\setlength{\intextsep}{12pt}
\setlength{\abovecaptionskip}{0pt}
\setlength{\belowcaptionskip}{6pt}
%
\setlength{\parindent}{0.25in}
\setlength{\parskip}{6pt}
%
% ------------------------------------------------------------------
% Font Sizes
\usepackage{times}
%
% ======================================================================
\begin{document}

% ----------------------------------------------------------------------
\section{Schur Complement Preconditioner}

We have a Jacobian of the form
\begin{equation}
  J = \left( \begin{array}{cc}
    A & C^T \\
    C & 0
  \end{array} \right).
\end{equation}
Using the Schur complement, we want to form the preconditioner
\begin{equation}
  P = \left( \begin{array}{cc}
    A^{-1} + A^{-1} C^T (-C A^{-1} C^T)^{-1} C A^{-1} &
    -A^{-1} C^T (-C A^{-1} C^T)^{-1} \\
    (C A^{-1} C^T)^{-1} C A^{-1} & -(C A^{-1} C^T)^{-1}
  \end{array} \right).
\end{equation}
We approximate this preconditioner by using only the diagonal terms of $A$ and $C A^{-1} C^T$:
\begin{equation}
  Pd = \left( \begin{array}{cc}
    \Adiag^{-1} + \Adiag^{-1} C^T (-C \Adiag^{-1} C^T)^{-1} C \Adiag^{-1} &
    \Adiag^{-1} C^T (C \Adiag^{-1} C^T)^{-1} \\
    (C \Adiag^{-1} C^T)^{-1} C \Adiag^{-1} & -(C \Adiag^{-1} C^T)^{-1}
  \end{array} \right).
\end{equation}


% ======================================================================
\end{document}