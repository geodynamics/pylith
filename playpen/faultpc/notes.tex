\documentclass{article}[10pt]
\usepackage{amsmath}

\renewcommand{\matrix}[1]{\bar{#1}}
\newcommand{\Adiag}{\ensuremath{A_\mathit{diag}}}

% ------------------------------------------------------------------
% Basic page layout
\setlength{\textheight}{9.0in}
\setlength{\textwidth}{7.0in}
\setlength{\topmargin}{-20pt}
\setlength{\headheight}{16pt}
\setlength{\headsep}{4pt}
\setlength{\oddsidemargin}{-0.25in}
\setlength{\footskip}{24pt}
%
\setlength{\floatsep}{12pt}
\setlength{\textfloatsep}{12pt}
\setlength{\intextsep}{12pt}
\setlength{\abovecaptionskip}{0pt}
\setlength{\belowcaptionskip}{6pt}
%
\setlength{\parindent}{0.25in}
\setlength{\parskip}{6pt}
%
% ------------------------------------------------------------------
% Font Sizes
\usepackage{times}
%
% ======================================================================
\begin{document}

% ----------------------------------------------------------------------
\section{Schur Complement Preconditioner}

We have a Jacobian of the form
\begin{equation}
  J = \left( \begin{array}{cc}
    A & C^T \\
    C & 0
  \end{array} \right).
\end{equation}
We provide PETSc with $P$ so that it can create $P^{-1}$. Using the
field split preconditioner, we form
\begin{equation}
  P = \left( \begin{array}{cc}
    A_\mathit{ml} & 0 \\
    0 & P_f
  \end{array} \right),
\end{equation}
where we use the ML package to form $A_\mathit{ml}$ and we create a
custom matrix for the portion of the preconditioner associated with
the Lagrange constraints, $P_f$. Let $n$ be the number of conventional
degrees of freedom and $l$ be the number of Lagrange constraints. This
means $A$ and $P$ are $(n+l) \times (n+l)$, $A$ and $A_\mathit{ml}$
are $n \times n$, $C$ is $l \times n$, and $P_f$ is $l \times l$.

Using the {\tt multiplicative} field split type, PETSc will form
$P^{-1}$ as
\begin{equation}
  P^{-1} = \left( \begin{array}{cc}
    A_\mathit{ml}^{-1} & -A_\mathit{ml}^{-1} C^T \\
    0 & C A_\mathit{ml}^{-1} C^T P_f^{-1}
  \end{array} \right).
\end{equation}

\subsection*{QUESTIONS FOR MATT}
We are solving $J u = b$.
\begin{itemize}
\item Does PETSc solve the right-preconditioned system using
  \begin{gather}
    (J P^{-1}) v = b \\
    u = P^{-1} v
  \end{gather}
  or the left-preconditioned system using
  \begin{gather}
    P^{-1}(J u - b) = 0.
  \end{gather}
\item
  Do we want to reduce the conditioner number (ratio of largest to
  smallest eigenvalue) of $J P^{-1}$ or $P^{-1} J$?
\end{itemize}
      

\subsection*{Old, wrong stuff}

Using the Schur complement, we want to form the preconditioner
\begin{equation}
  P = \left( \begin{array}{cc}
    A^{-1} + A^{-1} C^T (-C A^{-1} C^T)^{-1} C A^{-1} &
    -A^{-1} C^T (-C A^{-1} C^T)^{-1} \\
    (C A^{-1} C^T)^{-1} C A^{-1} & -(C A^{-1} C^T)^{-1}
  \end{array} \right).
\end{equation}
We approximate this preconditioner by using only the diagonal terms of $A$ and $C A^{-1} C^T$:
\begin{equation}
  Pd = \left( \begin{array}{cc}
    \Adiag^{-1} + \Adiag^{-1} C^T (-C \Adiag^{-1} C^T)^{-1} C \Adiag^{-1} &
    \Adiag^{-1} C^T (C \Adiag^{-1} C^T)^{-1} \\
    (C \Adiag^{-1} C^T)^{-1} C \Adiag^{-1} & -(C \Adiag^{-1} C^T)^{-1}
  \end{array} \right).
\end{equation}


% ======================================================================
\end{document}