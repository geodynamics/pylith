% ----------------------------------------------------------------------
\section{Elasticity with Infinitesimal Strain and No Faults}

We begin with the elasticity equation including the inertial term,
\begin{gather}
  \label{eqn:elasticity:strong:form}
  \rho \frac{\partial^2\vec{u}}{\partial t^2} - \vec{f}(\vec{x},t) - \tensor{\nabla} \cdot 
\tensor{\sigma}
(\vec{u}) = \vec{0} \text{ in }\Omega, \\
%
  \label{eqn:bc:Neumann}
  \tensor{\sigma} \cdot \vec{n} = \vec{\tau}(\vec{x},t) \text{ on }\Gamma_\tau, \\
%
  \label{eqn:bc:Dirichlet}
  \vec{u} = \vec{u}_0(\vec{x},t) \text{ on }\Gamma_u,
\end{gather}
where $\vec{u}$ is the displacement vector, $\rho$ is the mass
density, $\vec{f}$ is the body force vector, $\tensor{\sigma}$ is the
Cauchy stress tensor, $\vec{x}$ is the spatial coordinate, and $t$ is
time. We specify tractions $\vec{\tau}$ on boundary $\Gamma_\tau$, and
displacements $\vec{u}_0$ on boundary $\Gamma_u$. Because both $\vec{\tau}$
and $\vec{u}$ are vector quantities, there can be some spatial overlap
of boundaries $\Gamma_\tau$ and $\Gamma_u$; however, a degree of freedom at
any location cannot be associated with both prescribed displacements
(Dirichlet) and traction (Neumann) boundary conditions simultaneously.

\begin{table}[htbp]
  \caption{Mathematical notation for elasticity equation with
    infinitesimal strain.}
  \label{tab:notation:elasticity}
  \begin{tabular}{lcp{3in}}
    \toprule
    {\bf Category} & {\bf Symbol} & {\bf Description} \\
    \midrule
    Unknowns & $\vec{u}$ & Displacement field \\
    & $\vec{v}$ & Velocity field \\
    Derived quantities & $\tensor{\sigma}$ & Cauchy stress tensor \\
                   & $\tensor{\epsilon}$ & Cauchy strain tensor \\
    Common constitutive parameters & $\rho$ & Density \\
  & $\mu$ & Shear modulus \\
  & $K$ & Bulk modulus \\
Source terms & $\vec{f}$ & Body force per unit volume, for example $\rho \vec{g}$ \\
    \bottomrule
  \end{tabular}
\end{table}


\subsection{Quastistatic}

If we neglect the inertial term
($\rho \frac{\partial \vec{v}}{\partial t} \approx \vec{0}$), then
time dependence only arises from history-dependent constitutive
equations and boundary conditions. Our solution vector is the
displacement vector and the elasticity equation reduces to
\begin{gather}
  \label{eqn:elasticity:strong:form:quasistatic}
  \vec{f}(\vec{x},t) + \tensor{\nabla} \cdot \tensor{\sigma}(\vec{u}) = \vec{0} \text{ in }\Omega, \\
%
  \tensor{\sigma} \cdot \vec{n} = \vec{\tau}(\vec{x},t) \text{ on }\Gamma_\tau, \\
%
  \vec{u} = \vec{u}_0(\vec{x},t) \text{ on }\Gamma_u.
\end{gather}
Because we will use implicit time stepping, we place all of the terms
in the elasticity equation on the LHS. We create the weak form by
taking the dot product with the trial function $\trialvec[u]$ and
integrating over the domain:
\begin{equation}
    \int_\Omega \trialvec[u] \cdot \left( \vec{f}(t) + \tensor{\nabla}
      \cdot \tensor{\sigma} (\vec{u}) \right) \, d\Omega = 0. 
\end{equation}
Using the divergence theorem and incorporating the Neumann boundary conditions, we have
\begin{equation}
% 
  \int_\Omega \trialvec[u] \cdot \vec{f}(\vec{x},t) + \nabla \trialvec[v] : -\tensor{\sigma}(\vec{u}) \, d\Omega
  + \int_{\Gamma_\tau} \trialvec[v] \cdot \vec{\tau}(\vec{x},t) \, d\Gamma = 0
\end{equation}

\subsubsection{Residual Pointwise Functions}

Identifying $F(t,s,\dot{s})$ and $G(t,s)$, we have
\begin{align}
  % Fu
  F^u(t,s,\dot{s}) &=  \int_\Omega \trialvec[u] \cdot \eqnannotate{\vec{f}(\vec{x},t)}{\vec{f}^u_0} + \nabla \trialvec[u] : \eqnannotate{-\tensor{\sigma}(\vec{u})}{\tensor{f^u_1}} \, d\Omega
  + \int_{\Gamma_\tau} \trialvec[u] \cdot \eqnannotate{\vec{\tau}(\vec{x},t)}{\vec{f}^u_0} \, d\Gamma, \\
  % Gu
  G^u(t,s) &= 0
\end{align}
Note that we have multiple $\vec{f}_0$ functions, each associated with
a trial function and an integral over a different domain or
boundary. Each material and boundary condition (except Dirichlet)
contribute pointwise functions. The integral over the domain $\Omega$
is subdivided into integrals over the materials and the integral over
the boundary $\Gamma_\tau$ is subdivided into integrals over the
Neumann boundaries. Each bulk constitutive model provides a different
pointwise function for the stress tensor
$\tensor{\sigma}(\vec{u})$. With $G=0$ it is clear that we have a
formulation that will use implicit time stepping algorithms.

\subsubsection{Jacobian Pointwise Functions}

We only have a Jacobian for the LHS:
\begin{align}
  J_F^{uu} &= \frac{\partial F^u}{\partial u} = \int_\Omega \nabla \trialvec[u] : 
\frac{\partial}{\partial u}(-\tensor{\sigma}) \, d\Omega 
  = \int_\Omega \nabla \trialvec[u] : -\tensor{C} : \frac{1}{2}(\nabla + \nabla^T)\basisvec[u] 
\, d\Omega 
  = \int_\Omega \trialscalar[u]_{i,k} \, \eqnannotate{\left( -C_{ikjl} \right)}{J_{f3}^{uu}} \, \basisscalar[u]_{j,l}\, d\Omega.
\end{align}


\subsection{Dynamic}

For compatibility with PETSc TS algorithms, we want to turn
the second order equation~\vref{eqn:elasticity:strong:form} into two first order
equations. We introduce the velocity as a unknown,
$\vec{v}=\frac{\partial u}{\partial t}$, which leads to
\begin{align}
  % Displacement-velocity
  \frac{\partial \vec{u}}{\partial t} &= \vec{v} \text{ in } \Omega, \\
  % Elasticity
  \rho(\vec{x}) \frac{\partial\vec{v}}{\partial t} &= \vec{f}(\vec{x},t) + \tensor{\nabla} \cdot \tensor{\sigma}(\vec{u}) \text{ in } \Omega, \\
  % Neumann
  \tensor{\sigma} \cdot \vec{n} &= \vec{\tau}(\vec{x},t) \text{ on } \Gamma_\tau, \\
  % Dirichlet
  \vec{u} &= \vec{u}_0(\vec{x},t) \text{ on } \Gamma_u.
\end{align}
We create the weak form by taking the dot product with the trial
function $\trialvec[u]$ or $\trialvec[v]$ and
integrating over the domain:
\begin{gather}
  % Displacement-velocity
  \int_\Omega \trialvec[u] \cdot \frac{\partial \vec{u}}{\partial t} \, d\Omega = 
  \int_\Omega \trialvec[u] \cdot \vec{v} \, d\Omega, \\
  % Elasticity
    \int_\Omega \trialvec[v] \cdot \rho(\vec{x}) \frac{\partial \vec{v}}{\partial t} \, d\Omega 
 = \int_\Omega \trialvec[v] \cdot \left( \vec{f}(t) + \tensor{\nabla} \cdot \tensor{\sigma} (\vec{u}) \right) \, d\Omega.
\end{gather}
Using the divergence theorem and incorporating the Neumann boundaries, we can rewrite the second equation as
\begin{equation}
% 
  \int_\Omega \trialvec[v] \cdot \rho(\vec{x}) \frac{\partial \vec{v}}{\partial t} \, d\Omega
  = \int_\Omega \trialvec[v] \cdot \vec{f}(\vec{x},t) + \nabla \trialvec[v] : -\tensor{\sigma}(\vec{u}) \, d\Omega
  + \int_{\Gamma_\tau} \trialvec[v] \cdot \vec{\tau}(\vec{x},t) \, d\Gamma.
\end{equation}

For explicit time stepping, we want $F(t,s,\dot{s})=\dot{s}$, so we
solve an augmented system in which we multiply the RHS residual
function by the inversion of the lumped LHS Jacobian,
\begin{gather}
  F^*(t,s,\dot{s}) = G^*(t,s) \text{, where} \\
  F^*(t,s,\dot{s}) = \dot{s} \text{ and} \\
  G^*(t,s) = J_F^{-1} G(t,s).
\end{gather}
With the augmented system, we have
\begin{gather}
  % Displacement-velocity
  \frac{\partial \vec{u}}{\partial t}  = M_u^{-1} \int_\Omega \trialvec[u] \cdot \vec{v} \, d\Omega, \\
  % Elasticity
  \frac{\partial \vec{v}}{\partial t} = M_v^{-1} \int_\Omega \trialvec[v] \cdot \left( \vec{f}(t) + \tensor{\nabla} \cdot \tensor{\sigma} (\vec{u}) \right) \, d\Omega, \\
  % Mu
  M_u = \mathit{Lump}\left( \int_\Omega \trialscalar[u]_i \delta_{ij} \basisscalar[u]_j \, d\Omega \right), \\
  % Mv
  M_v = \mathit{Lump}\left( \int_\Omega \trialscalar[v]_i \rho(\vec{x}) \delta_{ij} \basisscalar[v]_j \, d\Omega \right).
\end{gather}

\subsubsection{Residual Pointwise Functions}

With explicit time stepping the PETSc TS assumes the LHS is
$\dot{s}$, so we only need the RHS residual functions:
\begin{align}
  % Gu
  G^u(t,s) &= \int_\Omega \trialvec[u] \cdot \eqnannotate{\vec{v}}{\vec{g}^u_0} \, d\Omega, \\
  % Gv
  G^v(t,s) &=  \int_\Omega \trialvec[v] \cdot \eqnannotate{\vec{f}(\vec{x},t)}{\vec{g}^v_0} + \nabla \trialvec[v] : \eqnannotate{-\tensor{\sigma}(\vec{u})}{\tensor{g^v_1}} \, d\Omega
  + \int_{\Gamma_\tau} \trialvec[v] \cdot \eqnannotate{\vec{\tau}(\vec{x},t)}{\vec{g}^v_0} \, d\Gamma,
\end{align}
In the second equation these are the same pointwise functions as in the
quasistatic case, only now they are on the RHS instead of the LHS.


\subsubsection{Jacobian Pointwise Functions}

These are the pointwise functions associated with $M_u$ and $M_v$ for
computing the lumped LHS Jacobian. We premultiply the RHS residual
function by the inverse of the lumped LHS Jacobian while
$s_\mathit{tshift}$ remains on the LHS with $\dot{s}$. As a result, we
use LHS Jacobian pointwise functions, but set $s_\mathit{tshift}=1$.
The LHS Jacobians are:
\begin{align}
  % J_F uu
  M_u = J_F^{uu} &= \frac{\partial F^u}{\partial u} + s_\mathit{tshift} \frac{\partial F^u}{\partial \dot{u}} =
             \int_\Omega \trialscalar[u]_i \eqnannotate{s_\mathit{tshift} \delta_{ij}}{J^{uu}_{f0}} \basisscalar[u]_j  \, d\Omega, \\
  % J_F vv
  M_v = J_F^{vv} &= \frac{\partial F^v}{\partial v} + s_\mathit{tshift} \frac{\partial F^v}{\partial \dot{v}} =
             \int_\Omega \trialscalar[v]_i \eqnannotate{\rho(\vec{x}) s_\mathit{tshift} \delta_{ij}}{J ^{vv}_{f0}} \basisscalar[v]_j \, d\Omega
\end{align}


% End of file